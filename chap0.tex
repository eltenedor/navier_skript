\setcounter{chapter}{-1}
\chapter{Motivation}

Die Navier-Stokes-Gleichungen sind ein Modell zur Beschreibung kompressibler, viskoser und Newton'scher Fluide (Flüssigkeiten/Gase).
Viskos heißt, dass die Flüssigkeit aufgrund innerer Reibungseffekte \emph{zähe} ist.
Newton'sch ist z.B. Luft oder Wasser, aber nicht z.B. Blut, Ketchup oder Speisestärke in Wasser.

Die Navier-Stokes-Gleichungen sind ein System nicht-lineare pDGLen und sind gegeben durch:
\begin{align*}
  \rho(\partial_t u + (u \cdot \nabla)u) - \mu \Delta u - (\lambda + \mu) \nabla \div(u) + \nabla \pi &= f \quad t \in (0,T), x \in \Omega \\
  \partial_t \rho + \div(u\rho) &= 0 \quad t \in (0,t), x \in \Omega.
\end{align*}

Hierbei sind
\begin{itemize}
  \item $u \colon [0,T) \times \Omega \to \R^3$, Geschwindigkeit
  \item $\pi \colon [0,T) \times \Omega \to \R$, Druck
  \item $\rho \colon [0,T) \times \Omega \to \R$, Dichte
  \item $f \colon [0,T) \times \Omega \to \R^3$, (bekannte) Volumenkraftdichte
  \item $\mu > 0$, dynamische Viskosität
  \item $\lambda$, Volumenviskosität, wobei $2\mu + 3\lambda > 0$.
\end{itemize}

Das System wird durch Anfangs- und Randbedingungen komplementiert:
\begin{align*}
  u(0) &= a \quad x \in \Omega \\
  u|_{\partial\Omega} &= 0 \quad t \in (0,T)
\end{align*}
Die Bedingung $u|_{\partial\Omega} = 0$ heißt \emph{no-slip} Randbedingung oder auch \emph{Dirichlet}-Randbedingung.

Wir stellen uns ein Testvolumen $V$ vor, welches sich mit der Strömung mitbewegt, wobei $V(t) = \phi(t, V(0))$ gelten soll. Hierbei ist $\phi(t,x)$ die Position des Fluidteilchens zum Zeitpunkt $t$, welches zum Zeitpunkt $0$ bei $x$ war.
Die Abbildung $t \mapsto \phi(t,x)$ heißt \emph{Bahnkurve} von $x$, also gilt $\frac{\d{}}{\d t} \phi(t,x) = u(t, \phi(t,x))$ (\emph{Lagrange Koordinaten})

Wir rechnen
\begin{align*}
  \partial_t \det J_\phi
  &= \partial_t \det (\nabla \phi_1, \nabla \phi_2, \nabla \phi_3) \\
  &= \det (\nabla \partial_t \phi_1, \nabla \phi_2, \nabla \phi_3) + \cdots  +  \det(\nabla\phi_1, \nabla\phi_2, \nabla\partial_t \phi_3) \\
  &= \det( \sum_{i = 1}^3 \partial_i u_1 \nabla \phi_i, \nabla\phi_2, \nabla \phi_3) + \cdots + \det(\nabla\phi_1, \nabla\phi_2, \sum_{i = 1}^3 \partial_i u_3  \nabla \phi_i) \\
  &= \div(u)(t,\phi(t,x)) \cdot \det J_\phi.
\end{align*}
Da $\det J_\phi(0,x) = 1$, folgt 
$$
\det J_\phi(t,x) = \exp\Big( \int_0^t \div(u)(s,\phi(s,x))\d x \Big).
$$
Damit bestimmt $\div(u)$ die Kompressibilität, denn es gilt:
Das Fluid ist inkompressibel, wenn folgende äquivalente Bedingungen erfüllt sind:
\begin{enumerate}[1)]
  \item $|V(0)| = |V(t)| = \int_{\phi(t,V(0)} 1 \d y = \int_{V(0)} \det J_\phi(t,x) \d x$
  \item $\det J_\phi(t,x) = 1$ für alle $t, x$.
  \item $\int_0^t \div(u)(s,\phi(s,x)) \d s = 0$ für alle $t,x$.
  \item $\div(u)(t,\phi(t,x)) = 0$ für alle $t,x$.
\end{enumerate}
Da $\Omega$ komplett mit Fluid ausgefüllt ist (d.h. $x \mapsto \phi(t,0)$ surjektiv) folgt: Das Fluid ist inkompressibel genau dann, wenn $\div(u) = 0$ für alle $t \in (0,T), x \in \Omega$ gilt.

Wir definieren außerdem ein Fluid als \emph{homogen}, falls $\rho(t,x) = \rho(t,y)$ für alle $t$ und $x,y \in \Omega$ gilt.

Für die Bernoulli-Gleichung gilt dann
$$
\partial_t \rho = -\div(\rho u) = -\rho \div(u) - \nabla \rho \cdot u = 0,
$$
falls das Fluid inkompressibel und homogen ist.

Teile nun die Navier-Stokes-Gleichungen durch $\rho$, ersetze $\frac{\pi}{\rho}$ durch $\pi$ und $\frac{f}{\rho}$ durch $\rho$.
Damit erhält man die Navier-Stokes-Gleichungen für homogene, inkompressible Fluide:
\begin{align*}
  \partial_t u - \nu \Delta u + (u \cdot \nabla) u + \nabla \pi &= f &t \in (0,T), x \in \Omega \\
  \div(u) &= 0 &t \in (0,T), x \in \Omega\\
  u(0) &= a &x \in \Omega \\
  u|_{\partial\Omega} &= 0  &t \in (0,T),
\end{align*}
wobei $\nu = \frac{\mu}{\rho}$ die \emph{kinematische Viskosität} genannt wird.

Die Linearisierung ist die Stokes-Gleichung:
\begin{align*}
  \partial_t u - \nu \Delta u + \nabla \pi &= f & t \in (0,T), x \in \Omega \\
  \div(u) &= 0 &t \in (0,T), x \in \Omega \\
  u(0) &= a & x \in \Omega) \\
  u|_{\partial\Omega} &= 0 & t \in (0,T).
\end{align*}

In dieser Vorlesung werden wir das Anfagswertproblem, d.h. $f = 0$, dieser Systeme betrachten. 
Außerdem setzen wir im Folgenden $\nu = 1$.

\section{Existenz von Lösungen -- skizzenhafte Darstellung}

\subsubsection*{1. Schritt}

Konstruiere Operator $A$ mit Definitionsbereich $\DD(A)$ auf geeignetem Banachraum $X$, sodass $u \in \DD(A)$ gilt, genau dann, wenn $\div(u) = 0$, $u|_{\partial \Omega}$ gelten und ein $\pi$ existiert mit $Au = -\Delta u + \nabla \pi$.

Die Stokes-Gleichungen können damit als gewöhnliche DGL auf $X$ interpretiert werden:
\begin{align*}
  u'(t) &= Au(t) \quad 0 < t < T\\
  u(0) &= a.
\end{align*}

Entwickle nun einen Formalismus, um diese gewöhnliche DGL zu lösen. Dazu gebe dem Ausdruck ``$\, \e^{tA} a \, $'' einen Sinn.

\subsubsection*{2.Schritt}

Als Nächstes würde man gerne die Navier-Stokes-Gleichungen als Integralgleichung schreiben, d.h.
$$
 u(t) = \e^{-tA} a - \int_0^t \e^{-(t - s)A} (u(s) \cdot \nabla) u(s) \d x.
$$

\textbf{ACHTUNG}: $(u(s) \cdot \nabla) u(s)$ muss nicht in $X$ liegen. Darum: Konstruiere Projektion $\PP$ mit Bild $X$, sodass das Bild $I - \PP$ Gradientenfelder sind. Dies führt zur Helmholtz-Projektion.

\subsubsection*{3. Schritt}

Löse die auf $X$ projizierten Navier-Stokes-Gleichungen
\begin{align*}
  u'(t) + Au(t) &= -\PP(u(t) \cdot \nabla) u(t) \quad t \in (0,T) \\
  u(0) &= a,
\end{align*}
indem man die Integralgleichung
\begin{equation}
  u(t) = \e^{-tA} a - \int_0^t \e^{-(t - s)A} \PP(u(s) \cdot \nabla)u(s) \d s \tag{$\ast$}
\end{equation}
löst.
Lösungen dieses Typs bezeichnet man auch als \emph{milde Lösung}.
Gehe hierzu iterativ vor: Starte mit linearem Problem
$$
u_0(t) \coloneqq \e^{-tA} a.
$$
Löse danach (formal) die Gleichung
\begin{align*}
  u'_{j + 1}(t) + A u_{j + 1} &= \PP(u_j(t) \cdot \nabla) u_j(t) \quad t \in (0,T) \\
  u_{j + 1}(0) &= a,
\end{align*}
indem $u_{j + 1}$ definiert wird durch:
$$
u_{j + 1}(t) \coloneqq u_0(t) - \int_0^t \e^{-(t - s)A} \PP(u_j(s) \cdot \nabla) u_j(s) \d s.
$$

Falls $(u_j)_{j \in \N}$ (im geeigneten Sinne) konvergiert, so löst die Grenzfunktion $u$ die Gleichung ($\ast$)

Um Konvergenz zu zeigen, werden benötigt:
\begin{itemize}
  \item Abbildungseigenschaften der Halbgruppe $\e^{tA}$;
  \item ``Kleinheit'' entweder von $a$ oder $T$.
\end{itemize}

Im 1. Fall ist $u_0$ sehr klein.
Da der Integralterm quadratisch in $u_0$ ist folgt, das der Integralterm sogar sehr, sehr klein ist. 
Damit ist der Abstand zwischen $u_0$ und $u_1$ auch sehr, sehr klein und $u_1$ sehr klein. Dies geht für $u_2$ genauso weiter, bis zur Konvergenz.

Im 2. Fall ist $u_0$ ``fast'' konstant in $t$ falls $T$ klein ist. 
Damit ist der Integrand ``fast'' konstant und folglich der Integralterm sehr, sehr klein, falls $T$ sehr, sehr klein ist.
Entsprechend ist der Abstand zwischen $u_0$ und $u_1$ sehr, sehr klein und $u_1$ ``fast'' konstant.
Dies geht für $u_2$ genau so weiter, bis zur Konvergenz.
