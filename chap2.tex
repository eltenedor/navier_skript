\chapter{Die Stokes-Gleichungen auf \texorpdfstring{$\Ell_\sigma^2$}{L_sigma\textasciicircum 2}}

In diesem Kapitel untersuchen wir Lösungen der instationären Stokes-Gleichungen
$$
\begin{cases}
  \partial_t u - \Delta u + \nabla p &= 0, \quad x \in \Omega, t > 0 \\
  \div u &= 0, \quad x \in \Omega, t > 0 \\
  u(0) &= a, \quad x \in \Omega \\
  u &= 0, \quad x \in \partial\Omega, t > 0,
\end{cases}
$$
wobei $a \in \Ell^2(\Omega; \C^d)$, $d \geq 2$ und ``$\div(a) = 0$'' gelten soll.


\section{Der Stokes-Operator auf \texorpdfstring{$\Ell_\sigma^2$}{L_sigma\textasciicircum 2}}

Sei $\Omega \subset \R^d$, $d \geq 2$ und $1 < p < \infty$.
Definiere
\begin{align*}
  \CC_{\cc,\sigma}^\infty(\Omega) &\coloneqq \{ \varphi \in \CC_\mathrm{c}^\infty(\Omega; \C^d) \colon \div(\varphi) =0 \}.
  \intertext{ Weiterhin sei }
  \Ell^p_\sigma(\Omega) &\coloneqq \overline{\CC_{\cc,\sigma}^\infty(\Omega)}^{\Ell^p} \quad\text{mit}\quad \|\cdot\|_{\Ell^p_\sigma} \coloneqq \|\cdot\|_{\Ell_{}^p}
  \intertext{ und }
  \WW_{0,\sigma}^{1,p} &\coloneqq \overline{\CC_{\mathrm{c},\sigma}^\infty(\Omega)}^{\WW^{1,p}} \quad\text{mit}\quad \|\cdot\|_{\WW_{0,\sigma}^{1,p}} \coloneqq \|\cdot\|_{\WW_{}^{1,p}}.
\end{align*}
Im Falle $p= 2$ schreibt man auch $\HH_{0,\sigma}^1(\Omega)$ für $\WW_{0,\sigma}^{1,2}(\Omega)$.
Um den Stokes-Operator zu definieren, definiere folgende Sesquilinearform
$$
a \colon \HH_{0,\sigma}^1(\Omega) \times \HH_{0,\sigma}^1(\Omega) \to \C, \quad (u,v) \mapsto \int_\Omega \nabla u \cdot \overline{\nabla v} \d x = \sum_{i,j=1}^d \int_\Omega \partial_i u_j \, \partial_i \overline{v_j} \d x.
$$

\begin{defn}
  Der \emph{Stokes-Operator} $A$ auf $\Ell_\sigma^2(\Omega)$ ist gegeben durch
  \begin{align*}
  \DD(A) &\coloneqq \left\{ u \in \HH_{0,\sigma}^1(\Omega) \colon \exists!  f \in \Ell_\sigma^2(\Omega) \colon \forall v \in \HH_{0,\sigma}^1(\Omega) \colon a(u,v) = \int_\Omega f \cdot \overline v \d x\right\}, \\
  Au &\coloneqq f,
\end{align*}
wobei $f$ und $u$ durch $\DD(A)$ gegeben sind.
\end{defn}

\begin{prop}
  Der Stokes-Operator auf $\Ell_\sigma^2(\Omega)$ ist abgeschlossen und dicht definiert.
\end{prop}

\begin{proof}
  Zur Abgeschlossenheit: Sei $u_n \in \DD(A)$ mit $u_n \to u$ in $\Ell^2_\sigma(\Omega)$ und $f_n \coloneqq A u_n \to f$ in $\Ell^2_\sigma(\Omega)$.
  Dann
  $$
  \|\nabla(u_n - u_m)\|_{\Ell^2}^2 
  = a(u_n - u_m, u_n - u_m)
  = \int_\Omega (f_n - f_m) \overline{(u_n - u_m)} \d x
  \overset{\text{Hölder}}{\to} 0, \quad\text{für } u,m \to \infty.
  $$
  Folglich ist $(u_n)_{n \in \N}$ eine Cauchy-Folge in $\HH_{0,\sigma}^1(\Omega)$ und damit $u \in \HH_{0,\sigma}^1(\Omega)$.
  Hiermit ergibt sich
  $$
  a(u,v) 
  = \lim_{n \to \infty} a(u_n,v)
  = \lim_{n \to \infty} \int_\Omega f_n \cdot \overline{v} \d x
  = \int_\Omega f \cdot \overline v \d x,
  $$
 für alle $v \in \HH_{0,\sigma}^1(\Omega)$.

  Zur Dichtheit: Für $u \in \CC_{\cc,0}^\infty(\Omega)$, $v \in \HH_{0,\sigma}^1(\Omega)$ gilt
 $$
 a(u,v) = -\int_\Omega \Delta u \cdot \overline v \d x.
 $$
 Aus dem Satz von Schwartz folgt $\Delta u \in \CC_{c,\sigma}^\infty(\Omega)$ und damit $\CC_{c,\sigma}^\infty(\Omega) \subset \DD(A)$.
\end{proof}

\begin{lem*}[Lax-Milgram]
   Sei $H$ ein Hilbertraum über $\C$ und $b \colon H \times H \to \C$ eine Sesquilinearform, die stetig und koerziv ist, d.h., es existieren $\alpha, C > 0$, sodass 
   \begin{alignat*}{2}
     |b(u,v)| &\leq C\, \|u\|_H \|v\|_H, \quad && \text{für alle }u,v \in H, \\
     |b(u,v)| &\geq \alpha \, \|u\|_H^2, \quad&& \text{für alle } u \in H.
     \intertext{Dann existiert für jedes $F \in H^*$ ein eindeutiges $u \in H$ mit }
     b(u,v) &= F[v], \quad&&\text{für alle } v \in H.
   \end{alignat*}
\end{lem*}

\begin{prop}
  Sei $A$ der Stokes-Operator auf $\Ell_\sigma^2(\Omega)$, wobei $\Omega \subset \R^d, d \geq 2$ ein beschränktes Gebiet ist.
  Dann ist $0 \in \rho(A)$.
\end{prop}

\begin{proof}
  Für $f \in \Ell_\sigma^2(\Omega)$ ist $v \mapsto \int_\Omega f \cdot \overline v \d x \in \HH_{0,\sigma}^1(\Omega)^*$ (Antidualraum).
  Weiterhin ist 
  $$
  a \colon \HH_{0,\sigma}^1(\Omega) \times \HH_{0,\sigma}^1(\Omega) \to \C,\quad (u,v) \mapsto \int_\Omega \nabla u \cdot \overline{\nabla v} \d x
  $$ 
  stetig.
  Außerdem folgt mit der Poincar\'e-Ungleichung
  $$
  |a(u,u)| = \|\nabla u\|_{\Ell^2}^2 \geq \frac{1}{2} \|\nabla u\|_{\Ell^2}^2 + \frac{1}{2 c^2} \|u\|_{\Ell^2}^2
  $$
  und damit die Koerzivität von $a$.
  Das Lemma von Lax-Milgram liefert sodann, dass genau ein $u \in \HH_{0,\sigma}^1(\Omega)$ mit $a(u,v) = \int_\Omega f \cdot \overline v \d x$ für alle $v \in \HH_{0,\sigma}^1(\Omega)$ existiert.
  Daraus folgt schließlich $u \in \DD(A)$ mit $A u = f$ und $0 \in \rho(A)$.
\end{proof}

\begin{lem}
  \label{lem:inverseTriangle}
  Seien $\theta, \phi \in [0,\pi)$ mit $\theta + \phi < \pi$.
    Dann existiert $C = C(\phi,\theta) > 0$, sodass für alle $w \in \Sec_\theta, z \in \Sec_\phi$ gilt
    $$
    |w| + |z| \leq C \,|w + z|.
    $$
\end{lem}

\begin{proof}
  Eigentlich Übung. Wir rechnen
  \begin{align*}
    |w + z|^2 
    &= (w + z)(\overline w + \overline z)  
    = w \overline w + w \overline z + z \overline w + z \overline z 
    = |w|^2 + |z|^2 + 2 \operatorname{Re}(w \overline z) \\
    &= |w|^2 + |z|^2 + 2 \cos (\phi + \theta) |w \overline z|
    = |w|^2 + |z|^2 + 2|w||z| \big(\cos(\phi)\cos(\theta) - \sin(\phi)\sin(\theta)\big).
  \end{align*}
  Wir unterscheiden nun 2 Fälle:
  \begin{enumerate}[1.]
    \item $\phi +  \theta \leq \frac{\pi}{2}$: Dann sind die Cosinusterme der obigen Gleichung positiv und wir schätzen weiter ab zu
      $$
      |w + z|^2 \geq |w|^2 + |z|^2 - 2|w||z|\sin(\phi)\sin(\theta) \geq (1 - \sin(\phi)\sin(\theta)) (|w|^2 + |z|^2).
      $$
    \item $\phi + \theta > \frac{\pi}{2}$: Dann gilt $\cos(\phi + \theta) < 0$ und wir schätzen wie folgt ab:
      $$
      |w + z|^2 \geq (1 + \cos(\phi + \theta)) (|w|^2 + |z|^2 ).
      $$
  \end{enumerate}
  Die Behauptung folgt also für $C(\phi,\theta) \coloneqq \frac{1}{\sqrt{2}}\, \Big(\min\{1 + \cos(\phi + \theta), 1 - \sin(\phi)\sin(\theta)\}\Big)^{\frac{1}{2}}$.
\end{proof}

\begin{prop}
  Sei $\Omega \subset \R^d, d\geq2$, offen und $A$ der Stokes Operator auf $\Ell^2_\sigma(\Omega)$. Dann gilt $\sigma(A) \subset [0,\infty)$ und für alle $\theta \in (0,\pi]$ existiert $C > 0$, sodass
  $$
  \| \lambda (\lambda - A)^{-1}\|_{\Li(\Ell^2_\sigma(\Omega))} \leq C, \quad\text{für alle } \lambda \in \C \setminus \overline{\Sec_\theta}
  $$
  und
  $$
  \| |\lambda|^{\frac{1}{2}} \nabla (\lambda - A)^{-1} \|_{\Li(\Ell_\sigma^2, \Ell^2)} \leq C, \quad\text{für alle } \lambda \in \C \setminus \overline{\Sec_\theta}.
  $$
\end{prop}

\begin{proof}
  Sei $\theta \in (0,\pi]$. 
  Für $\lambda \in \C \setminus \overline{\Sec_\theta}$ definiere
  $$
  a_\lambda \colon \HH_{0,\sigma}^1(\Omega) \times \HH_{0,\sigma}^1(\Omega) \to \C, \quad (u,v) \mapsto \lambda \int_\Omega u \cdot \overline v \d x - \int_\Omega \nabla u \cdot \overline{\nabla v} \d x,
  $$
  dann ist $a_\lambda$ stetig.
  Für die Koerzivität beobachten wir, dass zunächst $- \C \setminus \overline{\Sec_\theta} = \Sec_{\pi - \theta}$ gilt, was
  $$
  |a_\lambda(u,u)| 
  = \Big| \underbrace{- \lambda\int_\Omega |u|^2 \d x}_{\Sec_{\pi - \theta}} + \underbrace{\int_\Omega |\nabla u|^2 \d x}_{\in \Sec_0} \Big| 
  \geq \frac{1}{C} \left( |\lambda| \int_\Omega |u|^2 \d x + \int_\Omega |\nabla u|^2 \d x\right)
  $$
  unter Verwendung von Lemma \ref{lem:inverseTriangle} ergibt, woraus mittels Lemma von Lax-Milgram $\lambda \in \rho(A)$ folgt.
  
  Um die Abschätzungen nachzuweisen testen wir mit der Lösung.  
  Sei $f \in \Ell_\sigma^2(\Omega)$ und $u \in \DD(A)$ mit $(\lambda - A) u = f$.
  Teste mit $u$:
  $$
  \lambda \int_\Omega |u|^2 \d x - \int_\Omega |\nabla u|^2 = \int_\Omega f \cdot \overline u.
  $$
  Nehme Betrag und nutze obige Ungleichung, dann folgt
  $$
  \frac{1}{C} |\lambda|\|u\|_{\Ell^2(\Omega)}^2 
  \leq \frac{1}{C} \left(|\lambda| \int_\Omega |u|^2 \d x + \int_\Omega |\nabla u|^2\right) 
  \leq \|f\|_{\Ell^2(\Omega)} \|u\|_{\Ell^2(\Omega)}
  $$
  und damit gilt die Resolventenabschätzung.
  Weiterhin folgt mit Young's Ungleichung
  $$
  \frac{1}{C} \left(|\lambda| \int_\Omega |u|^2 \d x + \int_\Omega |\nabla u|^2 \d x \right)
  \leq \frac{1}{2\varepsilon} \|f\|_{\Ell^2(\Omega)}^2 + \frac{\varepsilon}{2} \|u\|_{\Ell^2(\Omega)}^2.
  $$
  Wähle $\varepsilon = \frac{2| \lambda|}{C}$, dann gilt
  $$
  \frac{1}{C} \int_\Omega |\nabla u|^2 \d x \leq \frac{C}{4|\lambda|} \|f \|_{\Ell^2(\Omega)}^2,
  $$
  und damit ist auch die Gradientenabschätzung erfüllt.
\end{proof}

\begin{thm}
  Sei $\Omega \subset \R^d, d \geq 2$ offen und $A$ sei der Stokes-Operator auf $\Ell^2_\sigma(\Omega)$.
  Dann erzeugt $-A$ eine beschränkte analytische Halbgruppe $(\e^{-tA})_{t \geq 0}$.
  Diese wird als Stokes-Halbgruppe bezeichnet.
  Weiterhin ist für jedes $t > 0, \Rg(\e^{-tA}) \subset \HH_{0,\sigma}^1(\Omega)$ und es existiert $C > 0$, sodass für alle $t > 0$ und $a \in \Ell_\sigma^2(\Omega)$ gilt:
  $$
  \|\nabla \e^{-tA} a \|_{\Ell^2(\Omega)} \leq C t^{-\frac{1}{2}} \|a\|_{\Ell_\sigma^2(\Omega)}
  $$
\end{thm}

\begin{proof}
  Übung.
\end{proof}

\section{Wie man den Druck erhält}

Zuerst führen wir ein nützliches Handwerkszeug, den sogenannten Bogowski\u{\i}-Operator, ein.
Hierzu definieren wir für $1 < p < \infty$ und ein beschränktes Gebiet $\Omega \subset \R^d$ den Raum
$$
\Ell^p_0(\Omega) \coloneqq \Big\{ f \in \Ell^p(\Omega) \colon \frac{1}{|\Omega|} \int_\Omega f \d x \eqqcolon \dashint_\Omega f \d x \eqqcolon f_\Omega = 0 \Big\}
$$
der mittelwertfreien $\Ell^p$-Funktionen.

Wegen $\int_\Omega \div (u) \d x = 0$ für alle $u \in \WW_0^{1,p}(\Omega; \C^d)$ ist es notwendig, dass die rechte Seite $f$ der folgenden Gleichung in $\Ell^p_0(\Omega)$ liegt.
Betrachte das Problem
\begin{align*}
  \div(u) &= f \quad\text{in } \Omega, \\
  u &= 0 \quad\text{auf } \partial \Omega.
\end{align*}
Ist $\Omega$ ein beschränktes Lipschitz-Gebiet, so wurde ein Lösungsoperator (Bogowski\u{\i}-Operator) für diese Gleichung konstruiert.

\begin{thm}
  \label{thm:bogowskii}
  Sei $\Omega \subset \R^d, d \geq 2$, ein beschränktes Lipschitz-Gebiet, dann existiert ein Operator $\Bog$, sodass für jedes $1 < p < \infty$ gilt:
  \begin{align*}
    &\Bog \colon \Ell_0^p(\Omega) \to \WW_0^{1,p}(\Omega; \C^d), \quad \Bog \in \Li\big(\Ell_0^p(\Omega), \WW_0^{1,p}(\Omega; \C^d)\big) \\
    &\div(\Bog f) = f, \quad\text{für alle } f \in \Ell_0^p(\Omega).
  \end{align*}
\end{thm}

\begin{proof}
  Siehe z.B. \cite[Seiten 161-172]{galdi}.
\end{proof}

Für $u \in \Ell^p(\Omega)$ definiere $\nabla u \in \WW^{-1,p}(\Omega; \C^d)$ durch
$$
\langle v, \nabla u\rangle_{\WW_0^{1,p'}(\Omega), \WW^{-1,p}(\Omega)}
= -\int_\Omega u \cdot \overline{\div(v)}, \quad\text{für alle } v \in \WW^{1,p'}_0(\Omega; \C^d),
$$
wobei $\frac{1}{p} + \frac{1}{p'} = 1$.

\begin{lem}
  \label{lem:poincare}
  Sei $\Omega \subset \R^d, d \geq 2$, ein beschränktes Lipschitz-Gebiet und $1 < p< \infty$.
  Dann existiert $C > 0$, sodass für alle $u \in \Ell^p(\Omega)$
  $$
  \| u - u_\Omega\|_{\Ell^p(\Omega)} \leq C\, \| \nabla u\|_{\WW^{-1,p}(\Omega)}
  $$
  gilt.
\end{lem}

\begin{proof}
  Sei $\Bog$ der Bogowski\u{\i}-Operator aus Satz \ref{thm:bogowskii} und $f \in \Ell^{p'}(\Omega)$, wobei $\frac{1}{p} + \frac{1}{p'} = 1$.
  Dann gilt
  \begin{align*}
  \Big| \int_\Omega (u - u_\Omega) \, \overline f \d x \Big|
    &= \Big| \int_\Omega (u - u_\Omega) \, (\overline{f - f_\Omega})  \d x \Big|
    = \Big| \int_\Omega u \, (\overline{f - f_\Omega}) \d x \Big| \\
    &= \Big| \int_\Omega u \, \div( \Bog(\overline{f - f_\Omega}) ) \d x \Big|
    \leq \|\nabla u\|_{\WW^{-1,p}(\Omega)} \, \|\Bog(\overline{f - f_\Omega}) \|_{\WW_0^{1,p'}(\Omega)} \\
    &\overset{\text{Satz } \ref{thm:bogowskii}}{\leq}  C \,  \|\nabla u\|_{\WW^{-1,p}(\Omega)} \, \|f - f_\Omega\|_{\Ell^{p'}(\Omega)}
    \leq C \, \|\nabla u\|_{\WW^{-1,p}(\Omega)} \, \|f\|_{\Ell^{p'}(\Omega)},
  \end{align*}
  wobei im letzten Schritt ausgenutzt wurde, dass $\Omega$ beschränkt ist.
  Daraus folgt nun die Behauptung.
\end{proof}

\begin{lem}
  \label{lem:poincare2}
  Sei $\Omega \subset \R^d, d\geq 2$ ein beschränktes Lipschitz-Gebiet und $1 < p < \infty$.
  Dann existiert für jedes Teilgebiet $\Omega_0 \subset \Omega$ mit $\Omega_0 \neq \emptyset$ ein $C > 0$, sodass für alle $u \in \Ell^p(\Omega)$ mit $u_{\Omega_0} = 0$ gilt
  $$
  \|u \|_{\Ell^p(\Omega)} \leq C \, \| \nabla u \|_{\WW^{-1,p}(\Omega)}.
  $$
\end{lem}

\begin{proof}
  Angenommen die Aussage wäre falsch.
  Dann existiert für jedes $n \in \N$ ein $u_n \in \Ell^p(\Omega)$ mit $(u_n)_\Omega = 0$ und
  \begin{align*}
    \|u_n\|_{\Ell^p(\Omega)} > n \|\nabla u_n \|_{\WW^{-1,p}(\Omega)} \tag{$\ast$}.
  \end{align*}
  Sei ohne Einschränkung $\|u_n\|_{\Ell^p(\Omega)} = 1$.
  Da $(u_n)_{n \in \N} \subset \Ell^p(\Omega)$ beschränkt und $\Ell^p(\Omega)$ reflexiv ist, besitzt $(u_n)_{n \in \N}$ eine schwach konvergente Teilfolge.
  Bezeichne diese Teilfolge ohne Einschränkung wieder mit $(u_n)_{n \in \N}$.
  Dann existiert ein $u \in \Ell^p(\Omega)$ mit $\lim_{n \to \infty} \int_\Omega u_n \overline v \d x = \int_\Omega u \overline v \d x$ für alle $v \in \Ell^{p'}(\Omega)$.
  Hieraus folgt, dass
  \begin{align*}
  \int_{\Omega_0} u \d x 
  = \int_\Omega u \chi_{\Omega_0} \d x
  = \lim_{n \to \infty} \int_\Omega u_n \chi_{\Omega_0} \d x = 0. \tag{$\ast\ast$}
  \end{align*}
  Mit ($\ast$) folgt $\|\nabla u_n\|_{\WW^{-1,p}(\Omega)} < \frac{1}{n} \to 0$ für $n \to \infty$.
  Weiterhin folgt für $v \in \CC_\cc^\infty(\Omega;\C^d)$
  $$
  \Big| \int_\Omega u \, \overline{\div(v)} \d x \Big|
  = \lim_{n \to \infty} \Big| \int_\Omega u_n \overline{\div(v)} \d x \Big|
  = \lim_{n \to \infty} \Big| \langle v, \nabla u_n \rangle_{\WW_0^{1,p'}, \WW^{-1,p}(\Omega)} \Big|.
  $$
  Folglich ist $u$ schwach differenzierbar mit $\nabla u = 0$ und damit konstant.
  Mit ($\ast\ast$) folgt hieraus $u = 0$.
  Aus Lemma \ref{lem:poincare} ergibt sich
  \[
  1 = \|u_n\|_{\Ell^p(\Omega)}
  \leq C \left[ |(u_n)_\Omega| + \|\nabla u_n\|_{\WW^{-1,p}(\Omega)} \right] \to 0, \quad\text{für } n \to \infty. \qedhere
  \]
\end{proof}

Das folgende Lemma ist die Rechtfertigung dafür, die Stokes/Navier-Stokes-Gleichungen erst auf $\Ell_\sigma^p(\Omega)$ zu lösen und liefert den zugehörigen Druck.

Hierzu definieren wir
$$
f \in \WW_{\loc}^{-1,p}(\Omega; \C^d) \iff f \in \WW^{-1,p}(\Omega_0; \C^d) \text{ f. a. beschränkten Teilgebiete } \Omega_0 \subset \Omega \text{ mit } \overline\Omega_0 \subset \Omega.
$$

\begin{lem}
  \label{lem:pressureGrad}
  Sei $\Omega \subset \R^d, d \geq 2,$ ein Gebiet und $\Omega_0 \subset \Omega$ ein beschränktes Teilgebiet mit $\overline\Omega_0 \subset \Omega$ und $\Omega_0 \neq \emptyset$.
  Weiterhin sei $1 < p < \infty$ und $f \in \WW_{\loc}^{-1,p}(\Omega; \C^d)$ mit
  $$
  \langle v , f \rangle_{\WW_0^{1,p'}(\Omega), \WW_{\loc}^{-1,p}(\Omega)} = 0 \quad\text{für alle } v \in \CC_{\cc,\sigma}^\infty(\Omega).
  $$
  Dann existiert ein eindeutiges $\pi \in \Ell^p_{\loc}(\Omega)$ mit
  $$
  \nabla \pi = f
  $$
  im Sinne von Distributionen und $\int_{\Omega_0} \pi \d x = 0$.
\end{lem}

\begin{proof}
  Wir beweisen erst folgende Aussage: \emph{Für jedes beschränkte Lipschitz-Teilgebiet $\Omega_1 \subset \Omega$ mit $\overline\Omega_0 \subset \Omega_1$ und $\overline\Omega_1 \subset \Omega$ existiert ein eindeutiges $\pi \in \Ell^p(\Omega_1)$ mit $\nabla\pi = f$ im Sinne von Distributionen und $\int_{\Omega_0} \pi \d x = 0$}:

  Sei $x_0 \in \Omega_0$ und $\overline\Omega_1 \subset \Omega_2, \overline\Omega_2 \subset \Omega$.
  Des weiteren sei $\varepsilon < \dist(\partial\Omega_1, \partial(\Omega \cap \BB(x_0,1))$, und $\BB(x_1,\varepsilon),\dots,\BB(x_N,\varepsilon)$ für $x_1,\dots,x_N \in \partial\Omega$ eine offene Überdeckung von $\partial\Omega$.
  So definiert
  $$
  \Omega_2 \coloneqq (\Omega \cap \Bog(x_0, r)) \setminus \bigcup_{k = 1}^N \BB(x_k , \varepsilon)
  $$
  ein weiteres beschränktes Lipschitz-Gebiet.
  Weiterhin folgt aus $f \in \WW_{\loc}^{-1,p}(\Omega; \C^d)$, dass $f \in \WW^{-1,p}(\Omega_2; \C^d)$.
  Da $\Omega_2$ beschränkt ist, existiert (Übung) ein $F \in \Ell^p(\Omega_2; \C^{d\times d})$ mit $f = \div(F)$, wobei
  $$
  \div F = \sum_{i = 1}^d \left( \begin{array}{c} \partial_i F_{i1} \\ \vdots \\ \partial_i F_{id} \end{array} \right).
  $$
  Sei $\rho \in \CC_{\cc}^\infty(\BB(0,1))$ mit $\int_{\BB(0,1)} \rho \d x = 1$, $\rho(x) = \rho(-x)$ und definiere für $0 < \varepsilon < \dist(\Omega_1, \partial\Omega_2)$
  $$
  \rho_\varepsilon(x) \coloneqq \varepsilon^{-d} \rho\Big(\frac{x}{\varepsilon}\Big)
  $$
  und
  $$
  F^\varepsilon \coloneqq \rho_\varepsilon \ast F,
  $$
  wobei $F$ durch Null auf $\R^d$ fortgesetzt wurde.
  Aus AnaIV wissen wir, dass $F^\varepsilon$ glatt ist.
  Im Folgenden wollen wir zeigen, dass 
  $$
  \div F^\varepsilon = \nabla U_\varepsilon \quad\text{in } \Omega_1
  $$
  für ein $U_\varepsilon \in \CC^\infty(\overline\Omega_1)$ gilt.

  Sei $\gamma \colon [0,1] \to \overline\Omega_1$ ein stückweise stetig differenzierbarer Weg mit $\gamma(0) = \gamma(1)$.
  Aus Ana III wissen wir: $\div(F^\varepsilon)$ ist ein Gradientenfeld, falls für alle diese Wege gilt
  $$
  \int_0^1 (\div(F^\varepsilon))(\gamma(t))\cdot \gamma'(t) \d t = 0.
  $$
  Definiere 
  $$
  V_{\gamma,\varepsilon}(x) \coloneqq \int_0^1 \rho_\varepsilon(x - \gamma(t))\gamma'(t) \d t, \quad\text{für alle } x \in \Omega_2.
  $$
  Dann gilt $V_{\gamma,\varepsilon} \in \CC_{\cc}^\infty(\Omega_2; \R^d)$.
  Weiterhin gilt für alle $x \in \Omega_2$
  \begin{align*}
    \div(V_{\gamma,\varepsilon}(x))
    &= \int_0^1 \sum_{j = 1}^d (\partial_j \rho_\varepsilon)(x - \gamma(t)) \gamma_j'(t) \d t
    = - \int_0^1 \frac{\d{} }{\d t} \rho_\varepsilon(x - \gamma(t)) \d t \\
    &= \rho_\varepsilon(x - \gamma(0)) - \rho_\varepsilon(x - \gamma(1))
    = 0.
  \end{align*}
  Daraus folgt $V_{\gamma,\varepsilon} \in \CC_{\cc, \sigma}^\infty(\Omega_2)$ und weiter
  \begin{align*}
    \int_0^1(\div(F^\varepsilon))(\gamma(t)) \cdot \gamma'(t) \d t
    &= \int_0^1 \int_{\Omega_2} \sum_{i,j = 1}^d \big(\partial_i \rho_\varepsilon(\gamma(t) - x)\big) \gamma_j'(t) \d t \, F_{ij}(x)  \d x \\
    %&= -\int_{\Omega_2} \int_0^1 \sum_{i,j = 1}^d \partial_i \rho_\varepsilon(\gamma(t) - x) \gamma_j'(t) \d t \, F_{ij}(x) \d x \\
    &= -\int_{\Omega_2} \int_0^1 \sum_{i,j = 1}^d (\partial_i \rho_\varepsilon) (\gamma(t) - x) \gamma_j'(t) \d t \, F_{ij}(x) \d x \\
    &= -\int_{\Omega_2} \sum_{i,j=1}^d \partial_i\big(  \int_0^1 \rho_\varepsilon(x - \gamma(t)) \gamma_j'(t) \d t \big) \, F_{ij}(x) \d x \\
    &= \langle V_{\gamma,\varepsilon}, \div F \rangle_{\WW_0^{1,p'}(\Omega_2), \WW^{-1,p}(\Omega_2)} \\
    &= 0.
  \end{align*}
  Hieraus ergibt sich, dass ein $U_\varepsilon \in \CC^\infty(\overline\Omega_1)$ existiert mit $\nabla U_\varepsilon = \div(F^\varepsilon)$, welches eindeutig bis auf eine additive Konstante ist.
  Wähle diese Konstante derart, dass $\int_{\Omega_0} U_\varepsilon \d x = 0$.

  Lemma \ref{lem:poincare2} liefert nun
  \begin{align*}
  \| U_\varepsilon\|_{\Ell^p(\Omega_1)} 
  &\leq C \|\nabla U_\varepsilon\|_{\WW^{-1,p}(\Omega_1; \C^d)} 
    = C \| \div(\F^\varepsilon)\|_{\WW^{-1,p}(\Omega_1;\C^d)} \\
    &= C \sup_{\substack{v \in \CC_{\cc}^\infty(\Omega_1; \C^d) \\ \|v\|_{\WW^{1,p'}} \leq 1}} \Big| \sum_{j = 1}^d \langle F_{ij}^\varepsilon, \nabla v_j \rangle_{\Ell^p(\Omega_2), \Ell^{p'}(\Omega_2)} \Big| \\
    &\leq C \, \|F^\varepsilon\|_{\Ell^p(\Omega_1)}.
  \end{align*}
  Mit demselben Argument zeigt man, dass für $0< \eta < \varepsilon$ gilt
  $$
  \|U_\varepsilon - U_\eta \|_{\Ell^p(\Omega_1)} \leq C \| F^\varepsilon - F^\eta\|_{\Ell^p(\Omega_1)}.
  $$
  Aus Ana IV weiß man, dass $F^\varepsilon \to F$ in $\Ell^p(\Omega_1; \C^{d\times d})$, für $\varepsilon \to 0$ gilt, woraus mittels obiger Abschätzung folgt, dass $(U_\varepsilon)_\varepsilon$ ein Cauchy-Netz in $\Ell^p(\Omega_1)$ ist.
  Daher existiert ein $U \in \Ell^p(\Omega_1)$ mit $\int_{\Omega_0} U \d x = 0$, $\|U_\varepsilon - U\|_{\Ell^p(\Omega_1)} \to 0$ für $\varepsilon \to 0$ und
  \begin{align*}
    \langle v, \nabla U \rangle_{\WW_0^{1,p'}(\Omega_1), \WW^{-1,p}(\Omega_1)}
    &= - \int_{\Omega_1} U \, \overline{\div v } \d x
    = - \lim_{\varepsilon \to 0} \int_{\Omega_1} U_\varepsilon \, \overline{\div v} \d x \\
    &= \lim_{\varepsilon \to 0} \langle v, \nabla U_\varepsilon \rangle_{\WW_0^{1,p'}(\Omega_1), \WW^{-1, p}(\Omega_1)}
    = \lim_{\varepsilon \to 0} \langle v, \div(F^\varepsilon) \rangle_{\WW_0^{1,p'}(\Omega_1), \WW^{-1, p}(\Omega_1)} \\
     &= \langle v, \div(F) \rangle_{\WW_0^{1,p'}(\Omega_1), \WW^{-1, p}(\Omega_1)}.
  \end{align*}
  Also gilt $\nabla U = \div F$ in $\WW^{-1,p}(\Omega_1; \C^d)$.
  
  Schöpfe $\Omega$ nun durch beschränkte Lipschitz-Gebiete $\Omega_n$ aus, mit $\overline\Omega_0 \subset \Omega_1$ und $\overline\Omega_n \subset \Omega_{n + 1}, n \in \N$.
  Auf jedem $\Omega_n$ erhält man ein eindeutiges $\pi_n \in \Ell^p(\Omega_n)$ mit $\nabla \pi_n = f$ und $\int_{\Omega_0} \pi_n \d x = 0$.
  Aus der Eindeutigkeit folgt $\pi_n = \pi_{n -1}$ auf $\Omega_{n - 1}$.
  Also existiert ein $\pi \in \Ell^p_{\loc}(\Omega)$ mit $\nabla\pi = f$ und $\int_{\Omega_0} \pi \d x = 0$.
\end{proof}

Eine Anwendung für Lemma \ref{lem:pressureGrad} sieht wie folgt aus:
Sei $A$ der Stokes-Operator auf $\Ell^2_\sigma(\Omega)$ und $(e^{-tA})_{t \geq 0}$ die Stokes-Halbgruppe.
Für $a \in \Ell^2_\sigma(\Omega)$, $t > 0$ gilt dann:
$$
\frac{\d\ }{\d t} \underbrace{\e^{-tA} a}_{\eqqcolon u(t)} = - A \underbrace{\e^{-t A} a}_{\eqqcolon u(t)}.
$$
Mit der Definition des Stokes-Operators folgt einerseits
$$
\int_\Omega u'(t) \overline v \d x + \int_\Omega \nabla u(t)  \overline{\nabla v} \d x= 0, \quad\text{für alle } v \in \HH^1_{0,\sigma}(\Omega).
$$
Andererseits ist für jedes $t > 0$ 
$$
\big( v \mapsto \int_\Omega u'(t) \overline v \d x + \int_\Omega \nabla u(t) \overline{\nabla v} \d x \big) \in \WW^{-1,2}(\Omega; \C^d)
$$
und mit Lemma \ref{lem:pressureGrad} folgt daher, dass $\pi(t) \in \Ell^2_{\loc}(\Omega)$ existiert mit
$$
\int_\Omega u'(t) \overline v \d x + \int_\Omega \nabla u(t) \overline{\nabla v} \d x = -\int_\Omega \pi(t) \overline{\div (v)} \d x
$$
und für alle $v \in \CC_{\cc}^\infty(\Omega; \C^d)$, d.h. $u$ und $\pi$ lösen im Sinne von Distributionen die Stokes-Gleichung:
\begin{alignat*}{2}
  u'(t) &= -\Delta u(t) + \nabla \pi(t) &&\quad\text{in } (0,\infty) \times \Omega, \\
  \div(u(t)) &= 0 &&\quad\text{in } (0,\infty) \times \Omega, \\
  u(0) &= a &&\quad\text{in }  \Omega, \\
  u(t)  &= 0 &&\quad\text{auf } \partial\Omega.
\end{alignat*}

\begin{thm}[Helmholtz-Zerlegung]
  Sei $\Omega \subset \R^d, d \geq 2$ ein Gebiet und
  $$
  G(\Omega) \coloneqq \{ f \in \Ell^2(\Omega; \C^d) \colon \text{es existiert } \pi \in \Ell^2_{\loc}(\Omega) \text{ mit } \nabla \pi = f \}.
  $$
  Dann gilt $\Ell^2_\sigma(\Omega)^\perp = G(\Omega)$.
  Insbesondere existiert für jedes $f \in \Ell^2(\Omega; \C^d)$ eine eindeutige Zerlegung
  $$
  f = f_0 + \nabla \pi
  $$
  mit $f_0 \in \Ell^2_\sigma(\Omega)$ und $\nabla\pi \in G(\Omega)$.
  Weiterhin gilt
  $$
  \|f_0\|_{\Ell^2(\Omega)} \leq \|f\|_{\Ell^2(\Omega} \quad\text{und}\quad \|\nabla\pi \|_{\Ell^2(\Omega)} \leq \|f\|_{\Ell^2(\Omega)}.
  $$
  Die orthogonale Projektion 
  $$
  \PP \colon \Ell^2(\Omega; \C^d) \to \Ell^2(\Omega; \C^d), \quad f \to f_0,
  $$
  wird als Helmholtz-Projektion und obige Zerlegung als Helmholtz-Zerlegung bezeichnet.
\end{thm}

\begin{proof}
  Es genügt $\Ell^2_\sigma(\Omega)^\perp = G(\Omega)$ zu zeigen.
  Die restlichen Aussagen folgen dann aus der Funktionalanalysis.
  
  Sei $\nabla\pi \in G(\Omega)$ und $\varphi \in \CC_{\cc, \sigma}^\infty(\Omega)$.
  Dann gilt
  $$
  \int_\Omega \varphi \cdot \overline{\nabla\pi} \d x = -\int_\Omega \div \varphi \, \overline \pi \d x = 0.
  $$
  Ein Dichtheitsargument liefert $\nabla \pi \in \Ell_\sigma^2(\Omega)^\perp$.

  Nun sei $f \in \Ell_\sigma^2(\Omega)^\perp$.
  Dann gilt 
  $$
  \int_\Omega f \cdot \overline v \d x = 0 \quad\text{für alle } v \in \Ell^2_\sigma(\Omega).
  $$
  Die Hölder-Ungleichung liefert nun
  $$
  \Big( v \mapsto \int_\Omega f \cdot \overline v \d x \Big) \in \WW^{-1,2}(\Omega;\C^d).
  $$
  Mit Lemma \ref{lem:pressureGrad} folgt sodann die Existenz von $\pi \in \Ell^2_{\loc}(\Omega)$ mit $\nabla \pi = f$ im Sinne von Distributionen, d.h.
  $$
  \int_\Omega f \cdot \overline v \d x 
  = - \int_\Omega \pi \, \overline{\div v} \d x, \quad \text{für alle } v \in \CC_{\cc}^\infty(\Omega; \C^d).
  $$
  Daraus folgt $f = \nabla \pi \in G(\Omega)$.
\end{proof}
