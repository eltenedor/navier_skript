\chapter{Die Stokes-Gleichungen auf \texorpdfstring{$\Ell_\sigma^2$}{L_sigma\textasciicircum 2}}

In diesem Kapitel untersuchen wir Lösungen der (instationären) Stokes-Gleichungen
$$
\begin{cases}
  \partial_t u - \Delta u + \nabla p &= 0, \quad x \in \Omega, t > 0 \\
  \div u &= 0, x \in \Omega, t > 0 \\
  u(0) &= a, \quad x \in \Omega \\
  u &= 0, \quad x \in \partial\Omega, t > 0,
\end{cases}
$$
wobei $a \in \Ell^2(\Omega, \C^d)$, $d \geq 2$ und \glqq$\div(a) = 0$\grqq gelten soll.


\section{Der Stokes-Operator auf \texorpdfstring{$\Ell_\sigma^2$}{L_sigma\textasciicircum 2}}

Sei $\Omega \subset \R^d$, $d \geq 2$ und $1 < p < \infty$.
Definiere
$$
\CC_{0,\sigma}^\infty(\Omega) \coloneqq \{ \varphi \in \CC_c^\infty(\Omega, \C^d) \colon \div(\varphi) =0 \}.
$$
Weiterhin sei
$$
\Ell^p_\sigma(\Omega) \coloneqq \overline{\CC_{c,\sigma}^\infty(\Omega)}^{\Ell^p} \quad\text{mit} \|\cdot\|_{\Ell^p_\sigma} = \|\cdot\|_{\Ell^p}
$$
und
$$
\WW_{0,\sigma}^{1,p} \coloneqq \overline{\C_{c,\sigma}^\infty(\Omega)}^{\WW^{1,p}} \quad\text{mit} \|\cdot\|_{\WW_{0,\sigma}^{1,p}} \coloneqq \|\cdot\|_{\WW^{1,p}}.
$$
Im Falle $p= 2$ schreibt man auch $\HH_{0,\sigma}^1(\Omega)$ für $\WW_{0,\sigma}^{1,2}(\Omega)$.
Um den Stokes-Operator zu definieren, definiere folgende Sesquilinearform
$$
a \colon \HH_{0,\sigma}^1(\Omega) \times \HH_{0,\sigma}^1(\Omega) \to \C, \quad (u,v) \mapsto \int_\Omega \nabla u \cdot \overline{\nabla v} \d x = \sum_{i,j=1}^d \int_\Omega \partial_i u_j \partial_i \overline{v_j} \d x
$$

\begin{defn}
  Der Stokes-Operator $A$ auf $\Ell_\sigma^2(\Omega)$ ist gegeben durch
  \begin{align*}
  \DD(A) &\coloneqq \left\{ u \in \HH_{0,\sigma}^1(\Omega) \colon \exists! \colon f \in \Ell_\sigma^2(\Omega) \forall v \in \HH_{0,\sigma}^1(\Omega) \colon a(u,v) = \int_\Omega f \cdot \overline v \d x\right\}, \\
  Au &\coloneqq f,
\end{align*}
wobei $f$ und $u$ durch $\DD(A)$ gegeben sind.
\end{defn}

\begin{prop}
  Der Stokes-Operator auf $\Ell_\sigma^2(\Omega)$ ist abgeschlossen und dicht definiert.
\end{prop}

\begin{proof}
  Zur Abgeschlossenheit: Sei $u_n \in \DD(A)$ mit $u_n \to u$ in $\Ell^2_\sigma(\Omega)$ und $f_n \coloneqq A u_n \to f$ in $\Ell^2_\sigma(\Omega)$.
  Dann
  $$
  \|\nabla(u_n - u_m)\|_{\Ell^2}^2 
  = a(u_n - u_m, u_n - u_m)
  = \int_\Omega (f_n - f_m) \overline{(u_n - u_m)} \d x
  \overset{\text{Hölder}}{\to} 0, \quad\text{für } u,m \to \infty.
  $$
  Folglich ist $(u_n)_{n \in \N}$ eine Cauchy-Folge in $\HH_{0,\sigma}^1(\Omega)$ und damit $u \in \HH_{0,\sigma}^1(\Omega)$.
  Hiermit ergibt sich
  $$
  a(u,v) 
  = \lim_{n \to \infty} a(u_n,v)
  = \lim_{n \to \infty} \int_\Omega f_n \overline{v} \d x
  = \int_\Omega f \cdot \overline v \d x,
  $$
 für alle $v \in \HH_{0,\sigma}^1(\Omega)$.

 Zur Dichtheit: Für $u \in \CC_{c,0}^\infty(\Omega)$, $v \in \HH_{0,\sigma}^1(\Omega)$ gilt
 $$
 a(u,v) = -\int_\Omega \Delta u \cdot \overline v \d x.
 $$
 Aus dem Satz von Schwartz folgt $\Delta u \in \CC_{c,\sigma}^\infty(\Omega)$ und damit $\CC_{c,\sigma}^\infty(\Omega) \subset \DD(A)$..
\end{proof}

\begin{lem*}[Lax-Milgram]
   Sei $H$ ein Hilbertraum über $C$ und $b \colon H \times H \to \C$ eine Sesquilinearform, die stetig und koerziv ist, d.h., es existieren $\alpha, C > 0$, sodass 
   \begin{align*}
     |b(u,v)| &\leq C \|u\|_H \|v\|_X, \quad \text{für alle }u,v \in H, \\
     |b(u,v)| &\geq \alpha \|u\|_H^2, \quad\text{für alle } u \in H.
   \end{align*}
   Dann existiert für jedes $F \in H^*$ ein eindeutiges $u \in H$ mit 
   $$
   b(u,u) = F[v], \quad\text{für alle } v \in H.
   $$
\end{lem*}

\begin{prop}
  Sei $A$ der Stokes-Operator auf $\Ell_\sigma^2(\Omega)$, wobei $\Omega \subset \R^d, d \geq 2$ ein beschränktes Gebiet ist.
  Dann ist $0 \in \rho(A)$.
\end{prop}

\begin{proof}
  Für $f \in \Ell_\sigma^2(\Omega)$ ist $v \mapsto \int_\Omega f \cdot \overline v \d x \in \HH_{0,\sigma}^1(\Omega)^*$ (Antidualraum).
  Weiterhin ist $$a \colon \HH_{0,\sigma}^1(\Omega) \times \HH_{0,\sigma}^1(\Omega) \to \CC, (u,v) \mapsto \int_\Omega \nabla u \cdot \overline{\nabla v} \d x$$ stetig.
  Außerdem folgt mit der Poincar\'e Ungleichung
  $$
  |a(u,u)| = \|\nabla u\|_{\Ell^2}^2 \geq \frac{1}{2} \|\nabla u\|_{\Ell^2}^2 + \frac{1}{2 c^2} \|u\|_{\Ell^2}^2
  $$
  und damit die Koerzivität von $a$.
  Das Lemma von Lax-Milgram liefert sodann, dass genau ein $u \in \HH_{0,\sigma}^1(\Omega)$ mit $a(u,v) = \int_\Omega f \cdot \overline v \d x$ für alle $v \in \HH_{0,\sigma}^1(\Omega)$ existiert.
  Daraus folgt schließlich $u \in \DD(A)$ mit $A u = f$ und $0 \in \rho(A)$.
\end{proof}

\begin{lem}
  Seien $\theta, \phi \in [0,\pi)$ mit $\theta + \phi < \pi$.
    Dann existiert $C = C(\phi,\theta) > 0$, sodass für alle w \in $\SS_\theta_\theta, z \in \SS_\phi$ gilt
    $$
    |w| + |z| \leq c |w + z|.
    $$
\end{lem}

\begin{proof}
  Übung.
\end{proof}<++>

