\chapter{Die Ungleichung von Gagliardo-Nirenberg}

\begin{ntion}
  Sei $-\infty < \frac{1}{p} \leq 1$.
  Fall $0 \leq \frac{1}{p} \leq 1$, dann definiere
  $$
  \| \cdot \|_{X_{\frac{1}{p}}} \coloneqq \| \cdot \|_{\Ell^p}
  $$
  und falls $-\infty < \frac{1}{p} < 0$ sei $\alpha \in [0,1)$ und $k \in \N_0$ derart, dass $-\frac{d}{p} = k + \alpha$. Definiere
  $$
  \| \cdot \|_{X_{\frac{1}{p}}} \coloneqq 
  \begin{cases} 
    \| \nabla^k \cdot \|_{\Ell^\infty}, \quad &\alpha = 0, \\ [ \nabla^k \, \cdot \, ]_\alpha, \quad &\alpha \neq 0,
  \end{cases}
  $$
  wobei $[\,\cdot\,]_\alpha$ die Hölder-Halbnorm zum Exponenten $\alpha$ bezeichne.
\end{ntion}

\begin{hsatz}[Gagliardo-Nirenberg]
  \label{hsatz:gagliardoNirenberg}
  Seien $1 \leq q,r < \infty$, $d \geq 2$ und $j,m \in \N_0$ mit $0 \leq j \lneq m$.
  Weiterhin sei
  $$
  \begin{cases}
    \frac{j}{m} \leq \alpha \leq 1, \quad \text{falls } m - j - \frac{d}{r} \not\in \N_0 \\
    \frac{j}{m} \leq \alpha < 1, \quad \text{falls } m - j - \frac{d}{r} \in \N_0.
  \end{cases}
    $$
    und
    $$
    \frac{1}{p} \coloneqq \frac{j}{d} + \alpha\, \Big( \frac{1}{r} - \frac{m}{d} \Big) + (1 - \alpha) \, \frac{1}{q}.
    $$
    Dann ist $\frac{1}{p} \leq 1$ und es existiert eine Konstante
    $$
    C = C(d,m,j,q,r,\alpha) > 0,
    $$
    sodass für alle $u \in \CC_{\cc}^m(\R^d)$ gilt
    $$
    \| \nabla^j u\|_{X_{\frac{1}{p}}} \leq C \, \|\nabla^m u\|_{\Ell^r}^\alpha \, \|u\|_{\Ell^q}^{1 - \alpha}.
    $$
\end{hsatz}

Für den Beweis benötigen wir einige Vorbetrachtungen.

\begin{lem}
  \label{lem:hoelderIneq}
  Sei $r > d \geq 2$.
  Dann existiert $C = C(d,r) > 0$, sodass für alle $u \in \CC_c^1(\R^d)$ und $x,y \in \R^d$ gilt
  $$
  \frac{\left| u(x) - u(y) \right|}{|x - y|^{1 - \frac{d}{r}}} < C \, \|\nabla u\|_{\Ell^r}.
  $$
\end{lem}

\begin{proof}
  Sei $\delta \coloneqq | x - y|$ und $\BB \coloneqq \BB(x,\delta) \cap \BB(y,\delta)$.
  Dann gilt
  $$
  |u(x) - u(y)| \cdot |\BB|
  \leq \int_{\BB} |u(x) - u(z)| \d z + \int_{\BB} |u(z) - u(y)| \d z.
  $$
  Anwendung des Hauptsatzes liefert für das erste Integral
  \begin{align*}
    \int_\BB |u(x) - u(z)| \d z 
    &\leq \int_{\BB(x,\delta)} \int_0^{|x - z|} \Big| \frac{\d{}}{\d t} \Big[u(x + t \frac{z - x}{|z - x|})\Big] \Big| \d t \d z \\
    &= \int_{\BB(0,\delta)} \int_0^{|z'|} \Big| \frac{\d{}}{\d t} \Big[\, u(x + t \frac{z'}{|z'|} ) \, \Big] \Big| \d t \d z' \\
    &= \int_{\partial\BB(0,1)} \int_0^\delta \Big( \int_0^\rho \Big| \frac{\d{}}{\d t} u(x + t \omega) \Big| \d t \Big) \; \rho^{d - 1} \d \rho \d \sigma(\omega)  \\
    &= \int_{\partial\BB(0,1)} \int_0^\delta \Big( \int_t^\delta \rho^{d - 1} \d \rho \; \Big) \Big| \frac{\d{}}{\d t} \big[ u(x + t \omega) \big] \Big| \d t \d \sigma(\omega) \\
    &\leq \frac{\delta^d}{d} \int_{\BB(0,\delta)} |z'|^{1 - d} |\nabla u(x + z')| \d z' \\
    &\leq \frac{\delta^d}{d} \Big( \int_{\BB(0,\delta)} |z'|^{\frac{r(1 - d)}{r - 1}} \d z' \Big)^{\frac{r - 1}{r}} \, \|\nabla u\|_{\Ell^r(\BB(x,\delta))} \\
    &\leq\frac{\delta^d}{d}\sigma(\BB(0,1))^{\frac{r - 1}{r}}  \Big( \int_0^\delta s^{d - 1} s^{\frac{r(1 - d)}{r - 1}} \d s \Big)^{\frac{r - 1}{r}} \, \|\nabla u\|_{\Ell^r(\BB(x,\delta))} \\
    \intertext{und, da $d - 1 + \frac{r(1 - d)}{r - 1} = \frac{(d - 1)(r - 1) + r - rd}{r - 1} = \frac{1 - d}{r - 1}$, folgt}
    &= \sigma(\BB(0,1))^{\frac{r - 1}{r}} \frac{\delta^{d + 1 - \frac{d}{r}}}{d(\frac{r - d}{r - 1})^{\frac{r - 1}{r}}} \, \| \nabla u\|_{\Ell^r(\BB(x,\delta))}\;.
  \end{align*}
  Aus Symmetriegründen folgt
  $$
  \int_{\BB} |u(z) - u(y)| \d z
  \leq C \delta^{d + 1 - \frac{d}{r}} \|\nabla u\|_{\Ell^r(\BB(y,\delta))},
  $$
  wobei $ C \coloneqq (d(\frac{r - d}{r - 1})^{\frac{r - 1}{r}})^{-1} $.
  Weiterhin folgt aus $\BB(\frac{1}{2}(x + y), \frac{\delta}{2}) \subset \BB$, dass $|\BB| \geq |\BB(0,1)| 2^{-d} \delta^d$.
  Hieraus ergibt sich 
  $$
  |u(x) - u(y)| \delta^d \leq C \delta^{d + 1 - \frac{d}{r}} \| \nabla u\|_{\Ell^r(\R^d)},
  $$
  wobei $C = C(d,r)$.
\end{proof}

Das folgende Lemma reduziert den Beweis von Hauptsatz \ref{hsatz:gagliardoNirenberg} auf wenige Spezialfälle.

\begin{lem}
  \label{lem:reducingGagliardo}
  \begin{enumerate}[a)]
    \item Angenommen die Ungleichung in Hauptsatz \ref{hsatz:gagliardoNirenberg} gelte für $\alpha = \frac{j}{m}$ mit $j = 1$ und $m = 2$, dann gilt die Ungleichung auch für $\alpha = \frac{j}{m}$ und jedes $0 \leq j < m$.

    \item Angenommen die Ungleichung in Hauptsatz \ref{hsatz:gagliardoNirenberg} gelte für $\alpha = 1$, $j = 0$ und $m = 1$ (wobei $d \neq r$), dann gilt die Ungleichung auch für $\alpha = 1$ und jedes $0 \leq j < m$ vorausgesetzt $m - j - \frac{d}{r} \not\in \N_0$.
      
    \item Für alle $-\infty < \lambda \leq \mu \leq \nu \leq 1$ existiert $C = C(\lambda, \mu, \nu) > 0$, sodass für alle $f \in X_\nu \cap X_\lambda$ die sogenannte Interpolationsungleichung
      $$
      \| f\|_{X_\mu} \leq C \, \|f\|_{X_\lambda}^{\frac{\nu - \mu}{\nu - \lambda}} \, \|f\|_{X_\nu}^{\frac{\mu - \lambda}{\nu - \lambda}}
      $$
      gilt.
      Insbesondere ist $f \in X_\mu$.

    \item Angenommen die Ungleichung in Hauptsatz \ref{hsatz:gagliardoNirenberg} gelte für $\alpha = \frac{j}{m}$ und $\alpha = 1$, dann gilt diese auch für jedes $\frac{j}{m} \leq \alpha \leq 1$.
  \end{enumerate}
\end{lem}

\begin{proof}
  Übung für Ehrgeizige. Für Faule folgt der Beweis. Alle Ungleichungen sind bis auf Konstanten zu verstehen.

  \begin{enumerate}[a)]
    \item Angenommen, die Aussage gelte bis einschließlich $m - 1$ für ein $m \in \N$. Sei $1 < j < m - 1$ und damit $\alpha = \frac{j}{m}$. 
      Seien zusätzlich $1 \leq q,r < \infty$ und damit
      \begin{align*}
        \frac{1}{p} = \frac{j}{d} + \alpha \Big( \frac{1}{r} - \frac{m}{d} \Big) + (1 - \alpha) \cdot \frac{1}{q} = \frac{j}{mr}  + \Big( 1 - \frac{j}{m} \Big) \frac{1}{q} 
      \end{align*}
      Insbesondere gilt also $0 \leq \frac{1}{p} \leq 1$.
      Wir setzen
      $$
      j^* = 1, \quad m^* = m - j + 1 \leq m - 1 \quad\text{und}\quad \alpha^* = \frac{j^*}{m^*}
      $$
      sowie
      $$
      j^{**} = j - 1, \quad m^{**} = j \quad\text{und}\quad \alpha^{**}=\frac{j^{**}}{m^{**}}
      $$
      und rechnen
      \begin{align*}
        \|\nabla^j u \|_p
        &= \|\nabla^1(\nabla^{j - 1} u)u\|_p \\
        &\leq \|\nabla^{m^*}(\nabla^{j-1}) u \|_{r_1}^{\alpha^*} \|\nabla^{j - 1} u\|_{q_1}^{(1 - \alpha^*)} \\
        &\leq \|\nabla^m u\|_{r_1}^{\alpha^*} \Big[ \|\nabla^{m^{**}} u \|_{r_2}^{\alpha^{**}} \|u\|_{q_2}^{(1 - \alpha^{**})} \Big]^{(1 - \alpha^*)} ,
        %&= \|\nabla^m u\|_{r_1} \|\nabla^j u\|_{r_2}^{\alpha^{**}(1 - \alpha^*)} \|u\|_{q_2}^{(1 - \alpha^{**})(1 - \alpha^*)}
      \end{align*}
      wobei $r_i, q_i$, $i \in \{1,2\}$ passend gewählt seien.
      Wir verrechnen zunächst die Exponenten:
      \begin{align*}
        \alpha^{**}(1 - \alpha^*) 
        &= \frac{j^{**}}{m^{**}} ( 1 - \frac{j^*}{m^*}) \\
        &= \frac{j - 1}{j} ( 1 - \frac{1}{m - j + 1}) \\
        &= \frac{j(m - j + 1) - (m - j + 1) - j + 1}{j(m - j + 1)} \\
        &= \frac{j(m - j + 1) - m}{j(m - j + 1)}.
        \intertext{Weiter erhalten wir}
        (1 - \alpha^{**})(1 - \alpha^*)
        &= 1 - \alpha^{**} - \alpha^* + \alpha^*\alpha^{**} \\
        &= 1 - \frac{j - 1}{j} - \frac{1}{m - j + 1} + \frac{1}{m - j + 1} \cdot \frac{j - 1}{j} \\
        &= 1 + \frac{-(j - 1)(m - j + 1) -j + j + 1}{j(m - j + 1)} \\
        &= \frac{j(m - j + 1) + (m - j + 1) - j(m - j + 1) - 1}{j(m - j + 1)} \\
        &= \frac{m - j}{j (m - j + 1)}.
      \end{align*}
      Setzen wir nun
      $$
      \beta \coloneqq 1 - \alpha^{**}(1 - \alpha^*) = \frac{m}{j(m - j + 1)},
      $$
      so erhalten wir
      \begin{align*}
        \frac{\alpha^*}{\beta} 
        &= \frac{1}{m - j + 1} \cdot \frac{j(m - j + 1)}{m} = \frac{j}{m} 
        \intertext{sowie}
        \frac{(1 - \alpha^{**})(1 - \alpha^*)}{\beta}  
        &= \frac{m - j}{j(m - j + 1)} \cdot \frac{j(m - j + 1)}{m} = 1 - \frac{j}{m}.
      \end{align*}
      Nehmen wir zusätzlich an, dass
      $$
      r_1 = r, \quad r_2 = p \quad\text{und}\quad q_2 = q
      $$
      gelten, so folgt
      $$
      \|\nabla^j u\|_p^\beta \leq \|\nabla^m u\|_r^{\alpha^*} \|u\|_q^{(1 - \alpha^{**})(1 - \alpha^*)}
      $$
      und daraus durch $(\cdot)^{\frac{1}{\beta}}$ die Behauptung.

      Wir prüfen nun, ob sich $r_1$, $r_2$ und $q_2$ nach unserem Wunsch wählen lassen. 
      Dazu müssen folgende Gleichungen erfüllt werden.
      \begin{align*}
        \frac{1}{p} &= \frac{j}{mr} + (1 - \frac{j}{m}) \frac{1}{q} \tag{I}\\
        \frac{1}{p} &= \frac{j^*}{m^* r} + (1 - \frac{j^*}{m^*}) \frac{1}{q_1} = \frac{1}{r(m - j + 1)} + (1 - \frac{1}{m - j + 1}) \frac{1}{q_1} \tag{II} \\
        \frac{1}{q_1} &= \frac{j^{**}}{m^{**}p} + (1 - \frac{j^{**}}{m^{**}} )\frac{1}{q} = \frac{j - 1}{jp} + (1 - \frac{j - 1}{j}) \frac{1}{q} \tag{III}.
      \end{align*}
      Einsetzen von (III) in (II) ergibt:
      \begin{align*}
        \frac{1}{p} \underbrace{\Big[ 1 - (1 - \frac{1}{m - j + 1})(\frac{j - 1}{j}) \Big]}_{\eqqcolon A}
        &= \frac{1}{r(m - j + 1)} + \underbrace{(1 - \frac{1}{m - j + 1})(1 - \frac{j - 1}{j})}_{\eqqcolon B} \frac{1}{q} \tag{IV}
      \end{align*}
      Und weiter
      $$
      A = 1 - \frac{j - 1}{j} + \frac{j - 1}{j(m - j + 1)} = \frac{1}{j} + \frac{j - 1}{j(m - j + 1)} = \frac{m - j + 1 + j - 1}{j(m - j + 1)} = \frac{m}{j(m - j + 1)}
      $$
      sowie
      $$
      B = \frac{m - j}{m - j + 1} \cdot \frac{1}{j} = \frac{m - j}{j(m - j + 1)}
      $$
      Division durch $A$ in (IV) ergibt
      \begin{align*}
        \frac{1}{p} 
        &= \frac{j(m - j + 1)}{m} \cdot \frac{1}{m - j + 1} \cdot \frac{1}{r} + \frac{j(m - j + 1)}{m} \cdot \frac{m - j}{j(m - j + 1} \cdot \frac{1}{q} \\
        &= \frac{j}{m} \cdot \frac{1}{r} + (1 - \frac{j}{m}) \frac{1}{q}.
      \end{align*}

      Im Falle $j = 1$ schätzen wir wie folgt ab:
      \begin{align*}
        \| \nabla u\|_p &\leq \|\nabla^{m - 1} u\|_{r_1}^{\alpha^*} \|u\|_{q_1}^{1 - \alpha^*} \\
        &= \|\nabla^{m - 2} \nabla u \|_{r_1}^{\alpha^*} \|u\|_{q_1}^{1 - \alpha^*} \\
        &\leq \Big[ \|\nabla^{m - 1} \nabla u\|_{r_2}^{\alpha^{**}} \|\nabla u\|_{q_2}^{1 - \alpha^{**}} \Big]^{\alpha^*} \|u\|_{q_1}^{1 - \alpha^*}
      \end{align*}
      Im Falle $j = m - 1$ schätzen wir ähnlich ab:
      \begin{align*}
        \| \nabla^{m - 1} u\|_p 
        &= \|\nabla^{m - 2} \nabla u \|_p \\
        &\leq \|\nabla^{m - 1} \nabla u\|_{r_1}^{\alpha^*} \|\nabla u\|_{q_1}^{1 - \alpha^*} \\
        &\leq \|\nabla^{m } u\|_{r_1} \Big[ \|\nabla^{m - 1} u\|_{r_2}^{\alpha^{**}} \|u\|_{q_2}^{1 - \alpha^{**}} \Big]^{1 - \alpha^*}
\end{align*}
Analoge Rechnungen ergeben, dass sich $r_i$, $q_i$, $i \in \{1,2\}$ immer passend wählen lassen.
    \item Angenommen, die Aussage gelte bis einschließlich $m - 1$ für ein $m \in \N$. Sei $0 \leq j < m $ und $\alpha = 1$. 
      Sei zusätzlich $1 \leq r < \infty$ und damit
      $$
      \frac{1}{p} = \frac{j}{d} + (\frac{1}{r} - \frac{m}{d}) = \frac{1}{d} ( j - m + \frac{d}{r}),
      $$
      wobei wir voraussetzen wollen, dass $-(j - m + \frac{d}{r}) \notin \N_0$.
      Wir rechnen
      $$
      \|\nabla^j u\|_p \leq  \|\nabla^{m - 1} u\|_{r_1} \leq \|\nabla \nabla^{m - 1} u\|_{r_2} = \|\nabla^m u\|_{r_2},
      $$
      wobei wir gerne $r_2 = r$ wählen würden.
      Um zu gewährleisten, dass dies möglich ist müssen folgende Gleichungen erfüllt sein:
      \begin{align*}
        \frac{1}{p} &= \frac{1}{d}( j - (m - 1) + \frac{d}{r_1} ) \tag{I} \\
        \frac{1}{r_1} &= \frac{1}{d} ( 0 - 1 + \frac{d}{r} ) \tag{II}.
      \end{align*}
      Einsetzen von (II) in (I) ergibt
      $$
      \frac{1}{d}( j - (m - 1) + (- 1 + \frac{d}{r})) = \frac{1}{d} (j - m + \frac{d}{r}).
      $$
      Somit ist die Behauptung erfüllt, falls $1 - \frac{d}{r} \notin \N_0$ gilt. Angenommen, das Gegenteil wäre der Fall, so würde gelten
      $$
      m - j - \frac{d}{r} = m - j - 1 + 1 - \frac{d}{r} \in \N_0
      $$
      im Widerspruch zur Voraussetzung.

    \item Für $0 \leq \lambda \leq \mu \leq \nu \leq 1$ ist dies gerade die aus Ana IV bekannte Interpolationsungleichung.
  \end{enumerate}


\end{proof}

Nun sind wir in der Lage Hauptsatz \ref{hsatz:gagliardoNirenberg} zu beweisen.

\begin{proof}[Beweis von Haupsatz \ref{hsatz:gagliardoNirenberg}]
  Dass $\frac{1}{p} \leq 1$ gilt, ist Übungsaufgabe.
  Lemma \ref{lem:reducingGagliardo} reduziert den Beweis auf die folgenden Fälle
  \begin{enumerate}[\text{Fall} 1:]
    \item $\alpha = 1$, $j = 0$, $m = 1$ (für $r \neq d$).
    \item $\frac{j}{m} < \alpha < 1$ und $m - j - \frac{d}{r} \in \N_0$.
    \item $\alpha = \frac{1}{2}$, $j = 1$, $m = 2$.
  \end{enumerate}

  \begin{enumerate}[\text{Fall} 1:]
    \item $\alpha = 1$, $j = 0$, $m = 1$.
      Es gilt $\frac{1}{p} = \frac{1}{r} - \frac{1}{d}$.
      Sei erst $r > d$, und damit $\frac{1}{p} < 0$ und
      $$
      -\frac{d}{p} = 1 - \frac{d}{r} \quad\text{(Hölder-Exponent zu $X_{\frac{1}{p}}$)}
      $$
      Dann folgt die Behauptung aus Lemma \ref{lem:hoelderIneq}.
      Sei nun $r = 1 < d$, $x \in \R^d$, $1 \leq i \leq d$ und $\gamma_i \colon (-\infty, \infty) \to \R^d$ definiert durch $\gamma_i (t) \coloneqq x + t e_i$.
      Es sei $u \in \CC^1(\R^d)$ mit kompaktem Träger. Dann gilt
      $$
      u(x) = \int_{-\infty}^0 \frac{\d{}}{\d t} (u(\gamma_i(t))) \d t = -\int_0^\infty \frac{\d{}}{\d t} (u(\gamma_i(t))) \d t.
      $$
      Wir rechnen zunächst
      $$
      |u(x)| \leq \frac{1}{2} \int_{-\infty}^\infty \big| \frac{\d{}}{\d t} u(\gamma_i(t)) \big| \d t = \frac{1}{2} \int_{-\infty}^\infty \big| \partial_i u(x_1,\dots,x_{i - 1},y_i,x_{i + 1}, \dots,x_d) \big| \d y_i
      $$
      Daraus folgt
      $$
      |u(x)|^{\frac{d}{d - 1}} \leq \frac{1}{2^{\frac{d}{d - 1}}} \prod_{i = 1}^d \Big( \int_{-\infty}^\infty \big| \frac{\d{}}{\d t} u(\gamma_i(t)) \big| \d t \Big)^{\frac{1}{d - 1}}.
      $$
      Integration über $x_1$ ergibt
      \begin{align*}
        &\int_{-\infty}^\infty |u(x)|^{\frac{\d{}}{d - 1}} \d x_1  \\
        &\qquad \leq \frac{1}{2^{\frac{d}{d - 1}}} \Big( \int_{-\infty}^\infty \big| \frac{\d{}}{\d t} u(\gamma_1(t)) \big| \d t \Big)^{\frac{1}{d - 1}} \int_{-\infty}^\infty \prod_{i = 2}^d \Big(\int_{-\infty}^\infty  \big| \frac{\d{}}{\d t} u(\gamma_i(t)) \big| \d t \Big)^{\frac{1}{d - 1}} \d x_1 \\
      &\qquad \leq \frac{1}{2^{\frac{d}{d - 1}}} \Big( \int_{-\infty}^\infty |\partial_i u(x)| \d x_1 \Big)^{\frac{1}{d - 1}} \prod_{i = 2}^d \Big(\int_{-\infty}^\infty \int_{-\infty}^\infty \big| \frac{\d{}}{\d t} u(\gamma_i(t)) \big| \d x_1 \d t \Big)^{\frac{1}{d - 1}}
      \end{align*}
      Induktive Integration über $x_1,\dots,x_d$ liefert
      \begin{align*}
        \int_{\R^d} |u(x)|^{\frac{d}{d - 1}} \d x \leq \frac{1}{2^{\frac{d}{d - 1}}} \prod_{i = 1}^d \Big( \int_{\R^d} \big|\partial_i u(x) \big| \d x \Big)^{\frac{1}{d - 1}} \tag{$\ast$}
      \end{align*}
      Daraus ergibt sich die Behauptung für $r = 1$.
      Sei nun $1 < r < d$. 
      Definiere $v \coloneqq |u|^{\frac{(d - 1) r}{d - r}}$.
      Da $\frac{(d - 1)r}{d - r} > 1$ folgt $v \in \CC_{\mathrm{c}}^1(\R^d)$, somit ist ($\ast$) mit $u = v$ anwendbar und wir rechnen
      \begin{align*}
        \Big( \int_{\R^d} |u(x)|^{\frac{rd}{d - r}} \d x \Big)^{\frac{d - 1}{d} }
       &= \Big( \int_{\R^d} |v(x)|^{\frac{d}{d - 1}} \d x \Big)^{ \frac{d - 1}{d}} \\
        &\leq \frac{1}{2^{\frac{d - r}{r(d - 1)}}} \prod_{i = 1}^d \Big( \int_{\R^d} |\partial_i v(x) | \d x \Big)^{\frac{1}{d}} \\
        &\leq C(d,r) \prod_{i = 1}^d \Big( \int_{\R^d} |\partial_i u(x)| |u(x)|^{\frac{d (r - 1)}{d - r}} \d x \Big)^{\frac{1}{d}} \\
        &\leq C(d,r) \Bigg[ \prod_{i = 1}^d \Big( \int_{\R^d} |\partial_i u(x)|^r \d x \Big)^{\frac{1}{r d}} \Bigg] \Big( \int_{\R^d} |u(x)|^{\frac{rd}{d - r}} \d x \Big)^{\frac{(r - 1)}{r}}.
      \end{align*}
      Abschließend teilen wir durch das $u$ Integral und erhalten
      $$
      \Big( \int_{\R^d} |u(x)|^{\frac{rd}{d - r}} \d x \Big)^{\frac{d - r}{rd}} \leq C(d,r) \prod_{i = 1}^d \Big( \int_{\R^d} |\partial_i u(x)|^r \d x \Big)^{\frac{1}{rd}}.
      $$
      Es gilt übrigens
      $$
      C(d,r) = \frac{r}{2} \, \frac{d - 1}{d - r}.
      $$
      Damit wäre die Behauptung im Falle $1 < r < d$ gezeigt.
    \item 
      Es seien $0 < \alpha < 1$, $j = 0$, $m = 1$, $d = r$.
      Dann ist $ \frac{1}{p} = \frac{1 - \alpha}{q} $ und wir definieren
      $$
      q_k \coloneqq q + k \cdot \frac{d}{d - 1}, \quad k \in \N, \quad q_0 \coloneqq q.
      $$
      Wenn $k \geq 1$, so gilt $q_k \cdot \frac{d - 1}{d} > 1$.
      Idee: Wähle $k$ groß genug, sodass $q < p < q_k$ gilt, um die Interpolationsungleichung anzuwenden.
      Definiere
      $$
      v_k \coloneqq |u|^{q_k \cdot \frac{d - 1}{d}}.
      $$
      Dann gilt $v_k \in \CC_c^1(\R^d)$ und aus ($\ast$) mit $u = v_k$ folgt
      \begin{align*}
        \Big( \int_{\R^d} |u(x)|^{q_k} \d x \Big)^{\frac{1}{q_k}}
        &= \Big( \int_{\R^d} |v_k(x)|^{\frac{d}{d - 1}} \d x \Big)^{\frac{1}{q_k}} \\
        &\leq C(d,q,k) \Bigg[ \prod_{i = 1}^d \Big( \int_{\R^d} |\partial^i u(x)||u(x)|^{\frac{q_k(d - 1)}{d }} \d x \Big)^{\frac{1}{d - 1}} \Bigg]^{\frac{1}{q_k}} \\
        &\leq C(d,q,k) \Bigg[ \big\|\nabla u\big\|_{\Ell^d}^{\frac{d}{d - 1 }} \; \Big\| |u|^{\frac{q_k (d - 1)}{d} - 1} \Big\|_{\Ell^{\frac{d}{d - 1}}}^{\frac{d}{d - 1}} \Bigg]^{\frac{1}{q_k}} \\
        &= C(d,q,k) \, \big\|\nabla u\big\|_{\Ell^d}^{\frac{d}{q_k (d - 1)}} \; \big\|u\big\|_{\Ell^{q_{k - 1}}}^{\frac{q_{k - 1}}{q_k}}.
      \end{align*}
      Wähle nun $k \in \N$ mit $q_{k - 1} \leq p \leq q_k$ und definiere 
      $$
      \mu \coloneqq \frac{1}{p}, \quad \nu \coloneqq \frac{1}{q}, \quad \lambda_k \coloneqq \frac{1}{q_k} \quad\text{und}\quad \lambda_l \coloneqq \frac{1}{q_l} \quad\text{für } l < k.
      $$
      Dann gilt
      \begin{align*}
        \|u\|_{\Ell^p}
        &\leq C \, \|u\|_{\Ell^{q_k}}^{\frac{\nu - \mu}{\nu - \lambda_k}} \, \|u\|_{\Ell^q}^{\frac{\mu - \lambda_k}{\nu - \lambda_k}} 
        \leq C \, \|\nabla u\|_{\Ell^d}^{\lambda_k \cdot \frac{d}{d - 1} \, \frac{\nu - \mu}{\nu - \lambda_k}} \; \|u\|_{\Ell^{q_{k - 1}}}^{\frac{\lambda_k}{\lambda_{k - 1}} \frac{\nu - \mu}{\nu - \lambda_k}}  \; \|u\|_{\Ell^{q}}^{\frac{\mu - \lambda_k}{\nu - \lambda_k}}  \tag{\#}
      \end{align*}
      Falls $k = 1$, so gelten 
      $$
      q_{k - 1} = q, \quad \lambda_k \cdot \frac{d}{d - 1}\frac{\nu - \mu}{\nu - \lambda_k} = \alpha \quad\text{und}\quad \frac{\lambda_k}{\lambda_{k - 1}} \frac{\nu - \mu}{\nu - \lambda_k} + \frac{\mu - \lambda_k}{\nu - \lambda_k}=  (1 - \alpha),
      $$
      denn
      \begin{align*}
      \lambda_k \cdot \frac{d}{d - 1} \cdot \frac{\nu - \mu}{\nu - \lambda_k}
        &= \frac{1}{q_1} \cdot \frac{d}{d - 1} \cdot \frac{\frac{1}{q} - \frac{1}{p}}{\frac{1}{q} - \frac{1}{q_1}} 
        = \frac{1}{q + \frac{d}{d - 1}} \cdot \frac{d}{d - 1} \cdot \frac{\frac{1}{q} - \frac{1 - \alpha}{q}}{\frac{1}{q} - \frac{1}{q + \frac{d}{d - 1}}}  \\
        &= \frac{d}{q(d - 1) + d} \cdot \frac{\alpha}{1 - \frac{q}{q + \frac{d}{d - 1}}} 
        = \frac{d}{q(d - 1) + d} \cdot \frac{\alpha (q + \frac{d}{d - 1})}{q + \frac{d}{d - 1} - q} \\
        &= \frac{d}{q(d - 1) + d} \cdot \frac{(q(d - 1) + d) \alpha}{d} 
        = \alpha.
      \end{align*}
      Falls $k \geq 2$ so rechnen wir weiter
      \begin{align*}
        (\#) 
        &\leq C\, \|\nabla u\|_{\Ell^d}^{\lambda_k \cdot \frac{d}{d - 1} \cdot \frac{\nu - \mu}{\nu - \lambda_k} + \lambda_{k - 1} \cdot \frac{\lambda_k}{\lambda_{k - 1}} \cdot \frac{d}{d - 1} \frac{\nu - \mu}{\nu - \lambda_k}}
        \cdot \| u\|_{\Ell^{q_{k - 2}}}^{\frac{\lambda_k}{\lambda_{k - 1}} \cdot \frac{\nu - \mu}{\nu - \lambda_k} \cdot \frac{\lambda_{k - 1}}{\lambda_{k - 2}}} 
        \cdot \|u\|_{\Ell^q}^{\frac{\mu - \lambda_k}{\nu - \lambda_k}} \\
        &\dots \\
        &\leq C \, \|\nabla u\|_{\Ell^d}^{k \cdot \frac{d}{d - 1} \cdot \frac{\nu - \mu}{\nu - \lambda_k}} \|u\|_{\Ell^q}^{\frac{\mu - \lambda_k}{\nu - \lambda_k} + \frac{\lambda_k}{\lambda_0} \cdot \frac{\nu - \mu}{\nu - \lambda_k}} \\
        &\leq C \, \|\nabla u\|_{\Ell^d}^{\alpha} \|u\|_{\Ell^q}^{1 - \alpha}.
      \end{align*}
      Die Verallgemeinerung $0 < \alpha < 1$, $0 \leq j < m$, $m - j - \frac{d}{r} \in \N_0$ ist Übungsaufgabe.
    \item
      Seien nun $\alpha = \frac{1}{2}$, $j = 1$ und $m = 2$.
      Sei erst $r > 1$.
      Weiterhin ist $\frac{2}{p} = \frac{1}{r} + \frac{1}{q}$ und $x \in \R^d$.
      Wir zeigen erst, dass
      \begin{align*}
        \int_{\R} |\partial_i u(x)|^p \d x_i 
        \leq C^p \Big( \int_{\R} |\partial_i \partial_i u(x)|^r \d x_i \Big)^{\frac{p}{2r}} \Big(\int_{\R} |u(x)|^q \d x_i \Big)^{\frac{p}{2q}}
        \tag{$\triangle$}
      \end{align*}
      die Behauptung impliziert.
      Integration bezüglich der restlichen Variablen liefert nämlich
      \begin{align*}
        &\int_{\R^d} |\partial_i u(x)|^p \d x \\
        &\qquad\leq C^p \int_{\R^{d - 1}} \Big( \int_{\R} |\partial_i \partial_i u(x)|^r \d x_i \Big)^{\frac{p}{2r}} \Big(\int_{\R} |u(x)|^q \d x_i \Big)^{\frac{p}{2q}} \d (x_1,\dots,x_{i - 1}, x_{i + 1}, \dots, x_d ) \\
        &\qquad\leq C^p \; \Big( \int_{\R^d} |\partial_i \partial_i u(x)|^r \d x \Big)^{\frac{p}{2r}} \Big( \int_{\R^d} |u(x)|^q \d x \Big)^{\frac{p}{2q}}
      \end{align*}
      und damit die Behauptung für $r > 1$.
      Falls die Konstante $C$ für $r \to 1$ beschränkt bleibt, so folgt die Behauptung für $r = 1$ durch majorisierte Konvergenz.

      Um ($\triangle$) zu zeigen, genügt es
      \begin{align*}
        \int_0^L |\partial_i u(x)|^p \d x_i
        \leq C^p \; \Big( \int_0^\infty |\partial_i \partial_i u(x) |^r \d x_i \Big)^{\frac{p}{2r}} \Big( \int_0^\infty |u(x)|^q \d x_i \Big)^{\frac{p}{2q}}
        \tag{$\square$}
      \end{align*}
      mit $C$ unabhängig von $L$ und uniform für $r \to 1$ zu zeigen.
      Ungleichung ($\square$) folgt aus 
      \begin{align*} 
        \tag{$\heartsuit$}       
        \begin{split}
        &\int_I |\partial_i u(x)|^p \d x_i \\
        &\qquad \leq C^p \; |I|^{1 + p - \frac{p}{r}} \Big( \int_I |\partial_i \partial_i u(x)|^r \d x_i \Big)^{\frac{p}{r}} + C^p \; |I|^{-(1 + p - \frac{p}{r})} \Big( \int_I |u(x)|^q \d x_i \Big)^{\frac{p}{q}},
      \end{split}
      \end{align*}
      wobei $I$ ein beliebiges kompaktes Intervall bezeichne.
      Dies sieht man wie folgt: 
      Sei $k \in \N$ fest und nehme an, dass 
      $$
      \int_0^\infty |\partial_i \partial_i u(x)|^r \d x_i = 1.
      $$
      Wir werden das Intervall $[0,L]$ durch eine Folge von (nicht notwendigerweise disjunkten) Intervallen $I_1,I_2,\dots,I_N$ überdecken.
      Falls $J \coloneqq [0, \frac{L}{k}]$ die Eigenschaft hat, dass der erste Summand auf der rechten Seite von ($\heartsuit$) größer als der zweite ist, setze $I_1 = J$.
      Falls nicht, vergrößere $J$ (mit fixiertem linken Endpunkt) so lange, bis beide Summanden gleich groß sind (dies geht, da $1 + p - \frac{p}{r} > 0$ ist). 
      Setze dann ebenso $I_1 \coloneqq J$.

      Falls $I_1 \subset [0,L]$, wiederhole Prozedur für das oben festgelegte $k$ und wähle $I_2$ derart, dass der linke Endpunkt von $I_2$ der rechte Endpunkt von $I_1$ ist.
      Wiederhole das beschriebene Vorgehen so lange, bis $[0,L] \subset \bigcup_{i = 1}^N I_i$ erreicht ist.
      Per constructionem gilt $N \leq k$, da $|I_j| \geq \frac{L}{k}$ für alle $j = 1,\dots,N$ gewählt wurde.
      Es folgt
      \begin{align*}
        &\int_0^L |\partial_i u(x)|^p \d x_i \\
        &\qquad\leq \sum_{i = 1}^N \int_{I_i} |\partial_i u(x)|^p \d x_i \\ 
        &\qquad\leq 2k \Big( \frac{L}{k} \Big)^{1 - p - \frac{p}{r}} + 2C^p \sum_{\substack{i = 1 \\ |I_i| > \frac{L}{k}}}^N \Big( \int_{I_i} |\partial_i \partial_i u(x)|^r \d x_i \Big)^{\frac{p}{2r}} \cdot \Big( \int_{I_i} |u(x)|^q \d x_i \Big)^{\frac{p}{2q}} \\
        &\qquad\leq 2C^p L \Big(\frac{L}{k}\Big)^{p - \frac{p}{r}} + 2C^p \Big( \sum_{\substack{i = 1 \\ |I_i| > \frac{L}{k}}} \int_{I_i} |\partial_i \partial_i u(x)|^r \d x_i \Big)^{\frac{p}{2r}} \cdot \Big( \sum_{\substack{i = 1 \\ |I_i| > \frac{L}{k}}} \int_{I_i} |u(x)|^q \d x_i \Big)^{\frac{p}{2q}},
      \end{align*}
      wobei $C$ die Konstante aus ($\heartsuit$) ist.
      Da $r > 0$ folgt ($\square$) für $k \to \infty$.

      Um ($\heartsuit$) zu zeigen, sei $I = [a,b]$ ein kompaktes Intervall.
      Definiere
      $$
      v(y) \coloneqq u(x_1,\dots,x_{i - 1},y,x_{i + 1},\dots,x_d).
      $$
      Sei $y \in I$, $y_1 \in [a,a + \frac{|I|}{4}]$, $y_2 \in [b - \frac{|I|}{4}, b]$ und $y_{12} \in [y_1, y_2]$ mit
      $$
      \frac{v(y_1) - v(y_2)}{y_1 - y_2} = v'(y_{12}).
      $$
      So folgt
      $$
      v'(y) 
      = v'(y_{12}) + \int_{y_{12}}^y v''(z) \d z
      = \frac{v(y_1) - v(y_2)}{y_1 - y_2} + \int_{y_{12}}^y v''(z) \d z
      $$
      Hieraus ergibt sich
      $$
      |v'(y)| \leq 2\; \frac{|v(y_1)| + |v(y_2)|}{|I|} + \int_I |v''(z)| \d z.
      $$
      Integrieren wir nun über $y_1$ und dann $y_2$, so erhalten wir
      \begin{align*}
        \Big( \frac{|I|}{4} \Big)^2 |v'(y)|
        &\leq 2 \frac{|I|}{4} \frac{1}{|I|} 2 \int_I |v(y)| \d y + \Big( \frac{|I|}{4}\Big)^2 \int_I |v''(z)| \d z \\
        &\leq |I|^{\frac{q - 1}{q}} \Big( \int_I |v(y)|^q \d y \Big)^{\frac{1}{q}} + \Big( \frac{|I|}{4} \Big)^2 |I|^{\frac{r - 1}{r}} \Big( \int_I |v''(z)|^r \d z \Big)^{\frac{1}{r}}.
      \end{align*}
      Bildung der $p$-ten Potenz und Integration über $y$ liefert
      \begin{align*}
        &\Big(\frac{|I|}{4}\Big)^{2p} \int_I |v'(y)|^p \d y\\  
        &\qquad\leq 2^{p - 1} \Bigg[ |I|^{1 + p - \frac{p}{q}} \Big( \int_I |v(y)|^q \d y \Big)^{\frac{p}{q}} + \Big(\frac{1}{16}\Big)^p |I|^{1 + 2p + p - \frac{p}{r}} \Big(\int_I |v''(z)|^r \d z \Big)^{\frac{p}{r}} \Bigg].
      \end{align*}
      Stellen wir die Gleichung noch etwas um, so erhalten wir
      \begin{align*}
        &\int_I |v'(y)|^p \d y  \\
        &\qquad \leq 2^{5p - 1} \Bigg[ |I|^{1 - p - \frac{p}{q}} \Big( \int_I |v(y)|^q \d y \Big)^{\frac{p}{r}} + \Big(\frac{1}{16})^p |I|^{1 + p - \frac{p}{r}} \Big( \int_I |v''(z)|^r \d z \Big)^{\frac{p}{r}} \Bigg].
      \end{align*}
      Aus $\frac{2}{p} = \frac{1}{r} + \frac{1}{q}$ folgt $1 - p - \frac{p}{q} = 1 - p - (2 - \frac{p}{r}) = -1 - p + \frac{p}{r}$ und damit folgt ($\heartsuit$) mit $C$ unabhängig von $r$.
\end{enumerate}
Damit sind alle Fälle bewiesen.
\end{proof}

