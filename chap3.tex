\chapter{Die Ungleichung von Gagliardo-Nirenberg}

\begin{ntion}
  Sei $-\infty < \frac{1}{p} \leq 1$.
  Fall $0 \leq \frac{1}{p} \leq 1$, dann definiere
  $$
  \| \,\cdot\, \|_{X_{\frac{1}{p}}} \coloneqq \| \,\cdot\, \|_{\Ell^p}
  $$
  und falls $-\infty < \frac{1}{p} < 0$ sei $\alpha \in [0,1)$ und $k \in \N_0$ derart, dass $-\frac{d}{p} = k + \alpha$. Definiere
  $$
  \| \,\cdot\, \|_{X^{\frac{1}{p}}} \coloneqq 
  \begin{cases} 
    \| \nabla^k \cdot \|_{\Ell^\infty}, \quad &\alpha = 0, \\ [ \nabla^k \, \cdot \, ]_\alpha, \quad &\alpha = 0,
  \end{cases}
  $$
  wobei $[\,\cdot\,]_\alpha$ die Hölder-Halbnorm zum Exponenten $\alpha$ bezeichne.
\end{ntion}

\begin{hsatz}[Gagliardo-Nirenberg]
  \label{hsatz:gagliardoNirenberg}
  Seien $1 \leq q,r < \infty$, $d \geq 2$ und $j,m \in \N_0$ mit $0 \leq j \lneq m$.
  Weiterhin sei
  $$
  \begin{cases}
    \frac{j}{m} \leq \alpha \leq 1, \quad \text{falls } m - j - \frac{d}{r} \not\in \N_0 \\
    \frac{j}{m} \leq \alpha < 1, \quad \text{falls } m - j - \frac{d}{r} \in \N_0.
  \end{cases}
    $$
    und
    $$
    \frac{1}{p} \coloneqq \frac{j}{d} + \alpha\, \Big( \frac{1}{r} - \frac{m}{d} \Big) + (1 - \alpha) \, \frac{1}{q}.
    $$
    Dann ist $\frac{1}{p} \leq 1$ und es existiert eine Konstante
    $$
    C = C(d,m,j,q,r,\alpha) > 0,
    $$
    sodass für alle $u \in \CC_c^m(\R^d)$ gilt
    $$
    \| \nabla^j u\|_{X_{\frac{1}{p}}} \leq C \, \|\nabla^m u\|_{\Ell^r}^\alpha \, \|u\|_{\Ell^q}^{1 - \alpha}.
    $$
\end{hsatz}

Für den Beweis benötigen wir einige Vorbetrachtungen.

\begin{lem}
  Sei $r > d \geq 2$.
  Dann existiert $C = C(d,r) > 0$, sodass für alle $u \in \CC_c^1(\R^d)$ und $x,y \in \R^d$ gilt
  $$
  \frac{\left| u(x) - u(y) \right|}{|x - y|^{1 - \frac{d}{r}}} < C \, \|\nabla u\|_{\Ell^r}.
  $$
\end{lem}

Das folgende Lemma reduziert den Beweis von Haupsatz \ref{hsatz:gagliardoNirenberg} auf wenige Spezialfälle.

\begin{lem}
  \begin{enumerate}[a)]
    \item Angenommen die Ungleichung in Haupsatz \ref{hsatz:gagliardoNirenberg} gelte für $\alpha = \frac{j}{m}$ mit $j = 1$ und $m = 2$, dann gilt die Ungleichung auch für $\alpha = \frac{j}{m}$ und jedes $0 \leq j < m$.

    \item Angenommen die Ungleichung in Haupsatz \ref{hsatz:gagliardoNirenberg} gelte für $\alpha = 1$, $j = 0$ und $m = 1$ (wobei $d \neq r$), dann gilt die Ungleichung auch für $\alpha = 1$ und jedes $0 \leq j < m$ varausgesetzt $m - j - \frac{d}{r} \not\in \N_0$.
      
    \item Für alle $-\infty < \lambda \leq \mu \leq \nu \leq 1$ existiert $C = C(\lambda, \mu, \nu) > 0$, sodass für alle $f \in X_\nu \cap X_\lambda$ die sogennante Interpolationsungleichung
      $$
      \| f\|_{X_\mu} \leq C \, \|f\|_{X_\lambda}^{\frac{\nu - \mu}{\nu - \lambda}} \, \|f\|_{X_\mu}^{\frac{\mu - \lambda}{\nu - \lambda}}
      $$
      gilt.
      Insbesondere ist $f \in X_\mu$.

    \item Angenommen die Ungleichung in Haupsatz \ref{hsatz:gagliardoNirenberg} gelte für $\alpha = \frac{j}{m}$ und $\alpha = 1$, dann gilt diese auch für jedes $\frac{j}{m} \leq \alpha \leq 1$.
  \end{enumerate}
\end{lem}

\begin{proof}
  Übung.
\end{proof}

Nun sind wir in der Lage Haupsatz \ref{hsatz:gagliardoNirenberg} zu beweisen.

\begin{proof}[Beweis von Haupsatz \ref{hsatz:gagliardoNirenberg}]
\end{proof}

