\chapter{Der Stokes-Operator auf \texorpdfstring{$\Ell_\sigma^p$}{Lsigma\textasciicircum p}}

In diesem Kapitel geben wir einen Überblick über die $\Ell^p$-Theorie der Helmholtz-Zerlegung und des Stokes-Operators.

Sei $\Omega \subset \R^d, d \geq 2$ offen, $1 < p < \infty$ und 
$$
\GG_p(\Omega) \coloneqq \{ f \in \Ell^p(\Omega; \C^d) \colon \text{es ex. } \pi \in \Ell_{\loc}^p(\Omega) \text{ mit } \nabla \pi = f\}.
$$
Wir sagen, dass auf $\Omega$ die Helmholtz-Zerlegung existiert, falls
$$
\Ell^p(\Omega; \C^d) = \Ell_\sigma^p(\Omega) \oplus \GG_p(\Omega)
$$
im Sinne einer algebraischen Summenzerlegung gilt.

Als Nächstes betrachten wir folgendes Neumann-Problem (NP):

\emph{Gegeben sei $u \in \Ell^p(\Omega; \C^d)$.
Finde eine bis auf Konstanten eindeutige Funktion $\pi$ in $\Ell_{\loc}^p(\Omega)$ mit $\nabla \pi \in \Ell^p(\Omega; \C^d)$, sodass
$$
\int_\Omega \nabla \pi \cdot \overline f \d x = \int_\Omega u \cdot \overline f \d x, \quad\text{für alle } f \in \GG_{p'}(\Omega),
$$
wobei $\frac{1}{p} + \frac{1}{p'} = 1$.}

Formal gilt (wenn $\Omega$ und $u$ regulär genug sind): Schreibt man $f = \nabla \phi$, wobei $\phi \in \Ell^{p'}_{\loc}(\Omega)$, so folgt durch partielle Integration
\begin{align*}
  -\int_\Omega \Delta\pi \cdot \overline{\phi} \d x
  &= -\int_{\partial\Omega} n \cdot \nabla\pi \overline\phi \d s
     +\int_\Omega \nabla \pi \cdot \overline{\nabla \phi} \d x \\
  &= - \int_{\partial\Omega} n \cdot \nabla \pi \overline \phi \d s 
     + \int_\Omega u \cdot \overline{\nabla \phi} \d x \\
     &= -\int_{\partial\Omega} n \cdot [u - \nabla\pi] \overline \phi \d x
        - \int_\Omega \div(u) \overline\phi \d x,
\end{align*}
d.h. $\pi$ löst (formal) das Neumann-Problem
$$
\begin{cases}
  -\Delta \pi = \div (u)         &\quad\text{in } \Omega \\
  n \cdot \nabla \pi = n \cdot u &\quad\text{auf } \partial\Omega.
\end{cases}
$$
Hier bezeichnet $n$ den äußeren Einheitsnormalenvektor von $\Omega$.

\begin{thm}
  \label{thm:helmholtzIffNeumann}
  Sei $\Omega \subset \R^d, d \geq 2$ (ausreichend regulär), offen und $1 < p < \infty$.
  Dann existiert genau dann die Helmholtz-Zerlegung auf $\Ell^p(\Omega; \C^d)$, wenn (NP) für alle $u \in \Ell^p(\Omega; \C^d)$ lösbar ist.
\end{thm}

\begin{proof}
  "$\Rightarrow$": Sei $u \in\Ell^p(\Omega; \C^d)$.
  Dann existiert eine eindeutige Zerlegung 
  $$
  u = u_0 + \nabla \pi \quad\text{mit}\quad u_0 \in \Ell_\sigma^p(\Omega), \nabla\pi \in \GG_p(\Omega).
  $$
  Weiterhin gilt für $f \in \GG_{p'}(\Omega)$
  $$
  \int_\Omega \nabla\pi \cdot \overline f = \int_\Omega u \cdot \overline f \d x - \int_\Omega u_0 \overline f \d x  = \int_\Omega u \cdot \overline f \d x,
  $$
  da $f = \nabla \phi$ für ein $\phi \in \Ell_{\loc}^{p'}(\Omega)$ und $\phi_n \to u_0$ in $\Ell^p$ für eine Folge $(\phi_n)_{n \in \N} \subset \CC_{\cc,\sigma}^\infty(\Omega)$.

  Eindeutigkeit folgt aus der Rückrichtung des Beweises. Denn existiert ein weiteres $\vartheta$ mit $\nabla\vartheta \in \GG_p(\Omega)$, das (NP) löst, liefert dies eine weitere Zerlegung von $u$, die nach Eindeutigkeit der Helmholtz-Zerlegung $\nabla(\vartheta - \pi) = 0$ impliziert.

  "$\Leftarrow$": Sei $u \in \Ell^p(\Omega; \C^d)$. Dann existiert $\pi \in \Ell^p_{\loc}(\Omega)$ mit $\nabla \pi \in \Ell^p(\Omega; \C^d)$, sodass
  \begin{displaymath}
  \int_\Omega \nabla \pi \cdot \overline f \d x = \int_\Omega u \cdot \overline f \d x, \quad\text{für alle } f \in \GG_{p'}(\Omega).
  \tag{$\ast$}
  \end{displaymath}
  Definiere $u_0 \coloneqq u - \nabla\pi$. 
  Zeige nun $u_0 \in \Ell_\sigma^p(\Omega)$:
  Aus ($\ast$) folgt zunächst $u_0 \in \GG_{p'}(\Omega)^\perp$.
  Gilt nun auch noch 
  \begin{displaymath}
    \Ell^p_\sigma(\Omega)^\perp \subset \GG_{p'}(\Omega) \tag{$\ast\ast$}
  \end{displaymath}
  so folgt die Behauptung aus nochmaliger Bildung des Annihilators, also
  $$
  u_0 \in \GG_{p'}(\Omega)^\perp \subset ( \Ell_\sigma^p(\Omega)^\perp)^\perp = \Ell^p_\sigma(\Omega).
  $$

  Weise also ($\ast\ast$) nach. Für $v \in \Ell_\sigma^p(\Omega)^\perp$ gilt per definitionem $v \in \Ell^{p'}(\Omega; \C^d)$ und
  $$
  \int_\Omega v \cdot \overline w \d x = 0, \quad\text{für alle } w \in \Ell_\sigma^p(\Omega).
  $$
  Dann liefert Lemma \ref{lem:pressureGrad}, dass ein $\phi \in \Ell_{\loc}^{p'}(\Omega)$ existiert mit
  $$
  \int_\Omega v \cdot \overline \varphi \d x = -\int_\Omega \phi \; \overline{\div(\varphi)} \d x, \quad\text{für alle } \varphi \in \CC_{\cc}^\infty(\Omega; \C^d).
  $$
  Da $v \in \Ell^{p'}(\Omega; \C^d)$, folgt $\nabla \phi \in \Ell^{p'}(\Omega; \C^d)$ und $v = \nabla \phi$.
  Hieraus ergibt sich $v \in \GG_{p'}(\Omega)$, womit die Inklusion $\Ell_\sigma^p(\Omega)^\perp \subset \GG_{p'}(\Omega)$ bewiesen wäre.

  Es bleibt die Eindeutigkeit der Zerlegung $u = u_0 + \nabla \pi$ zu zeigen.
  Angenommen es gelte
  $$
  u_0 + \nabla \pi = \tilde u_0 + \nabla \tilde \pi, \quad\text{für }
  u_0,\tilde u_0 \in \Ell_\sigma^p(\Omega) \text{ und } \nabla \pi, \nabla \tilde \pi \in \GG_p(\Omega).
  $$
  Dies ist äquivalent dazu, dass 
  $$
  v \coloneqq u_0 - \tilde u_0 = \nabla( \tilde \pi - \pi) \eqqcolon \nabla \phi \in \Ell_\sigma^p.
  $$
  Wegen $\Ell_\sigma^p(\Omega) \subset \GG_{p'}(\Omega)^\perp$ folgt
  $$
  \int_\Omega \nabla \phi \cdot \overline f \d x = 0, \quad\text{für alle } f \in \GG_{p'}(\Omega).
  $$
  Die Eindeutige Lösbarkeit (bis auf Addition von Konstanten) von (NP) liefert $\nabla\phi = 0$.
\end{proof}

Satz \ref{thm:helmholtzIffNeumann} wird benutzt, um die Existenz der Helmholtz-Zerlegung auf $\Ell^p(\Omega; \C^d)$ zu beweisen.
Auf beschränkten Lipschitz-Gebieten wurde z.B. folgendes Resultat durch Fabes, Mendez und Mitrea im Jahr 1998 \cite{fabes} bewiesen.

\begin{hsatz}
  \label{hsatz:contProj}
  Sei $\Omega \subset \R^d, d \geq 3$, ein beschränktes Lipschitz-Gebiet.
  Dann existiert $\varepsilon = \varepsilon(\Omega, d) > 0$, sodass für alle $\frac{3}{2} - \varepsilon < p < 3 + \varepsilon$ die Helmholtz-Zerlegung auf $\Ell^p(\Omega; \C^d)$ existiert.
  Weiterhin ist die Projektion
  $$
  \PP \colon \Ell^p(\Omega; \C^d) \to \Ell^p(\Omega; \C^d)
  $$
  stetig.
\end{hsatz}

\begin{rem}
  \begin{itemize}
    \item Im Falle $d = 2$ gilt Hauptsatz \ref{hsatz:contProj} für $\frac{4}{3} - \varepsilon < p < 4 + \varepsilon$.
    \item Das Intervall $\frac{3}{2} - \varepsilon < p < 3 + \varepsilon$ in Hauptsatz \ref{hsatz:contProj} ist scharf, d.h., für jedes $p \in (1,\infty) \setminus [\frac{3}{2}, 3]$ existiert ein beschränktes Lipschitz-Gebiet $\Omega$, sodass die Helmholtz-Zerlegung auf $\Ell^p(\Omega; \C^d)$ nicht existiert.
    \item Ist $\Omega$ beschränkt mit $\CC^1$-Rand oder konvex, so gilt Hauptsatz \ref{hsatz:contProj} für $1 < p < \infty$.
    \item Es existieren unbeschränkte $\CC^\infty$-Gebiete, sodass die Helmholtz-Zerlegung für $p$ genügend groß (aber auch für $p$ genügend nah bei $1$) nicht existiert.
  \end{itemize}
\end{rem}

\begin{prop}
  Sei $\Omega \subset \R^d, d\geq 2$, ein beschränktes Gebiet und $1< p< \infty$ derart, dass die Helmholtz-Zerlegung auf $\Ell^p(\Omega; \C^d)$ existiert und die zugehörige Projektion $\PP_p$ mit Bild $\Ell^p_\sigma(\Omega)$ beschränkt ist.
  Dann existiert die Helmholtz-Zerlegung auf $\Ell^{p'}(\Omega; \C^d)$, wobei $\frac{1}{p} + \frac{1}{p'} = 1$, die zugehörige Projektion $\PP_{p'}$ mit Bild $\Ell^{p'}_\sigma(\Omega)$ ist beschränkt, es gilt $(\PP_p)' = \PP_{p'}$ in dem Sinne, dass die Adjungierte von $\PP_p$ kanonisch als Operator auf $\Ell^{p'}(\Omega; \C^d)$ aufgefasst wird.
  Weiterhin gilt $(\Ell_\sigma^p(\Omega))' \simeq \Ell_\sigma^{p'}(\Omega)$.
\end{prop}

\begin{proof}
  Wir wissen, dass aus der Beschränktheit von $\PP_p$ auch die Beschränktheit von $(\PP_p)'$ folgt.
  Ist $\PP_p$ eine Projektion, so ist insbesondere $(\PP_p)'$ eine Projektion.
  Sei nun $\PP_{p'}$ die kanonische Identifizierung von $(\PP_p)'$ auf $\Ell^{p'}(\Omega; \C^d)$.
  Der Beweis gliedert sich in 3 Schritte.
  \begin{enumerate}[(i)]
    \item In diesem Schritt bestimmen wir das Bild der Projektion, genauer wollen wir zeigen, dass $\Rg \PP_{p'} = \Ell_\sigma^{p'}(\Omega)$ gilt.
      Seien dazu $\varphi \in \CC_{\cc, \sigma}^\infty(\Omega)$, $f \in \CC_{\cc}^\infty(\Omega; \CC^d)$.
  Dann gilt mit der Definition der dualen Abbildung
  \begin{align*}
    \int_\Omega \PP_{p'}\varphi \cdot \overline f 
    = \int_\Omega \varphi \cdot \overline{\PP_p f} \d x  
    = \int_\Omega \varphi \cdot \overline{\PP_2 f} \d x 
    = \int_\Omega \varphi \cdot \overline{f} \d x,
  \end{align*}
      da $\Omega$ beschränkt (es gilt $\Ell^2 \subseteq \Ell^p$ oder $\Ell^p \subseteq \Ell^2$) und die Helmholtz-Zerlegung eindeutig ist und schließlich auch $\PP_2$ selbstadjungiert ist.
  Hieraus ergibt sich $\PP_{p'} \varphi = \varphi$.
  Da $\CC_{\cc, \sigma}^\infty(\Omega)$ dicht liegt in $\Ell^{p'}_\sigma(\Omega)$, $\PP_{p'}$ beschränkt ist und zudem als Projektion ein abgeschlossenes Bild besitzt, folgt
  $$
  \Ell^{p'}_\sigma(\Omega) \subset \Rg(\PP_{p'}).
  $$
  Da per constructionem $\Ell_\sigma^{p'}(\Omega)$ abgeschlossen ist, gilt
  $$
  \Rg(\PP_{p'}) \subset \Ell_\sigma^{p'}(\Omega)\quad \text{genau dann, wenn}\quad \Ell_\sigma^{p'}(\Omega)^\perp \subset \Rg(\PP_{p'})^\perp.
  $$
  Zeige nun die rechte Seite der Äquivalenz für die noch ausstehende Inklusion.
  Sei $f \in \Ell^{p'}_\sigma(\Omega)^\perp$.
  Dann gilt
  $$
  \int_\Omega f \cdot \overline v \d x = 0, \quad\text{für alle } v \in \Ell_\sigma^{p'}(\Omega)
  $$
  und $f \in \Ell^p(\Omega; \C^d)$.
  Mit Lemma \ref{lem:pressureGrad} folgt nun die Existenz eines $\phi \in \Ell^p_{\loc}(\Omega)$ mit
  $$
  \int_\Omega f \cdot \overline v = - \int_\Omega \varphi \; \overline{\div(v)} \d x \quad\text{für alle } v \in \CC_{\cc}^\infty(\Omega; \C^d),
  $$
  woraus sich $\nabla\varphi = f \in \Ell^p(\Omega; \C^d)$ ergibt.
  Nun gilt für $g \in \Ell^{p'}(\Omega; \C^d)$
  $$
  \int_\Omega \nabla\phi \cdot \overline{\PP_{p'} g} \d x = \int_\Omega \PP_p\nabla \phi \cdot \overline g \d x = 0,
  $$
  da $\nabla \phi \in \GG_p(\Omega) = \ker(\PP_p)$.
  Daraus folgt $f \in \Rg(\PP_{p'})^\perp$ und folglich gilt 
  $$
  \Rg(\PP_{p'}) = \Ell_\sigma^{p'}(\Omega).
  $$
\item Wir bestimmen nun den Kern von $\PP_{p'}$.
  Genauer zeigen wir
  $$
  \ker(\PP_{p'}) = \GG_{p'}(\Omega).
  $$
  Sei hierzu $\nabla \phi \in \GG_{p'}(\Omega)$, dann gilt für $f \in \CC_{\cc}^\infty(\Omega; \C^d)$
  $$
  \int_\Omega \PP_{p'}(\nabla\phi) \cdot \overline f \d x
  = \int_\Omega \nabla \phi \cdot \overline{\PP_p f} \d x
  = \lim_{n \to \infty} \int_\Omega \nabla \phi \cdot \overline{\varphi_n} \d x
  = 0,
  $$
  wobei $(\varphi_n)_{n \in \N} \subset \CC_{\cc,\sigma}^\infty(\Omega)$ mit $\lim_{n \to \infty} \varphi_n = \PP_p f \in \Ell^p_\sigma(\Omega)$.
  Daraus folgt $\GG_{p'}(\Omega) \subset \ker(\PP_{p'})$.

  Sei nun $f \in \ker(\PP_{p'})$ und $\varphi \in \CC^\infty_{c,\sigma}(\Omega)$.
  Dann gilt
  $$
  \int_\Omega f \cdot \overline\varphi \d x 
  = \int_\Omega f \cdot \overline{\PP_p \varphi} \d x
  = \int_\Omega \PP_{p'} f \cdot \overline{\varphi} \d x = 0.
  $$
  Mit Lemma \ref{lem:pressureGrad} folgt somit, dass ein $\phi \in \Ell_{\loc}^{p'}(\Omega)$ existiert mit 
  $$
  \int_\Omega f \cdot \overline\varphi \d x
  = -\int_\Omega \phi \; \overline{\div(\varphi)} \d x
  $$
  für alle $\varphi \in \CC_{\cc}^\infty(\Omega; \C^d)$.
  Daraus folgt $f = \nabla\phi \in \Ell^{p'}(\Omega; \C^d)$, also $\ker(\PP_{p'}) \subset \GG_{p'}(\Omega)$.
  Damit ist $\PP_{p'}$ die Helmholtz-Projektion auf $\Ell^{p'}(\Omega; \C^d)$.
  Insbesondere existiert die Helmholtz-Zerlegung auf $\Ell^{p'}_\sigma(\Omega)$.

\item Zeige nun, dass $(\Ell_\sigma^{p}(\Omega))' \simeq \Ell^{p'}_\sigma(\Omega)$.
  Die Inklusion $\Ell_\sigma^{p'}(\Omega) \subseteq (\Ell_\sigma^p(\Omega))'$ folgt aus Hölders Ungleichung.
  Sei $f \in (\Ell_\sigma^p(\Omega))'$.
  Setze $f$ auf $\Ell^p(\Omega; \C^d)$ fort durch
  $$
  F(g) \coloneqq f(\PP_p g), \quad g \in \Ell^p(\Omega; \C^d)
  $$
  Dann ist $F \in (\Ell^p(\Omega; \C^d))'$. Weiterhin existiert ein $\tilde f \in \Ell^{p'}(\Omega; \C^d)$ mit 
  $$
  \int_\Omega \tilde f \; \overline g \d x = f(g), \quad\text{für alle } g \in \GG_p(\Omega)
  $$
  und
  $$
  \int_\Omega \tilde f \; \overline g \d x = 0, \quad\text{für alle } g \in \GG_p(\Omega).
  $$
  Folglich ist
  $$
  \int_\Omega (I - \PP_{p'} ) \tilde f \; \overline g \d x 
  = \int_\Omega \tilde f \; \underbrace{\overline{(I - \PP_p)g}}_{\in \GG_p(\Omega)} \d x
  = 0, \quad\text{für alle } g \in \Ell^p(\Omega; \C^d).
  $$
  Dies liefert $(I - \PP_{p'}) \tilde f = 0$ woraus wiederum $\tilde f \in \Ell_\sigma^{p'}(\Omega)$ folgt.
  \end{enumerate}
\end{proof}

\begin{defn}
  Sei $X$ ein Banachraum.
  Ein Operator $B \colon \DD(B) \subset X \to X$ heißt \emph{abschließbar in} $X$, falls für alle Folgen $(x_n)_{n \in \N} \subset \DD(B)$ mit $x_n \to 0$ und für die $(B x_n)_{n \in \N} \subset X$ eine Cauchy-Folge ist, auch $\lim_{n \to \infty} B x_n = 0$ folgt.

  In diesem Fall ist der \emph{Abschluss} $\overline B \colon \DD(\overline B) \subset X \to X$ \emph{von} $B$ definiert durch
  \begin{align*}
  \DD(\overline B) &\coloneqq \{ x \in X \colon \text{es ex. } (x_n)_{n \in \N} \subset \DD(B) \text{ mit } x_n \to x \text{ und } (B x_n)_{n \in \N} \text{ ist C.F.} \}, \\
  \overline B x &\coloneqq \lim_{n \to \infty} B x_n, \text{ für alle } x \in \DD(\overline B)
\end{align*}
ein wohldefinierter abgeschlossener Operator.
\end{defn}

\begin{defn}
  Sei $\Omega \subset \R^d, d \geq 2$ ein beschränktes Gebiet und $1 < p < \infty$.
  Ist $p > 2$, so ist der Stokes-Operator $A_p$ auf $\Ell_\sigma^p(\Omega)$ definiert als der \emph{Teil von $A_2$ in $\Ell^p_\sigma(\Omega)$}, d.h.
  \begin{align*}
    \DD(A_p) &\coloneqq \{ u \in \DD(A_2) \cap \Ell_\sigma^p(\Omega) \colon A_2 u \in \Ell_\sigma^p(\Omega) \} \\
    A_p u &\coloneqq A_2 u, \quad\text{für alle } u \in \DD(A_p).
  \end{align*}
  Ist $p < 2$ und $A_2$ abschließbar in $\Ell_\sigma^p(\Omega)$, so ist der Stokes-Operator $A_p$ auf $\Ell_\sigma^p(\Omega)$ definiert als der Abschluss von $A_2$ in $\Ell_\sigma^p(\Omega)$.
\end{defn}

\begin{prop}
  Sei $\Omega \subset \R^d, d\geq 2$ ein beschränktes Gebiet und $2 < p < \infty$ derart, dass die Helmholtz-Zerlegung auf $\Ell^p(\Omega; \C^d)$ existiert und die Helmholtz-Projektion beschränkt ist.
  Sei $\frac{1}{p'} + \frac{1}{p} = 1$.
  Dann ist $\DD(A_2)$ dicht in $\Ell_\sigma^{p'}(\Omega)$ und es sind äquivalent:
  \begin{enumerate}[i)]
    \item $A_2$ ist abschließbar in $\Ell_\sigma^{p'}(\Omega)$.
    \item $A_p$ ist dicht definiert.
  \end{enumerate}
  Ist entweder i) oder ii) erfüllt, so gilt
  $$
  (A_{p'})' = A_p.
  $$
  Insbesondere ist $A_p$ abgeschlossen.
\end{prop}

\begin{proof}
  Übung.
\end{proof}

Folgendes Resultat liefert auf einem abstrakten Weg die Dichtheit eines Definitionsbereiches, falls man in der Lage ist Resolventenabschätzungen zu beweisen.  (Siehe \cite[Prop. 2.1.1]{haase}.)

\begin{thm}
  Sei $A$ ein sektorieller Operator auf einem reflexiven Banachraum $X$.
  Dann ist $A$ dicht definiert.
\end{thm}

Folgendes Durchbruchresultat ist von Shen \cite{shen}.

\begin{hsatz}
  \label{hsatz:shen}
  Sei $\Omega \subset \R^d, d \geq 3$ ein beschränktes Lipschitz-Gebiet und $\theta \in [0,\pi)$.
    Dann existiert $\varepsilon(\theta, d, \Omega) > 0$, sodass für alle
    $$
    \frac{2d}{d + 1} - \varepsilon < p < \frac{2d}{d - 1} + \varepsilon
    $$
    der Stokes-Operator $A_p$ auf $\Ell_\sigma^p(\Omega)$ sektoriell von Winkel $\theta$ ist.
    Insbesondere ist $A_p$ abgeschlossen, dicht definiert, $0 \in \rho(A_p)$ und  $-A_p$ erzeugt eine exponentiell stabile analytische Halbgruppe.
\end{hsatz}

\begin{rem}
  Deuring hat 2002 im Falle d = 3 bewiesen: Für jedes $p < 3$ existiert ein beschränktes Lipschitz-Gebiet, sodass $-A_p$ keine $C_0$-Halbgruppe auf $\Ell_\sigma^p(\Omega)$ erzeugt.
\end{rem}

Folgendes Resultat ist von Tolksdorf, 2017.

\begin{hsatz}
  \label{hsatz:tolksdorf}
  Sei $\Omega \subset \R^d, d \geq 3$ ein beschränktes Lipschitz-Gebiet.
  Dann existiert $\varepsilon = \varepsilon(d,\Omega) > 0$, sodass für alle 
  $
  \frac{2d}{d + 1} - \varepsilon < p < \frac{2d}{d - 1} + \varepsilon
  $
  gilt:
  $$
  \DD(A_p^{\frac{1}{2}}) = \WW_{0,\sigma}^{1,p}(\Omega).
  $$
  Insbesondere existiert $C > 0$, sodass für alle $u \in \DD(A_p^{\frac{1}{2}})$ gilt, dass
  $$
  C^{-1} \, \|\nabla u\|_{\Ell^p} \leq \| A^{\frac{1}{2}} u\|_{\Ell^p} \leq C \, \|\nabla u\|_{\Ell^p}.
  $$
\end{hsatz}

\begin{thm}
  \label{thm:lplqSmoothing}
  Sei $\Omega \subset \R^d$, $d \geq 3$, ein beschränktes Lipschitz-Gebiet.
  Dann existiert ein $\varepsilon >0$, sodass für alle $\frac{2d}{d + 1} - \varepsilon < p \leq q < \frac{2d}{d - 1} + \varepsilon$ gilt:
  \begin{enumerate}[i)]
    \item Für alle $t > 0$ ist $\Rg(\e^{-tA_p}) \subset \WW_{0,\sigma}^{1,p}(\Omega)$ und es existiert $C > 0$, sodass für alle $t > 0$ und $a \in \Ell_\sigma^p(\Omega)$
      $$
      \|\nabla \e^{-tA_p} a \|_{\Ell^p} \leq C t^{-\frac{1}{2}} \|a\|_{\Ell^p}.
      $$
    \item Für alle $t > 0$ ist $\Rg(\e^{-t A_p}) \subset \Ell^q(\Omega; \C^d)$ und es existiert $C > 0$, sodass für alle $t > 0$ und $a \in \Ell_\sigma^p(\Omega)$ gilt
      $$
      \|\e^{-tA_p} a\|_{\Ell^q} \leq C t^{-\frac{d}{2}(\frac{1}{p} - \frac{1}{q})} \|a\|_{\Ell^p}.
      $$
  \end{enumerate}
\end{thm}

\begin{proof}
  \begin{enumerate}[i)]
    \item Nach Hauptsatz \ref{hsatz:shen} ist $(\e^{-tA_p})_{t \geq 0}$ eine analytische Halbgruppe.
      Daraus folgt zunächst die Relation
      $$
      \Rg(\e^{-tA_p}) \subset \DD(A_p) \subset \DD(A_p^{\frac{1}{2}}) = \WW_{0,\sigma}^{1,p}(\Omega).
      $$
      Hieraus erhalten wir durch Anwendung von Hauptsatz \ref{hsatz:tolksdorf}, Satz \ref{thm:moment} und Hauptsatz \ref{hsatz:shen}
      \begin{align*}
        \|\nabla \e^{-t A_p a}\|_{\Ell^p}
        &\leq C \, \|A_p^{\frac{1}{2}} \e^{-t A_p} a\|_{\Ell^p} \\
        &\leq C\, \|\e^{-tA_p} a \|_{\Ell^p}^{\frac{1}{2}} \, \|A_p \e^{-t A_p} a\|_{\Ell^p}^{\frac{1}{2}} \\
        &\leq C\, \|a\|_{\Ell^p}^{\frac{1}{2}} \; t^{-\frac{1}{2}} \|a\|_{\Ell^p}^{\frac{1}{2}} 
      \end{align*}

    \item Gagliardo-Nirenberg \ref{hsatz:gagliardoNirenberg} mit $p = q$, $r = p$, $q = p$, $j = 0$ und $m = 1$ liefert
      $$
      \frac{1}{q} = \alpha\cdot \big(\frac{1}{p} - \frac{1}{d}\big) + \big(1 - \alpha \big) \cdot \frac{1}{p},
      $$
      was genau dann der Fall ist, wenn
      $$
      \alpha = d \, \big( \frac{1}{p} - \frac{1}{q} \big).
      $$
      Da $0 \leq \alpha < 1$ gelten muss, betrachte erst den Fall
      $$
      \frac{1}{p} - \frac{1}{q} < \frac{1}{d}
      $$
      Identifiziere $\e^{-tA_p} a \in \WW_{0,\sigma}^{1,p}(\Omega)$ mit seiner Fortsetzung durch Null auf $\R^d$, d.h. $\e^{-t A_p} a \in \WW^{1,p}(\R^d)$.
      Die Gagliardo-Nirenberg Ungleichung liefert
      $$
      \|\e^{-t A_p} a \|_{\Ell^q} \leq C \| \nabla \e^{-t A_p} a \|_{\Ell^p}^\alpha \|\e^{-t A_p} a \|_{\Ell^p}^{1 - \alpha}
      \leq C \cdot t^{-\frac{d}{2} \, \big( \frac{1}{p} - \frac{1}{q} \big)} \|a\|_{\Ell^p}.
      $$
      Falls $\frac{1}{p} - \frac{1}{q} < \frac{2}{d}$, dann wähle $r$ mit $\frac{1}{p} - \frac{1}{r} > \frac{1}{d}$ und $\frac{1}{r} - \frac{1}{q} < \frac{1}{d}$.
      Anwendung des Halbgruppengesetze ergibt
      \begin{align*}
        \|\e^{-tA_p} a\|_{\Ell^q} 
        &\leq C \, t^{-\frac{d}{2} \, \big( \frac{1}{r} - \frac{1}{q} \big)} \|\e^{-\frac{t}{2} A_p} a \|_{\Ell^r}  
        \leq C\, t^{-\frac{d}{2} \, \big( \frac{1}{r} - \frac{1}{q} \big) - \frac{d}{2} \, \big( \frac{1}{p} - \frac{1}{r} \big)} \|a\|_{\Ell^p}.
      \end{align*}
      Iterativ folgt hieraus die Behauptung. \qedhere
  \end{enumerate}
\end{proof}
