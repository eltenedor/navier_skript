\chapter{Analytische Halbgruppen und gebrochene Potenzen}

In diesem Kapitel geht es darum, für eine möglichst große Klasse von abgeschlossenen Operatoren $A \colon \DD(A) \subset X \to X$, wobei $X$ ein Banachraum über $\C$ ist, die Ausdrücke $\e^{tA}$ und $A^\alpha$, $\alpha > 0$, $\alpha \in \R$ zu definieren und ihre Eigenschaften zu untersuchen.
Hauptgedanke ist hier, dass man für bestimmte holomorphe Funktionen $f$ die Cauchysche Integralformel
$$
f(z) = \frac{1}{2\pi \ii} \int_\gamma \frac{f(\lambda)}{\lambda - z} \d \lambda
$$
als Definition für $f(A)$ nimmt, indem man $(\lambda - z)^{-1}$ durch $(\lambda - A)^{-1}$ ersetzt.

Sei $I \subset \R$ ein Intervall, $X$ ein Banachraum und $f \colon I \to X$ stetig.
Ist $I$ kompakt, so konvergieren die Riemann-Summen $\sum_k l(\Delta_k) f(\xi_k)$, wobei $(\Delta_k)_k$ eine endliche Partition von $I$ bildet, $\xi_k \in \Delta_k$ und $l(\Delta_k)$ die Länge von $\Delta_k$ bezeichnet, gegen ein eindeutiges Element $x \in X$.
Definiere
$$
\int_I f(t) \d t \coloneqq x.
$$
Ist $I$ nicht kompakt und $t \mapsto \|f(t)\|_X$ uneigentlich Riemann-integrierbar, so existiert für alle kompakten Intervalle $I_k$ mit $I_k \subset I_{k + 1} \subset I$ und $\bigcup_k I_k = I$ der eindeutige Grenzwert
$$
\lim_{k \to \infty} \int_{I_k} f(t) dt \eqqcolon \int_I f(t) \d t \in X.
$$
In allen Fällen gilt
$$
\Big\| \int_I f(t) \d t \Big\|_X \leq \int_I \|f(t) \|_X \d t.
$$
Ist $\Gamma \subset \C$ eine Kurve mit stückweise stetig differenzierbarer $\CC^1$-Parametrisierung $\gamma \colon I \to \C$, $I \subset \R$ Intervall, $f \colon \Gamma \to X$ stetig, sodass $t \mapsto \|\gamma'(t) f(\gamma(t)) \|_X$ (uneigentlich) Riemann-integrierbar ist, so definiere
$$
\int_\Gamma f(z) \d z \coloneqq \int_I \gamma'(t) f(\gamma(t)) \d t.
$$

\section{Analytische Halbgruppen}

Im Folgenden bezeichnet $X$ immer einen Banachraum über $\C$.

\begin{defn}
  Sei $A \colon \DD(A) \subset X \to X$ abgeschlossen und $\omega \in [0,\pi)$.
    $A$ heißt \emph{sektoriell von Winkel} $\omega$, falls $\sigma(A) \subset \overline{\Sec_\omega}$, wobei
    $$
    \Sec_\omega \coloneqq \begin{cases} (0,\infty), \quad&\omega = 0 \\ \{z \in \C \setminus\{0\} \colon |\arg(z)| < \omega\}, \quad&\omega \neq 0 \end{cases}
    $$
    und für alle $\phi \in (\omega, \pi)$ ein $C_\phi > 0$ existiert, sodass für alle $\lambda \in \C \setminus \overline{\Sec_\phi}$ gilt, dass
    $$
    \|\lambda(\lambda - A)^{-1}\|_{\Li(X)} \leq C_\phi.
    $$
\end{defn}

\begin{ntion}
  Für $R > 0$ und $\theta \in (0,\pi)$ bezeichne mit $\gamma_{R, \theta}$ die kanonische Parametrisierung der Kurve, welche durch $\partial(\Sec_\theta \cup \BB(0,R))$ gegeben ist.
  Weiterhin bezeichne $\gamma_1$ die Parametrisierung des Geradenstücks in der oberen Halbebene, $\gamma_3$ in der unteren und $\gamma_2$ des Kreisbogens.
\end{ntion}

\begin{obs}
  \label{obs:integrablePath}
  Ist $A$ sektoriell von Winkel $\omega \in [0,\frac{\pi}{2} )$, $\theta \in (\omega, \frac{\pi}{2})$ und $z \in \Sec_{\frac{\pi}{2} - \theta}$, so ist 
  $$
  t \mapsto \|\gamma'_{R,\theta}(t) \e^{-z\gamma_{R,\theta}(t)} \left( \gamma_{R, \theta}(t) - A \right)^{-1} \|_{\Li(X)}
  $$
  uneigentlich Riemann integrierbar: 
  Wegen Symmetrie und Holomorphie der Resolvente auf $\C \setminus \overline{\Sec_\omega}$ genügt es Integrierbarkeit auf $\gamma_1$ nachzuweisen.
  Aus der Sektorialität von $A$ folgt zunächst
  $$
    \int_{R}^\infty \| \e^{\ii\theta} \e^{-zt \e^{\ii\theta}} ( t\e^{\ii\theta} - A )^{-1} \|_{\Li(X)} \d t
    \leq C_\theta \int_R^\infty \e^{-t \operatorname{Re}(z \e^{i \theta})} t^{-1} \d t.
  $$
  Dieses Integral ist endlich, da
  $$
  |\arg(z \e^{i \theta})| \leq |\arg(z)| + \theta < \frac{\pi}{2} - \theta + \theta = \frac{\pi}{2}
  $$
  und damit $\operatorname{Re}z \e^{\ii\theta} > 0$ folgt.
\end{obs}

\begin{defn}
  Sei $A$ sektoriell von Winkel $\omega \in [0,\frac{\pi}{2})$ und $z \in \Sec_{\frac{\pi}{2} - \omega}$.
    Wähle $R > 0$ und $\theta \in (\omega, \frac{\pi}{2} - |\arg(z)|)$.
    Definiere
    $$
    \e^{-zA} \coloneqq \frac{1}{2\pi \ii} \int_{\gamma_{R, \theta}} \e^{-z\lambda } (\lambda - A)^{-1} \d \lambda
    $$
    und $\e^{-0A} \coloneqq I$.
    Die Familie $(\e^{-zA})_{z \in \Sec_{\frac{\pi}{2} - \omega \cup \{0\}}}$  wird \emph{beschränkte analytische Halbgruppe genannt} und falls $A$ dicht definiert ist, wird $-A$ Erzeuger/Generator von $(\e^{-zA})_{z \in \Sec_{\frac{\pi}{2} - \omega \cup \{0\}}}$ genannt.
\end{defn}

\begin{lem}
  \label{lem:welldefinedsg}
  Die Definition von $\e^{-zA}$ ist unabhängig von der Wahl von $R$ und $\theta$.
\end{lem}

\begin{proof}
  Übung.
\end{proof}

\begin{prop}
  \label{prop:opInIntegral}
  Sei $I \subset \R$ ein Intervall, $f \colon I \to X$ stetig und uneigentlich Riemann integrierbar, $Y$ ein Banachraum, $T \in \Li(X,Y)$ und $A \colon \DD(A) \subset X \to Y$ abgeschlossen.
  \begin{enumerate}[(i)]
    \item Dann ist $Tf \colon I \to Y$ stetig und uneigentlich Riemann integrierbar und es gilt
      $$
      T \int_I f(t) \d t = \int_I Tf(t)\d t.
      $$
    \item Falls $f(t) \in \DD(A)$ für alle $t \in I$ gilt und $Af \colon I \to Y$ stetig und uneigentlich Riemann-integrierbar ist, dann ist $\int_I f(t) \d t \in \DD(A)$ und es gilt 
      $$
      A \int_I f(t) \d t = \int_I A f(t) \d t.
      $$
  \end{enumerate}
\end{prop}

\begin{proof}
  Übung.
\end{proof}

\begin{thm}
  Sei $A$ sektoriell von Winkel $\omega \in [0,\frac{\pi}{2})$.
    Dann ist für alle $z \in \Sec_{\frac{\pi}{2} - \omega}$ der Operator $\e^{-zA}$ in $\Li(X)$ und erfüllt
    \begin{enumerate}[(i)]
      \item Für alle $0 \leq \phi < \frac{\pi}{2} - \omega$ ist $(e^{-zA})_{z \in \Sec_\phi}$ gleichmäßig beschränkt.
      \item $z \mapsto \e^{-zA}$ ist analytisch in $\Sec_{\frac{\pi}{2} - \omega}$.
      \item Für alle $z,w \in \Sec_{\frac{\pi}{2} - \omega}$ gilt 
        $
        \e^{-(z + w)A} = \e^{-zA}\e^{-w A}.
        $
      \item Ist $A$ zusätzlich dicht definiert, so ist für alle $0 \leq \phi < \frac{\pi}{2} - \omega$ die Abbildung
        $$
        \Sec_\phi \cup \{0\} \ni z \mapsto \e^{-z A} \in \Li(X)
        $$
        stark stetig in $z = 0$, d.h. für alle $x \in X$ gilt
        $$
        \lim_{\substack{z \to 0 \\ z \in \Sec_\phi}} \|\e^{-zA} x - x\|_X = 0.
        $$
    \end{enumerate}
\end{thm}

\begin{proof}
  \begin{enumerate}[(i)]
    \item Für festes $\phi$ wähle $R > 0$ und $\theta \in (\omega, \frac{\pi}{2} - \phi)$, sodass  $|\arg(z \e^{\pm \ii\theta})| \leq \phi + \theta < \frac{\pi}{2}$ für alle $z \in \Sec_\phi$.
      Mit Beobachtung \ref{obs:integrablePath} folgt für $j \in \{1,3\}$
      \begin{align*}
        \Big\| \int_{\gamma_{j}} \e^{-z \lambda} (\lambda - A)^{-1} \d \lambda \Big\|_{\Li(X)} 
        &\leq C\int_R^\infty \e^{-t \operatorname{Re}(z \e^{\pm \ii\theta})} t^{-1} \d t
        \leq C \int_R^\infty \e^{-t |z| \cos(\theta + \phi)} t^{-1} \d t \\
        &= C \int_{R|z|}^\infty \e^{-t \cos(\phi+\theta)} t^{-1} \d t.
      \end{align*}
      Nach Lemma \ref{lem:welldefinedsg} hängt der Wert dieses Integrals nicht von der Wahl von $R$ ab. 
      Im Folgenden wähle daher $R = \frac{1}{|z|}$.
      Mit dieser Wahl gilt nun für das Kurvenintegral entlang $\gamma_2$
      \begin{align*}
        \Big\|\int_{\gamma_2} \e^{-z \lambda} (\lambda - A)^{-1} \d \lambda \Big\|_{\Li(X)}
        \leq C\int_\theta^{2\pi - \theta} \frac{1}{|z|} \big|e^{-\frac{z}{|z|}} \e^{i \varphi} \big| \, |z| \d \varphi
        \leq C \, 2\pi \e,
      \end{align*}
      da $|\e^z| \leq \e^{|z|}$.
      Folglich ist $\e^{-z A} \in \Li(X)$ und $(\e^{-zA})_{z \in \Sec_\phi}$ ist gleichmäßig beschränkt.

    \item Wie in Beobachtung \ref{obs:integrablePath} zeigt man erst, dass $\lambda \mapsto \lambda \e^{-z\lambda}(\lambda - A)^{-1}$ absolut integrierbar auf $\gamma_{\theta,R}$ für $\theta \in (\omega, \frac{\pi}{2} - \phi)$ ist, wobei $\phi$ wie in (i) gewählt sei.
      Außerdem ist für $z \in \Sec_\phi$ und $h \in \C\setminus\{0\}$ mit $z + h \in \Sec_\phi$ 
      $$
      \left[ \frac{1}{h} \left( \e^{-(z + h)\lambda} - \e^{-z\lambda} \right) - (- \lambda \e^{-z\lambda}) \right] (\lambda - A)^{-1}
      = \left[ \frac{1}{h\lambda} \left( \e^{-h\lambda} - 1\right) + 1\right] \lambda \e^{-z\lambda}  (\lambda - A)^{-1}
      $$
      auf jedem kompakten Teilweg von $\gamma_{\theta,R}$ gleichmäßig konvergent (mit Grenzwert 0), da $\e^{-z\lambda}$ holomorph und damit insbesondere stetig komplex differenzierbar ist.
      Weiter gilt für $|h| < c$
      \begin{align*}
        \Big| \frac{1}{h\lambda} (\e^{-h\lambda} - 1) + 1 \Big|
        &= \Big| \frac{1}{h\lambda} \big(\sum_{k = 1}^\infty \frac{(-h\lambda)^{n}}{n!} + h\lambda \big) \Big| \\
        &= \Big| \sum_{k = 2}^\infty \frac{(-h \lambda)^{n - 1}}{n!} \Big|
        \leq \sum_{n = 2}^\infty \frac{(|h| \, |\lambda|)^{n - 1}}{n!} \\
        &\leq \sum_{n = 2}^\infty \frac{(c |z| \, |\lambda|)^{n - 1}}{n!}
        = \frac{1}{c|z|\, |\lambda|} (\e^{c\,|z|\,|\lambda|} - 1) - 1,
      \end{align*}
      woraus wiederum
      \begin{align*}
        \left( \frac{1}{c\, |z|\, |\lambda|} (\e^{c\, |z| |\lambda|} - 1) + 1 \right) & | \lambda \e^{-z\lambda} | \| (\lambda - A)^{-1} \| \\
        &\quad\overset{\mathrm{(i)}}{\leq} \left( \frac{1}{c\, |z|\, |\lambda|} (\e^{c\, |z| |\lambda|} - 1) - 1 \right) | \lambda| \e^{-|z|\cos(\phi + \theta) |\lambda|} \frac{C}{|\lambda|}
      \end{align*}
      folgt.
      Wähle nun $c < \cos(\phi + \theta)$.
      Daraus folgt die uniforme Integrierbarkeit für $|h|$ klein, was wiederum
      $$
      \frac{1}{h} \left(\e^{-(z + h)A} - \e^{-zA} \right) \to 
      \frac{1}{2\pi \ii} \int_{\gamma_{R, \theta}} \lambda \e^{-z\lambda}(\lambda - A)^{-1} \d \lambda, \quad\text{für } h \to 0
      $$
      impliziert.

    \item Sei $x \in X, x' \in X'$.
      Dann gilt mit $R_w < R_z$ und $\theta_w < \theta_z$:
      \begin{align*}
        \langle \e^{-zA} \e^{-wA} x, x' \rangle
        &= \frac{1}{2\pi \ii} \langle \int_{\gamma_{R_z, \theta_z}} \e^{-z\lambda} (\lambda - A)^{-1} \e^{-wA} x \d \lambda, x' \rangle \\
        &= \frac{1}{2\pi \ii} \int_{\gamma_{R_z, \theta_z}} \e^{-z\lambda} \langle (\lambda - A)^{-1} \e^{-wA} x, x' \rangle \d \lambda\\
        &= \frac{1}{(2\pi \ii)^2} \int_{\gamma_{R_z, \theta_z}} \int_{\gamma_{R_w, \theta_w}}  \e^{-z\lambda} \e^{-w \mu} \langle (\lambda - A)^{-1} (\mu - A)^{-1} x, x' \rangle \d \mu \d \lambda\\
        &= \frac{1}{(2\pi \ii)^2} \int_{\gamma_{R_z, \theta_z}} \int_{\gamma_{R_w, \theta_w}}  \frac{\e^{-z\lambda} \e^{-w \mu}}{\mu - \lambda} \langle (\lambda - A)^{-1} x, x' \rangle \d \mu \d \lambda\\
        &\quad - \frac{1}{(2\pi \ii)^2} \int_{\gamma_{R_z, \theta_z}} \int_{\gamma_{R_w, \theta_w}}  \frac{\e^{-z\lambda} \e^{-w \mu}}{\mu - \lambda} \langle (\mu - A)^{-1} x, x' \rangle \d \mu \d \lambda\\
        &= - \frac{1}{(2\pi \ii)^2} \int_{\gamma_{R_z, \theta_z}} \int_{\gamma_{R_w, \theta_w}}  \frac{\e^{-z\lambda} \e^{-w \mu}}{\mu - \lambda} \langle (\mu - A)^{-1} x, x' \rangle \d \mu \d \lambda\\
        &= - \frac{1}{(2\pi \ii)^2} \int_{\gamma_{R_w, \theta_w}} \int_{\gamma_{R_z, \theta_z}}  \frac{\e^{-z\lambda} \e^{-w \mu}}{\mu - \lambda} \langle (\mu - A)^{-1} x, x' \rangle \d \lambda \d \mu \\
        &= - \frac{1}{(2\pi \ii)^2} \int_{\gamma_{R_w, \theta_w}} \int_{\gamma_{R_z, \theta_z}}  \frac{\e^{-(z+ w)\lambda}}{\mu - \lambda} \langle (\mu - A)^{-1} x, x' \rangle \d \lambda \d \mu\\
        &= \langle \e^{-(z + w)A} x, x' \rangle.
      \end{align*}
      Hahn-Banach liefert sodann $\e^{-zA}\e^{-wA} x = \e^{-(z + w)A}x$ für alle $x \in X$. 

    \item Sei $z \in \Sec_\phi$. 
      Dann liefert die Cauchy-Integralformel (für unbeschränkte Integrale)
      $$
      x 
      = \e^{-0z} x
      = \frac{1}{2\pi \ii} \int_{\gamma_{1,\theta}} \frac{\e^{-\lambda z}}{\lambda - 0} x \d \lambda, \quad\text{für alle } x \in X,
      $$
      wobei $\theta \in (\omega, \frac{\pi}{2} - \phi)$.
      Für $x \in \DD(A)$ folgt damit
     \begin{align*}
       \e^{-zA} x - x 
       &= \frac{1}{2\pi \ii} \int_{\gamma_{1,\theta}} \e^{-z\lambda} \Big( (\lambda - A)^{-1} x - \frac{x}{\lambda} \Big) \d \lambda\\
       \intertext{und, da $(\lambda - A)^{-1} x - \lambda^{-1} x = \lambda^{-1}(\lambda(\lambda - A)^{-1}x - x) = \lambda^{-1} A (\lambda - A){-1}$, folgt weiter}
       &= \frac{1}{2\pi \ii} \int_{\gamma_{1,\theta}} \frac{\e^{-z\lambda}}{\lambda}  \lambda (\lambda - A)^{-1} A x \d \lambda \\
       &\to \frac{1}{2\pi \ii} \int_{\gamma_{1,\theta}} \frac{1}{\lambda} (\lambda - A)^{-1} A x \d \lambda, \quad \text{für } z \to 0.
     \end{align*}
     Zudem gilt die folgende Majorisierung
     $$
     \| \frac{1}{\lambda} (\lambda - A)^{-1} A x \| \leq \frac{C}{|\lambda|^2} \|A x \|,
     $$
     woraus 
     $$
     \frac{1}{2\pi \ii} \int_{\gamma_{1,\theta}} \frac{1}{\lambda} (\lambda - A)^{-1} A x \d \lambda = \lim_{R \to \infty} \frac{1}{2\pi \ii} \int_{\gamma_{1,\theta}^R} \frac{1}{\lambda} (\lambda - A)^{-1} A x \d \lambda.
     $$
     Mit dem Cauchyschen Integralsatz (teste wieder mit $x' \in X'$) folgt
     $$
     \frac{1}{2\pi \ii} \int_{\gamma_{1,\theta}^R} \frac{1}{\lambda} (\lambda - A)^{-1} A x \d\lambda = 0 \quad\text{für alle }R > 1.
     $$

     Aus (i) und der vorausgesetzten Dichtheit von $\DD(A)$ ergibt sich schließlich
     \[
     \lim_{\substack{z \to 0\\ z \in \Sec_\phi}} \|\e^{-zA} x - x \| = 0 \quad\text{für alle } x \in X. \qedhere
     \]
  \end{enumerate}
\end{proof}

\begin{rem}
  Um Resultate von skalarwertigen Integralen auf banachraumwertige zu übertragen, ist es üblich mit Funktionalen zu testen, dann das skalarwertige Resultat zu benutzen und am Ende Hahn-Banach anzuwenden.
\end{rem}

\begin{thm}
  Sei $A$ sektoriell von Winkel $\omega \in [0,\frac{\pi}{2})$ und $z \in \Sec_{\frac{\pi}{2} - \omega}$.
    Dann ist $\operatorname{Rg}(\e^{-zA}) \subset \DD(A)$ (Glättungseigenschaft) und falls $x \in \DD(A)$ gilt $A \e^{-zA} x = \e^{-z A} A x$.
    Weiterhin existiert $C > 0$, sodass $\sup_{t > 0} \|t A \e^{-t A} \|_{\Li(X)} \leq C$.
\end{thm}

\begin{proof}
  Sei $R > 0$ und $\theta \in (w, \frac{\pi}{2} - |\arg(z)|)$.
  Dann sind
  $$
  \lambda \mapsto \e^{-z\lambda}(\lambda - A)^{-1} \quad\text{und}\quad \lambda \mapsto \e^{-z\lambda} A(\lambda - A)^{-1} = \e^{-z\lambda} \lambda (\lambda - A)^{-1} - \e^{-z\lambda} 
  $$
  auf $\gamma_{R,\theta}$ integrierbar.
  Proposition \ref{prop:opInIntegral} liefert $\Rg(\e^{-zA}) \subset \DD(A)$ und 
  $$
  A \e^{-zA} = \frac{1}{2\pi \ii} \int_{\gamma_{R,\theta}} \e^{-z\lambda} A(\lambda - A)^{-1} \d \lambda.
  $$
  Ist $x \in \DD(A)$ gilt folglich $A \e^{-zA} x = \e^{-z A} A x$.
  Für die zweite Aussage sei $t > 0,  R = \frac{1}{t}$ und $\theta \in (w, \frac{\pi}{2})$.
  Dann gilt
  \begin{align*}
    A \e^{-tA}
    &= \frac{1}{2 \pi \ii} \int_{\gamma_{t^{-1},\theta}} (\e^{-t\lambda}\lambda A (\lambda - A)^{-1} \d \lambda \\
    &= \frac{1}{2 \pi \ii} \int_{\gamma_{t^{-1},\theta}} (\e^{-t\lambda} \lambda (\lambda - A)^{-1} - \e{-z\lambda}) \d \lambda\\
    &= \frac{1}{2 \pi \ii} \int_{\gamma_{t^{-1},\theta}} (\e^{-t\lambda} \lambda (\lambda - A)^{-1} \d \lambda.
  \end{align*}
  Sektorialität liefert
  \begin{align*}
    \| A \e^{-t A} \|_{\Li(X)}
    &\leq C\, \Big( \int_{t^{-1}}^\infty \e^{-tr \cos \theta} \d r + \int_\theta^{2\pi - \theta} t^{-1} \e^{-t t^{-1} \cos \varphi} \d \varphi\Big) \\
    &\leq C\, \Big( \int_1^\infty t^{-1} \e^{-s\cos \theta} \d s + \int_\theta^{2\pi - \theta} t^{-1} \e^{-\cos \varphi} \d \varphi \Big). \qedhere
  \end{align*}
\end{proof}

\section{Gebrochene Potenzen}

In diesem Abschnitt definieren und untersuchen wir gebrochene Potenzen $A^\alpha$.

\begin{prop}
  \label{prop:resolventEstImproved}
  Sei $A$ sektoriell von Winkel $\omega \in [0,\pi)$ und $0 \in \rho(A)$.
    Dann existiert ein $R > 0$, sodass für alle $\theta \in (\omega, \pi)$ ein $C > 0$ existiert, sodass $\BB_R(0) \subset \rho(A)$ und für alle $\lambda \in \C \setminus \overline{\Sec_\theta} \cup \BB_R(0)$
    $$
    \| (1 + |\lambda| ) (\lambda - A)^{-1} \|_{\Li(X)} \leq C
    $$
    gilt.
\end{prop}

\begin{proof}
  Übung.
\end{proof}

\begin{ntion}
  Seien $a > 0$ und $\theta \in (0,\pi)$.
  Dann definieren wir $\Gamma_{a,\theta} \coloneqq \Gamma_1 - \Gamma_2$, wobei
  $$
    \Gamma_1 \colon [0,\infty) \to \C, t \mapsto a + t\e^{\ii\theta} \quad\text{und} \quad
    \Gamma_2 \colon [0,\infty) \to \C, t \mapsto a + t\e^{-\ii\theta}.
  $$
\end{ntion}

\begin{defn}
  Sei $A$ sektoriell von Winkel $\omega \in (0,\pi)$ und $0 \in \rho(A)$.
  Sei $\theta \in (\omega,\pi)$ und $0 < a < R$, mit $R > 0$ aus Proposition \ref{prop:resolventEstImproved}.
  Definiere für $\alpha > 0$
  $$
  A^{-\alpha} \coloneqq \frac{1}{2\pi \ii} \int_{\Gamma_{a,\theta}} \lambda^{-\alpha} (\lambda - A)^{-1} \d \lambda.
  $$
\end{defn}

\begin{prop}
  \label{prop:natPow}
  Sei $A$ sektoriell von Winkel $\omega \in (0,\pi)$ und $0 \in \rho(A)$.
  Dann ist für $\alpha > 0$ die Definition von $A^{-\alpha}$ unabhängig von $a$ sowie von $\theta$ und es gilt $A^{-\alpha} \in \Li(X)$. Falls $\alpha \in \N$, so stimmt $A^{-\alpha}$ mit der $\alpha$-ten Potenz von $A^{-1}$ überein.
\end{prop}

\begin{proof}
  Übung.
\end{proof}

\begin{thm}
  \label{thm:negFracPower}
  Sei $A$ sektoriell von Winkel $\omega \in (0,\pi)$ und $0 \in \rho(A)$.
  Weiterhin sei $n \in \N_0$ und $\alpha \in (0,n+1) \setminus \N$.
  Dann gilt
  $$
  A^{-\alpha} = \frac{1}{\pi} \frac{n!}{\prod_{i = 1}^n(i - \alpha)} \sin(\alpha\pi) \int_0^\infty t^{n - \alpha} ( t + A)^{-(n+1)} \d t.
  $$
\end{thm}

\begin{proof}
  $n$-fache partielle Integration liefert
  \begin{align*}
    A^{-\alpha}
    &= \frac{1}{2\pi \ii} \int_{\Gamma_{a, \theta}} \lambda^{-\alpha} (\lambda - A)^{-1} \d \lambda \\
    &= \frac{1}{2\pi \ii} \frac{n!}{\prod_{i = 1}^n (i - \alpha)} \int_{\Gamma_{a , \theta}} \lambda^{n - \alpha} (\lambda - A)^{-(n+1)} \d \lambda \\
    \intertext{und mit der Definition von $\Gamma_{a,\theta}$ gilt}
    &= \frac{1}{2\pi \ii} \frac{n!}{\prod_{i = 1}^n ( i - \alpha )} \Bigg[ \int_0^\infty \e^{\ii\theta} (t \e^{\ii\theta} + a )^{n - \alpha} (t \e^{\ii\theta} + a - A)^{-(n+1)} \d t  \\
    &\quad\quad\quad\quad\quad\quad\quad\quad- \int_0^\infty \e^{-\ii\theta} (t \e^{-\ii\theta} + a)^{n - \alpha} (t \e^{-\ii\theta} + a - A)^{-(n + 1)} \d t\Bigg], \\
    \intertext{woraus mit majorisierter Konvergenz dann}
    &\overset{a \to 0}{\longrightarrow} \frac{1}{2\pi \ii} \frac{n!}{\prod_{i = 1}^n ( i - \alpha )} 
    \Bigg[ \int_0^\infty \e^{\ii\theta} |t|^{n - \alpha} \e^{i(n - \alpha)\theta} (t \e^{\ii\theta} - A)^{-(n+1)} \d t  \\
    &\quad\quad\quad\quad\quad\quad\quad\quad - \int_0^\infty \e^{-\ii\theta} |t|^{n - \alpha} \e^{-i(n - \alpha)\theta} (t \e^{-\ii\theta} - A)^{-(n + 1)} \d t\Bigg],\\
    \intertext{folgt und mit nochmaliger Anwendung des Satzes von der majorisierten Konvergenz schließlich}
    &\overset{\theta \to \pi}{\longrightarrow} \frac{1}{2\pi \ii} \frac{n!}{\prod_{i = 1}^n (i - \alpha)}
    \left[ \e^{-i (n - \alpha) \pi} - \e^{i (n - \alpha)\pi} \right] \int_0^\infty t^{n -\alpha} (-t - A)^{n + 1} \d t.
  \end{align*}
  Hierbei wurde von der Tatsache Gebrauch gemacht, dass sich $|t \e^{\pm \ii \theta} + a|$ bis auf Konstanten so verhält, wie $|t| + |a|$ (siehe Lemma \ref{lem:inverseTriangle}).
\end{proof}

\begin{thm}
  \label{thm:fracPowSemigroup}
  Sei $A$ sektoriell von Winkel $\omega \in (0,\pi)$ und $0 \in \rho(A)$.
  Dann erfüllen die Operatoren $(A^{-\alpha})_{\alpha \geq 0}$, wobei $A^{-0} \coloneqq I$, das Halbgruppengesetz $A^{-\alpha - \beta} = A^{-\alpha} A^{-\beta}$, $\alpha,\beta \geq 0$.
  Ist $A$ dicht definiert, so ist die Abbildung
  $$
  [0,\infty) \ni \alpha \to A^{-\alpha}
  $$
  stark stetig.
\end{thm}

\begin{proof}
  Übung.
\end{proof}

\begin{kor}
  \label{kor:negFracPower}
  Die Identität in Satz \ref{thm:negFracPower} gilt sogar für alle $\alpha \in (0,n+1)$, indem man für $\alpha \in \N$ beide Seiten stetig fortsetzt.
\end{kor}

\begin{prop}
  Sei $A$ sektoriell von Winkel $\omega \in (0,\pi)$ und $0 \in \rho(A)$.
  Dann ist $A^{-\alpha}$ für alle $\alpha > 0$ injektiv.
\end{prop}

\begin{proof}
  Sei $n \in \N$ mit $n > \alpha$.
  Satz \ref{thm:fracPowSemigroup} liefert nun $A^{-n} = A^{-(n - \alpha)}A^{-\alpha}$.
  Nach Proposition \ref{prop:natPow} ist $A^{-n} = (A^{-1})^n$ und es folgt $A^n A^{-(n - \alpha)} A^{-\alpha} = I$.
  Damit ist $A^{-\alpha}$ injektiv.
\end{proof}

\begin{defn}
  \label{defn:posFracPow}
  Sei $A$ sektoriell von Winkel $\omega \in (0,\pi)$ und $0 \in \rho(A)$.
  Für $\alpha > 0$ definiere 
  $$
  A^\alpha \coloneqq (A^{-\alpha})^{-1}
  $$
  mit $\DD(A^\alpha) \coloneqq \Rg(A^{-\alpha})$.
\end{defn}

\begin{thm}
  Sei $A$ sektoriell von Winkel $\omega \in (0,\pi)$ und $0 \in \rho(A)$.
  Dann gilt für alle $\alpha, \beta \in \R$
  $$
    A^\alpha A^\beta x = A^{\alpha + \beta} x, \quad\text{für alle } x \in \DD(A^\gamma)),
  $$
  wobei $\gamma = \max\{\alpha, \beta, \alpha + \beta\}$.
\end{thm}

\begin{proof}
  Der Beweis folgt aus Kombination von Satz \ref{thm:fracPowSemigroup} und Definition \ref{defn:posFracPow}.
  Wir unterscheiden dazu folgende Fälle:
  \begin{enumerate}[(i)]
    \item Für $\alpha, \beta \geq 0$ gilt
  \[
  A^\alpha A^\beta x 
  = A^\alpha A^\beta (A^{-(\alpha + \beta)} A^{\alpha + \beta}) x
  = A^\alpha A^\beta (A^{-\beta} A^{-\alpha} A^{\alpha + \beta} ) x 
  = A^{\alpha + \beta} x. 
  \]
\item Für $\alpha > -\beta \geq 0$ gilt
  \[
    A^\alpha A^\beta x
      = A^{\alpha + \beta - \beta} A^\beta x
      \overset{\text{(i)}}{=} A^{\alpha + \beta} A^{-\beta} A^\beta x.
    \]
  \end{enumerate}
  Alle anderen Fälle folgen aus ähnlichen Überlegungen.
\end{proof}

\begin{thm}[Momentenungleichung]
  \label{thm:moment}
  Sei $A$ sektoriell von Winkel $\omega \in (0,\pi)$ und $0 \in \rho(A)$.
  Für alle $\alpha < \beta < \gamma$ existiert $C = C(\alpha, \beta, \gamma)$, sodass
  $$
  \| A^\beta x\|_X^{\frac{}{}} \leq C \|A^\alpha x\|_X^{\frac{\gamma - \beta}{\gamma - \alpha}} \, \|A^\gamma x \|_X^{\frac{\beta - \alpha}{\gamma - \alpha}}, \quad\text{für alle }x \in \DD(A^\gamma).
  $$
\end{thm}

\begin{proof}
Sei erst $\alpha_0 > \beta_0 > 0$ und $n \in \N$ mit $\alpha_0 \in (n, n + 1]$. 
Dann gilt insbesondere $\beta_0 \in (0,n+1)$.
Angenommen es gelten die Ungleichungen
\begin{align}
  \|s^{n - \beta_0} (s + A)^{-(n + 1)} x_0 \|_X &
  \leq C s^{\alpha_0 - \beta_0 - 1} \|A^{-\alpha_0} x\|_X \tag{1} \\
  \|s^{n - \beta_0} (s+ A)^{-(n + 1)} x_0 \|_X &
  \leq C s^{-\beta_0 - 1} \|x_0\|_X \tag{2}
\end{align}
für alle $s < 0, x_0 \in X$. 
Sei $\tau > 0$ beliebig. Dann folgt mit Satz \ref{thm:negFracPower} und Korollar \ref{kor:negFracPower}
\begin{align*}
  \|A^{-\beta_0} x_0 \|_X
  &\leq C \, \Big\|\int_0^\infty s^{n - \beta_0} (s + A)^{-(n+1)} x_0 \d s \Big\|_X \\
  &= C \, \Big\| \int_0^\tau s^{n - \beta_0} (s + A)^{-(n + 1)} x_0 \d s  + \int_\tau^\infty s^{n - \beta_0} (s + A)^{-(n + 1)} x_0 \d s\Big\|_X \\
  &\overset{\text{(1),(2)}}{\leq} C \, \Big( \int_0^\tau s^{\alpha_0 - \beta_0 - 1} \| A^{-\alpha_0} x_0 \|_X \d s + \int_\tau^\infty s^{-\beta_0 - 1} \|x_0\|_X \d s \Big) \\
  &= \frac{C}{\alpha_0 - \beta_0} \tau^{\alpha_0 - \beta_0} \|A^{-\alpha_0} x_0 \|_X + \frac{C}{\beta_0} \tau^{-\beta_0} \|x_0\|_X.
\end{align*}
  Nehme nun $\tau = \|A ^{-\alpha_0} x\|_X^{-\frac{1}{\alpha_0}} \|x_0\|_X^{\frac{1}{\alpha_0}}$. 
  Dann folgt
  $$
  \|A^{-\beta_0} x_0\|_X \leq C \, \|A^{-\alpha_0} x_0\|^{\frac{\beta_0}{\alpha_0}} \|x_0\|_X^{1 - \frac{\beta_0}{\alpha_0}}.
  $$
  Wähle nun $x_0 = A^\gamma x$, $\alpha = \gamma - \alpha_0$, $\beta = \gamma - \beta_0$, dann folgt daraus die Ungleichung.

  Beweise nun die Ungleichungen (1) und (2).
  (2) Folgt aus $(n + 1)$-facher Anwendung der Sektorialitätsabschätzung.
  Zu (1): Mit $(s + A)^{-(n + 1)} = A^{-n - 1 + \alpha_0} A (s + A)^{-1} A^n (s + A)^{-n} A^{-\alpha_0}$ gilt
  \begin{align*}
    \|s^{n - \beta_0} &(s + A)^{- (n + 1)} x_0 \|_X \\
    &\leq s^{\alpha_0 - \beta_0 - 1} \|A^{-n - 1+ \alpha_0} s^{n + 1 - \alpha} A(s + A)^{-1}\|_{\Li(X)} \|A^n(s+A)^{-n}\|_{\Li(X)} \|A^{-\alpha_0} x_0 \|_X
  \end{align*}
  Falls $\alpha_0 = n + 1$ folgt daraus bereits die Behauptung.
  Sei also im Folgenden $\alpha_0 \in (n, n + 1)$.
  Mit Satz \ref{thm:negFracPower} ergibt sich
  \begin{alignat*}{2}
    A^{-(n + 1 - \alpha_0)} s^{n + 1 - \alpha_0} A (s + A)^{-1}
    &= -\frac{1}{\pi} \sin(\alpha \pi) &&\int_0^\infty s^{n + 1 - \alpha_0} t^{-(n + 1 - \alpha_0)} A (s + A)^{-1} (t + A)^{-1} \d t \\
    &= -\frac{1}{\pi} \sin(\alpha \pi) &&\int_0^\infty s^{n + 1 - \alpha_0} \lambda^{-(n + 1 - \alpha_0)} s A(s + A)^{-1} (s\lambda + A)^{-1} \d \lambda \\
    &= -\frac{1}{\pi} \sin(\alpha \pi) \Big[ &&\int_0^1 \lambda^{-(n + 1 - \alpha_0)} s (s + A)^{-1} A (s \lambda + A)^{-1} \d \lambda \\
    &  &&+\int_1^\infty \lambda^{-(n + 2 - \alpha_0)} A (s + A)^{-1} s \lambda (s \lambda + A)^{-1} \d \lambda \Big].
  \end{alignat*}
  Mit der Sektorialität von $A$ ergibt sich Ungleichung (1).
\end{proof}
