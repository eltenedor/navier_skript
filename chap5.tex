\chapter{Die Navier-Stokes-Gleichungen im kritischen Raum $\Ell^\infty(0,T; \Ell^3_\sigma(\Omega))$}

Sei $\Omega \subset \R^3$ und $T > 0$.
Betrachte die inkompressiblen Navier-Stokes-Gleichungen
$$
\text{(NST)} \begin{cases}
  \partial_t u - \Delta u + (u \cdot \nabla) u + \nabla \pi &= 0, \quad 0 < t < T, x \in \Omega \\
  \div u &= 0, \quad 0 < t < T, x \in \Omega \\
  u &= 0, \quad 0 < t < T, x \in \partial\Omega \\
  u(0) &= a, \quad x \in \Omega.
\end{cases}
$$
Hierbei fordern wir als Kompatibilitätsbedingung, dass $a$ im richtigen Sinne ``divergenzfrei'' sei.

Betrachten wir nun für $\Omega = \R^3$, $T = \infty$ und $\lambda > 0$ die Skalierungen
\begin{align*}
  u_\lambda(x,t)  &\coloneqq \lambda u(\lambda x, \lambda^2 t), \\
  \pi_\lambda(x,t) &\coloneqq \lambda^2 \pi(\lambda x, \lambda^2 t).
\end{align*}
Man überzeugt sich leicht davon, dass falls $u$,$\pi$ das System (NST) lösen, so lösen auch $u_\lambda$, $\pi_\lambda$ dieses System.

Weiterhin gilt
\begin{align*}
  \sup_{t > 0} \Big( \int_{\R^3} |u_\lambda(x,t)|^3 \d x \Big)^{\frac{1}{3}} 
  &= \sup_{t > 0} \Big( \int_{\R^3} |\lambda u (\lambda x, \lambda^2 t) |^3 \d x \Big)^{\frac{1}{3}} 
  = \sup_{t > 0} \Big( \int_{\R^3} |u(y,t)|^3 \d y \Big)^{\frac{1}{3}},
  \end{align*}
also ist die Norm von $\Ell^\infty(0,\infty; \Ell_\sigma^3(\R^3))$ invariant unter dem natürlichen Skalierungsverhalten der Navier-Stokes-Gleichungen.
Wir wollen den Raum $\Ell^\infty(0,\infty; \Ell^3_\sigma(\R^3))$ als \emph{kritischen Raum} bezeichnen.

Angenommen, man könnte zeigen, dass für den Lifespan $T_0$ von $u$ eine Abschätzung
\begin{align*}
  \tag{\$} T_0 \geq C(\|a\|_{\Ell^3})
\end{align*}
existiert, wobei $C(\|a\|_{\Ell^3}) > 0$ im Wesentlichen von $\|a\|_{\Ell^3}$ abhängt.
Definiere nun für $\lambda > 0$
$$
a_\lambda(x) \coloneqq \lambda a(\lambda x),
$$
so folgt $\|a_\lambda\|_{\Ell^3} = \|a\|_{\Ell^3}$ und für jedes $\lambda > 0$ existiert $u_\lambda$ mindestens auf dem Zeitintervall $[0,T_0]$ ($T_0$ ist $\lambda$-unabhängig).
Hieraus ergibt sich, dass $u = (u_\lambda)_{\frac{1}{\lambda}}$ mindestens auf dem Zeitintervall $[0,\lambda^2 T_0]$ existiert und es gilt
\begin{align*}
  \tag{$\ast$} \|u\|_{\Ell^\infty(0,\lambda^2 T_0; \Ell^3_\sigma(\R^3))} = \|u_\lambda\|_{\Ell^\infty(0,T_0; \Ell^3_\sigma(\R^3))}.
\end{align*} 
Hätte man nun noch auf dem Lifespan-Intervall eine geeignete Abschätzung der Lösung gegen die Daten, d.h.
\begin{align*}
  \tag{\$\$} \|u\|_{\Ell^\infty(0,T_0; \Ell^3_\sigma(\R^3))} \leq C(\|a\|_{\Ell^3}),
\end{align*}
so folgt mit ($\ast$)
$$
\|u\|_{\Ell^\infty(0,\lambda^2 T_0; \Ell^3_\sigma(\R^3))}
\overset{\text{($\ast$)}}{=\vphantom{\leq}\!\!} \|u_\lambda\|_{\Ell^\infty(0,T_0; \Ell^3_\sigma(\R^3))}
\overset{\text{(\$\$)}}{\leq} C(\|a_\lambda\|_{\Ell^3}) = C(\|a\|_{\Ell^3})
$$
und für $\lambda \to \infty$ ergäbe sich $u \in \Ell^\infty(0,\infty; \Ell^3_\sigma(\R^3))$, wir hätten damit also aus einer lokalen Lösung eine globale Lösung gemacht.

Wie bekommt man jetzt die Million?
Falls die Anfangsdaten $a \in \Ell^2_\sigma(\R^3) \cap \Ell^3_\sigma(\R^3)$ und $v$ ``schwache Lösung'' von (NST) die ``Energieungleichung'' erfüllt, so haben Kozono und Sohr 1996 gezeigt dass dann $u = v$ gilt.

Wenn $a$ zusätzlich eine Schwartz-Funktion ist und die u ``schwache Lösung'' in $\Ell^\infty(0,\infty,\Ell^3_\sigma(\R^3))$ liegt, so haben Escramiaza, Seregin Sverak 2003 gezeigt, dass in diesem Falle $u$ glatt ist in $[0,\infty) \times \R^3$. Somit gilt:
$$
\text{Zeige (\$) und (\$\$)} \implies \text{Millionär}.
$$

Wir beschäftigen uns im Folgenden mit der Lösbarkeit von (NST) in $\Ell^\infty(0,T; \Ell^3_\sigma(\Omega))$, wobei $\Omega \subset \R^3$ ein beschränktes Lipschitz-Gebiet ist.
Genauer geht es um lokale Lösbarkeit in der Zeit für beliebig große Anfangsdaten und globale Lösbarkeit für kleine Anfangsdaten. Eine formale Anwendung der Helmholtz-Projektion auf (NST) liefert
$$
\begin{cases}
  \partial_t u + Au & = - \PP(u\cdot \nabla u) \\
  u(0) &= a.
\end{cases}
$$
Sieht man die Nichtlinearität als rechte Seite, so müsste $u$ durch die Variation-der-Konstanten-Formel
$$
u(t) = \e^{-tA} a - \int_0^t \e^{-(t - s)A_p}\PP(u(s) \cdot \nabla) u(s) \d s
$$
gegeben sein. 
Dies motiviert die folgende Definition:
\begin{defn}
  Seien $0 < T \leq \infty$, $r \geq 3$ und $a \in \Ell_\sigma^r(\Omega)$.
  Dann heißt $u \colon [0,T) \to \Ell_\sigma^r(\Omega)$ \emph{milde Lösung} von (NST) mit  Anfangswert $a$, falls $u \in \CC(0,T; \Ell_\sigma^r(\Omega))$ und für alle $0 < t< T$ und ein $p \geq r$
  $$
  \big( s \mapsto \e^{-(t - s)A} \PP (u(s) \cdot \nabla) u(s) \big) \in \Ell^1(0,t; \Ell_\sigma^p(\Omega))
  $$
  und
  $$
  u(t) = \e^{-t A} a - \int_0^t \e^{-(t- s) A} \PP(u(s) \cdot \nabla )u(s) \d s.
  $$
\end{defn}

\begin{hsatz}
  \label{hsatz:momomoney}
  Sei $\Omega \subset \R^d$ ein beschränktes Lipschitz-Gebiet.
  Dann existiert ein $\varepsilon > 0$, sodass für alle $3 \leq r < 3 + \varepsilon$ und alle $a \in \Ell_\sigma^r(\Omega)$ die folgenden Aussagen gelten.
  \begin{enumerate}[i)]
    \item Es existiert $T_0 > 0$ und eine milde Lösung $u \colon [0,T_0) \to \Ell_\sigma^r(\Omega)$ von (NST) mit Anfangswert $a$, sodass für alle $r \leq p < 3 + \varepsilon$ mit $\frac{3}{2} \big( \frac{1}{r} - \frac{1}{p} \big) < \frac{1}{4}$ gilt
      \begin{align*}
        &\big( t \mapsto t^{\frac{3}{2}(\frac{1}{r} - \frac{1}{p})} u(t) \big) \in \mathrm{BC}([0,T_0), \Ell^p_\sigma(\Omega)) \\
        &\big( t \mapsto t^{\frac{1}{2} + \frac{3}{2}(\frac{1}{r} - \frac{1}{p})} \nabla u(t) \big) \in \mathrm{BC}([0,T_0), \Ell^p(\Omega; \C^9) )
      \end{align*}
      Weiterhin gilt
      $$
      \|u(t) - a \|_{\Ell^r} \to 0 \quad\text{für } t \searrow.
      $$
      Falls $r < p < 3 + \varepsilon$ gilt, dass 
      $$
      t^{\frac{3}{2}(\frac{1}{r} - \frac{1}{p})} \|u(t)\|_{\Ell^p} \to 0 \quad\text{für } t \searrow 0
      $$
      und, falls $r \leq p < 3 + \varepsilon$ gilt, so folgt
      $$
      t^{\frac{1}{2} + \frac{3}{2}\big( \frac{1}{r} - \frac{1}{p} \big)} \|\nabla u (t) \|_{\Ell^p} \to 0 \quad\text{für } t \searrow 0.
      $$

    \item Falls $r > 3$, so existiert $C > 0$, sodass $T_0 \geq C \cdot \|a\|_{\Ell^r}^{-\frac{2r}{r - 3}}$.
    \item Für alle $3 \leq p < 3 + \varepsilon$ existieren $C_1, C_2 > 0$, sodass unter der Voraussetzung, dass $\|a\|_{\Ell^3} \leq C_1$, die milde Lösung global ist, d.h. $T_0 = \infty$.
      Außerdem gelten die Abschätzungen für das Langzeitverhalten
      \begin{align*}
        \|u(t)\|_{\Ell^p} &\leq C_2 \cdot t^{\frac{3}{2p} - \frac{1}{2}}, \quad\text{für alle } 0 < t < \infty \\
        \|\nabla u(t) \|_{\Ell^p} &\leq C_2 \cdot t^{\frac{3}{2p} - 1}, \quad\text{für alle } 0 < t < \infty.
      \end{align*}
  \end{enumerate}
\end{hsatz}
Für den Beweis von Hauptsatz \ref{hsatz:momomoney} definieren wir iterativ
\begin{align*}
  u_0(t) &\coloneqq \e^{-t A} a \\
  u_{j + 1}(t) &\coloneqq u_0(t) - \int_0^t \e^{-(t - s) A} \PP(u_j(s) \cdot \nabla) u_j(s) \d s \quad\text{für alle } j \in \N_0.
\end{align*}
Für $p \geq r$ definiere weiterhin $\sigma \coloneqq \frac{3}{2} \big( \frac{1}{r} - \frac{1}{p} \big)$ und für $T > 0$ definieren wir die Größen
\begin{align*}
  K_j &\coloneqq K_j(T) \coloneqq \sup_{0 < t < T} t^\sigma \|u_j(t) \|_{\Ell^p} \\
  R_j &\coloneqq R_j(T) \coloneqq \sup_{0 < t < T} t^{\frac{1}{2} + \sigma} \|\nabla u_j(t) \|_{\Ell^p}.
\end{align*}
Der Parameter $\sigma$ stammt aus Satz \ref{thm:lplqSmoothing}.

\begin{proof}[Beweis von Hauptsatz \ref{hsatz:momomoney}]
  Wir werden uns hier nicht um die Stetigkeit oder Messbarkeit der Integranden kümmern. Dies kann aber per Induktion als Übungsaufgabe bewiesen werden.

  \textbf{Schritt 1}: Wir zeigen zunächst die  Beschränktheit der Folgen $K_j$ und $R_j$ für $3 \leq r < p < 3 + \varepsilon$ und $3 < r \leq p < 3 + \varepsilon$.
  Mit Satz \ref{thm:lplqSmoothing} folgt zunächst
  \begin{align*}
    \|u_0(t) \|_{\Ell^p} &= \|\e^{-t A} a \|_{\Ell^p} \leq C \cdot t^{-\sigma} \|a\|_{\Ell^r} \\
    \|\nabla u_0(t) \|_{\Ell^p} &= \|\nabla \e^{-\frac{t}{2} A} \e^{-\frac{t}{2} A} a \|_{\Ell^p} \leq C\cdot t^{-\frac{1}{2}} \|\e^{-\frac{t}{2} A} a \|_{\Ell^p} \leq C\cdot t^{-\frac{1}{2}   - \sigma} \|a\|_{\Ell^r}.
  \end{align*}
  Dies zeigt $K_0, R_0 < \infty$.

  Sei $0 < t < T$. 
  Nehme induktiv an, dass $K_j, R_j < \infty$.
  Dann folgt
  \begin{align*}
    \|u_{j + 1}(t) \|_{\Ell^p}
    &\leq \|u_0(t)\|_{\Ell^p} + C \, \int_0^t \|\e^{-(t - s) A} \PP(u_j(s) \cdot \nabla) u_j(s) \|_{\Ell^p} \d s \\
    &\leq \|u_0(t)\|_{\Ell^p} + C \, \int_0^t (t - s)^{-\frac{3}{2} (\frac{2}{p} - \frac{1}{p})} \|(u_j(s) \cdot \nabla) u_j(s) \|_{\Ell^{\frac{p}{2}}} \d s \\
    &\leq \|u_0(t)\|_{\Ell^p} + C \, \int_0^t (t - s)^{\frac{3}{2p}} s^{-2\sigma - \frac{1}{2}} \d s \, K_j \cdot \R_j,
  \end{align*}
  wobei wir zunächst Satz \ref{thm:lplqSmoothing} und Hauptsatz \ref{hsatz:contProj} anwenden und danach höldern sowie die Induktionsvoraussetzung nutzen.
  Das obige Integral existiert wegen
  $$
  \frac{3}{2p} < \frac{1}{2} \quad\text{und}\quad 2\sigma + \frac{1}{2} < 1,
  $$
  da nach Voraussetzung $\sigma < \frac{1}{4}$ gilt.
  Daraus folgt mit der Substitution $s = xt$
  $$
  \int_0^t (t - s)^{-\frac{3}{2p}} s^{-2\sigma - \frac{1}{2}} \d s = t^{1 - \frac{3}{2p} - 2\sigma - \frac{1}{2}} \int_0^1 ( 1 - x )^{-\frac{3}{2p}}  x^{- 2\sigma - \frac{1}{2}} \d x.
  $$
  Und da für den Exponenten von $t$
  $$
  \frac{1}{2} - \frac{3}{2p} - \sigma = \frac{1}{2} - \frac{3}{2r}
  $$
  gilt, folgt
  $$
  K_{j + 1} \leq K_0 + C\; T^{\frac{1}{2} - \frac{3}{2r}} K_j R_j.
  $$
  Für den Gradienten gilt mit analoger Argumentation
  \begin{align*}
    \|\nabla u_{j + 1}(t) \|_{\Ell^p}
    &\leq \|\nabla u_0 (t) \|_{\Ell^p} + \int_0^t \|\nabla \e^{-(t - s) A} \PP(u_j(s) \cdot \nabla) u_j(s) \|_{\Ell^p} \d s \\
    &\leq \|\nabla u_0(t) \|_{\Ell^p} + C \int_0^t (t - s)^{-\frac{1}{2}} \; \|\e^{-\frac{1}{2}(t - s)A} \PP(u_j(s) \cdot \nabla) u_j(s) \|_{\Ell^p} \d s \\
    &\leq \|\nabla u_0(t) \|_{\Ell^p} + C \int_0^t (t - s)^{-\frac{1}{2} - \frac{3}{2p}} \|(u_j(s) \cdot \nabla) u_j(s) \|_{\Ell^{\frac{p}{2}}} \d s \\ 
    &\leq \|\nabla u_0(t) \|_{\Ell^p} + C \int_0^t (t - s)^{-\frac{1}{2} - \frac{3}{2p}} s^{-2\sigma - \frac{1}{2}} \d s \, K_j \cdot R_j.
  \end{align*}
  Obiges Integral existiert, da
  $$
  \frac{1}{2} + \frac{3}{2p} < 1 \quad\text{und}\quad 2\sigma + \frac{1}{2} < 1.
  $$
  Daraus ergibt sich
  $$
  \int_0^t (t - s)^{-\frac{1}{2} - \frac{3}{2p}} s^{-2\sigma - \frac{1}{2}} \d s 
  = t^{1 - \frac{1}{2} - \frac{3}{2p} - 2\sigma - \frac{1}{2}} \int_0^1 (1 - x)^{-\frac{1}{2} - \frac{3}{2p}} x^{-2\sigma - \frac{1}{2}} \d x.
  $$
  Wir erhalten
  $$
  R_{j + 1} \leq R_0 + C\, T^{\frac{1}{2} - \frac{3}{2r}} K_j R_j
  $$
  und damit gilt $K_{j + 1}, R_{j + 1} < \infty$.
  Sei $B$ das Maximum der Konstanten $C$ der Ungleichungen für $K_{j + 1}$ und $R_{j + 1}$.
  Dann gilt
  \begin{align*}
    K_{j + 1} &\leq K_0 + 2 B\, T^{\frac{1}{2} - \frac{3}{2r}} K_j R_j \\
    R_{j + 1} &\leq R_0 + 2 B\, T^{\frac{1}{2} - \frac{3}{2r}} K_j R_j.
  \end{align*}
  Angenommen es gelte
  \begin{align}
    T^{\frac{1}{2} - \frac{3}{2r}} \max\{ K_0, R_0 \} \leq \frac{1}{8B},
  \end{align}
  dann folgt zunächst mit $M \coloneqq 2 \max \{ K_0, R_0 \}$ 
  \begin{align*}
    K_1 \leq K_0 + 2 B T^{\frac{1}{2} - \frac{3}{2r}} K_0 R_0 \leq K_0 + 2 B \frac{1}{8B} \frac{M}{2} \leq M
  \end{align*}
  und hieraus induktiv
  \begin{align}
    K_{j + 1} < \frac{M}{2} + 2 B\, T^{\frac{1}{2} - \frac{3}{2r}} M^2
    &= \frac{M}{2} + 4 B\, T^{\frac{1}{2} - \frac{3}{2r}} \, M \max\{K_0, R_0\}
    \leq M.
  \end{align}
  Analog zeigt man 
  \begin{align}
    R_{j + 1} \leq M.
  \end{align}
  Mit der Definition von $M$ und (1) folgt
  \begin{align}
    2 B T^{\frac{1}{2} - \frac{3}{2r}} M \leq \frac{1}{2}.
  \end{align}
  Weiterhin gilt
  \begin{align}
    M \to 0, \quad\text{falls } K_0, R_0 \to 0 \quad\text{für } T \to 0.
  \end{align}
  Somit gelten (2)-(5), falls (1) gilt.

  Wir zeigen nun, dass (1) unter den Voraussetzungen des Hauptsatzes erfüllt ist.
  Zuerst gilt, falls $\sigma > 0$,
  \begin{align}
    t^\sigma \|\e^{-t A} a \|_{\Ell^p} \to 0 \quad\text{für } t \searrow 0.
  \end{align}
  Dies beweist man wie folgt:
  Sei $(a_n)_{n \in \N} \subset \CC_{\cc,\sigma}^\infty(\Omega)$ mit $a_n \to a$ in $\Ell^r_\sigma(\Omega)$.
  Dann folgt mit Satz \ref{thm:lplqSmoothing}
  \begin{align*}
    t^\sigma \|\e^{-t A} a \|_{\Ell^p} 
    &\leq t^\sigma \|\e^{-tA} (a - a_n) \|_{\Ell^p} + t^\sigma \|\e^{-tA} a_n \|_{\Ell^p} \\
    &\leq C\, \|a - a_n\|_{\Ell^r} + C t^\sigma \|a_n\|_{\Ell^p}.
  \end{align*}
  Sei $\varepsilon > 0$.
  Wähle $n \in \N$ derart, dass $C\, \|a - a_n\|_{\Ell^r} < \frac{\varepsilon}{2}$ und wähle
  $$
  \delta \coloneqq \Big( \frac{\varepsilon }{2 C \, \|a_n\|_{\Ell^p}} \Big)^{\frac{1}{\sigma}}.
  $$
  Dann gilt für alle $0 < t < \delta$
  $$
  t^\sigma \|\e^{-t A} a \|_{\Ell^p} < \varepsilon.
  $$
  Aus (6) folgt für $\sigma > 0$
  \begin{align}
    K_0 \to 0 \quad\text{für } T \to 0.
  \end{align}
  Weiterhin gilt für $\sigma \geq 0$ (dies gilt auch für $r = 3$ und $p = 3$)
  \begin{align}
    t^{\frac{1}{2} + \sigma} \|\nabla \e^{-t A} a \|_{\Ell^p} \to 0 \quad\text{für } t \searrow 0,
  \end{align}
  denn ähnlich wie oben zeigt man unter Verwendung von \ref{thm:lplqSmoothing}
  \begin{align*}
    t^{\frac{1}{2} + \sigma} \|\nabla \e^{-t A} a \|_{\Ell^p}
    &\leq t^{\frac{1}{2} + \sigma} \|\nabla \e^{-tA} (a - a_n)\|_{\Ell^p} + t^{\frac{1}{2} + \sigma} \|\nabla \e^{-tA} a_n \|_{\Ell^p} \\
    &\leq C\, \|a - a_n\|_{\Ell^r} + C t^{\frac{1}{2} + \sigma} \|A^{\frac{1}{2}} \e^{-t A} a_n \|_{\Ell^p} \\
    &\leq C \, \|a - a_n\|_{\Ell^r} + C t^{\frac{1}{2} + \sigma} \|A^{\frac{1}{2}} a_n \|_{\Ell^p}
  \end{align*}
  Aus (8) folgt insbesondere für $\sigma \geq 0$ und auch für $3 = r = p$
  \begin{align}
    R_0 \to 0 \quad\text{für } T \to 0.
  \end{align}

  Nun sind wir in der Lage (1) zu verifizieren.
  Wir definieren für $r > 3$
  $$
  T_0 \coloneqq \Big(\frac{1}{8 B\, C} \Big)^{\frac{2r}{r - 3}} \cdot \Big( \frac{1}{\|a\|_{\Ell^r}} \Big)^{\frac{2r}{r - 3}}.
  $$
  Dann gilt für alle $T \leq T_0$ mit \ref{thm:lplqsmoothing}
  $$
  T^{\frac{1}{2} - \frac{3}{2r}} \max \{K_0, R_0\} \leq C\, T_0^{\frac{1}{2} - \frac{3}{2r}} \|a\|_{\Ell^r} \leq \frac{1}{8B}.
  $$
  Wir zeigen, dass für jedes dieser $T$ eine milde Lösung auf $[0,T)$ existiert, d.h. sie existiert mindestens bis $T_0$. ($\implies $ Abschätzung an den Lifespan aus Behauptung ii))
  Falls $r = 3$, so gibt es keine explizite $T$-Abhängigkeit in (1).
  Ist jedoch $r > 0$ so folgt aus (7) und (9) $\max\{K_0, R_0\} \leq \frac{1}{8B}$ für $T$ klein genug und damit gilt auch (1) für $T$ klein genug.
  Weiterhin gilt im Falle $r = 3$ für alle $T > 0$
  $$
  \max\{K_0, R_0 \} \leq C\, \|a\|_{\Ell^3} \leq \frac{1}{8B},
  $$
  falls $\|a\|_{\Ell^3}$ klein genug ist.

  \textbf{Schritt 2}: Falls $3 \leq r < p < 3 + \varepsilon$ und $3 < r \leq p < 3 + \varepsilon$ gelten
  \begin{align*}
    \big( t \mapsto t^{\frac{3}{2}(\frac{1}{r} - \frac{1}{p})} u_{j + 1}(t) \big) &\in \mathrm{BC}([0,T_0), \Ell^p_\sigma(\Omega)) \\
      \big( t \mapsto t^{\frac{1}{2} + \frac{3}{2}(\frac{1}{r} - \frac{1}{p})} \nabla u_{j + 1}(t) \big) &\in \mathrm{BC}([0,T_0), \Ell^p(\Omega; \C^9)) 
  \end{align*}
  und die Konvergenzen für $t^\sigma \|u_{j + 1} (t) \|_{\Ell^p}$ sowie $t^{\frac{1}{2} + \sigma} \|\nabla u_{j + 1}(t) \|_{\Ell^p}$ für $t \searrow 0$.
  
  Aus (2) und (3) folgt bereits
  \begin{align*}
    &\big( t \mapsto t^{\frac{3}{2}(\frac{1}{r} - \frac{1}{p})} u(t) \big) \in \Ell^\infty([0,T_0), \Ell^p_\sigma(\Omega)) \\
    &\big( t \mapsto t^{\frac{1}{2} + \frac{3}{2}(\frac{1}{r} - \frac{1}{p})} \nabla u(t) \big) \in \Ell^\infty([0,T_0), \Ell^p(\Omega; \C^9) ).
  \end{align*}
  Wir zeigen die Stetigkeit von $u_{j + 1}$ und $\nabla u_{j + 1}$.
  Hier konzentrieren wir uns auf die rechtsseitige Stetigkeit.
  Seien $0 < t < T$ und $h > 0$ mit $t + h < T$.
  Dann gilt mit Hauptsatz \ref{hsatz:tolksdorf} für
  $$
  \max\{r, \frac{p}{2} \} \leq q \leq p
  $$
  folgendes Konvergenzverhalten
  \begin{align*}
    \|\e^{-(t + h)A} a - \e^{-t A} a \|_{\Ell^q} \to 0 &\quad\text{für } h \to 0, \\
    \|\nabla\e^{-(t + h)A} a - \nabla \e^{-tA} a \|_{\Ell^q} \leq C\, \| \big[ \e^{-hA} - I \big] A^{\frac{1}{2}} \e^{-tA} a \|_{\Ell^q} \to 0 &\quad\text{für } h \to 0,
  \end{align*}
  wobei wir verwendet haben, dass die Halbgruppe mit gebrochenen Potenzen von $A$ kommutiert.
  Weiterhin gilt
  \begin{align*}
    &\Big\| \int_0^{t + h} \e^{-(t + h - s) A} \PP(u_j(s) \cdot \nabla) u_j(s) \d s - \int_0^t \e^{-(t - s)A} \PP(u_j \cdot \nabla) u_j(s) \d s \Big\|_{\Ell^q} \\
    &\qquad\leq \int_t^{t + h} \|\e^{-(t + h - s)A} \PP(u_j(s) \cdot \nabla) u_j(s) \|_{\Ell^q} \d s + \int_0^t \| \big[ \e^{-hA} - I \big] \e^{-(t - s)} \PP(u_j(s) \cdot \nabla) u_j(s) \|_{\Ell^q} \d s.
  \end{align*}
  Wir konstruieren nun beispielhaft am zweiten Integral eine Majorante.
  Dazu verwenden wir zunächst Hauptsatz \ref{hsatz:shen} sowie Hauptsatz \ref{hsatz:contProj} und Satz \ref{thm:lplqSmoothing} mit dem Ergebnis
  \begin{align*}
    \|[ \e^{-hA} - I] \e^{-(t - s)A} \PP(u_j(s) \cdot u_j(s) \|_{\Ell^q} 
    &\leq C \, (t - s)^{-\frac{3}{2} \big( \frac{2}{p} - \frac{1}{q} \big)} \|(u_j(s) \cdot \nabla) u_j(s) \|_{\Ell^{\frac{p}{2}}} \\
    &\leq C\, (t - s)^{-\frac{3}{2} \big( \frac{2}{p} - \frac{1}{q} \big)} s^{-2\sigma - \frac{1}{2}} M^2,
  \end{align*}
  wobei wir im letzten Schritt gehöldert und (2) und (3) angewendet haben.
  Der letzte Ausdruck ist integrierbar, da
  $$
  \frac{3}{2} \big( \frac{2}{p} - \frac{1}{q} \big)
  \leq \frac{3}{2} \big( \frac{2}{p} - \frac{1}{p} \big)
  = \frac{3}{2p} 
  < \frac{1}{2}
  \quad\text{sowie}\quad 2 \sigma + \frac{1}{2} < 1.
  $$
  Majorisierte Konvergenz liefert Stetigkeit von $u_{j + 1} \colon (0,T) \to \Ell_\sigma^q(\Omega)$.

  Für $\nabla u_{j + 1}$ folgt ähnlich
  \begin{align*}
    \Big\| \int_0^{t + 1} \nabla \e^{-(t + h - s) A} \PP (u_j(s) \cdot \nabla ) u_j(s) \d s \Big\|_{\Ell^q}
    &\leq \int_t^{t + h} \|\nabla \e^{(-t + h - s) A} \PP(u_j(s) \cdot \nabla ) u_j(s) \|_{\Ell^q} \d s \\ 
    &\hphantom{\leq} + \int_0^t \|\nabla [\e^{-h A} - I ] \e^{-(t - s) A} \PP(u_j(s) \cdot \nabla) u_j(s) \|_{\Ell^q} \d s.
  \end{align*}
  Wir konstruieren nun beispielhaft eine Majorante für das erste Integral, nämlich
  \begin{align*}
    \|\nabla \e^{-(t + h - s) A} \PP(u_j(s) \cdot \nabla) u_j(s) \|_{\Ell^q} 
    &\leq C\, (t + h - s)^{-\frac{1}{2} - \frac{3}{2} \big( \frac{2}{p} - \frac{1}{q} \big)} s^{-2 \sigma - \frac{1}{2} } M^2,
  \end{align*}
  wobei die Abschätzung im Wesentlichen dieselben Resultate verwendet wie schon im Beweis für $u_{j + 1}$.
  Somit erhalten wir weiter
  \begin{align*}
    &\int_t^{t + h} \|\nabla \e^{-(t + h - s)A} \PP(u_j(s) \cdot \nabla) u_j(s) \|_{\Ell^q} \d s \\
    &\qquad\leq C\, M^2 \int_t^{t + h} (t + h - s)^{-\frac{1}{2} - \frac{3}{2} \big( \frac{2}{p} - \frac{1}{q} \big)} s^{-2\sigma - \frac{1}{2}} \d s \\
    &\qquad= C M^2 (t + h)^{1 - \frac{1}{2} - \frac{3}{2}\big( \frac{2}{p} - \frac{1}{q} \big) - 2\sigma - \frac{1}{2}} \int_{\frac{t}{t + h}}^1 (1 - x)^{-\frac{1}{2} - \frac{3}{2} \big(\frac{2}{p} - \frac{1}{q}\big)} x^{-2\sigma - \frac{1}{2}} \d x \to 0 \quad\text{für } h \searrow 0.
  \end{align*}
  Damit ist auch $\nabla u_{j + 1} \colon (0, T) \to \Ell^q(\Omega; \C^9)$ (rechtsseitig) stetig.

  Falls $\sigma > 0$, folgt aus (2), (3), (5), (7), (9) Stetigkeit in $0$, d.h.
  \begin{align*}
    \|t^\sigma u_{j + 1}(t) \|_{\Ell^p} \to 0 &\quad\text{für } t \to 0, \\
    \|t^{\frac{1}{2} + \sigma} \nabla u_{j + 1}(t) \|_{\Ell^p} \to 0 &\quad\text{für } t\to 0.
  \end{align*}
  Hiermit erhalten wir schließlich
  \begin{align*}
    \big( t \mapsto t^\sigma u_{j + 1}(t) \big) &\in \mathrm{BC}([0,T), \Ell_\sigma^p(\Omega)) \\
    \big( t \mapsto t^{\frac{1}{2} + \sigma} \nabla u_{j + 1}(t) \big) &\in \mathrm{BC}([0,T); \Ell^p(\Omega,\C^9))
  \end{align*}
  Falls $r > 3$ und $\sigma = 0$, so folgt mit Satz \ref{thm:lplqSmoothing} und Hauptsatz \ref{hsatz:contProj}
  \begin{align*}
    \|u_{j + 1} - a \|_{\Ell^r}
    &\leq \|u_0(t) - a\|_{\Ell^r} + \int_0^t \|\e^{-(t - s) A} \PP(u_j(s) \cdot \nabla) u_j(s) \|_{\Ell^r} \d s \\
    &\leq \|u_0(t) - a \| + C \int_0^t (t - s)^{-\frac{3}{2r}} s^{-\frac{1}{2}} \d s \cdot M \\
    &\leq \|u_0(t) - a \| + C t^{\frac{1}{2} - \frac{3}{2r}} \int_0^1 (1 - x)^{-\frac{3}{2r}} x^{-\frac{1}{2}} \d x \cdot M \to 0 \quad\text{für } t \to 0.
  \end{align*}
  Für $t^\frac{1}{2} \cdot \nabla u_{j + 1}$ folgt unter Zuhilfenahme von (8) analog
  $$
  t^{\frac{1}{2}} \|\nabla u_{j + 1}(t) \|_{\Ell^r} \to 0 \quad\text{für } t \to 0.
  $$

  \textbf{Schritt 3}: $(t \mapsto t^\sigma u_j(t))_{j \in \N}$ ist eine Cauchyfolge in $\mathrm{BC}([0,T); \Ell^p_\sigma(\Omega))$ und $(t \mapsto t^{\frac{1}{2} + \sigma} \nabla u_j(t))_{j \in \N}$ ist Cauchy in $\mathrm{BC}([0,T); \Ell^p(\Omega; \C^9))$, falls $3 \leq r < p < 3 + \varepsilon$ oder $3 < r \leq p < 3 + \varepsilon$.

  Es gilt wie bei der Abschätzung aus Schritt 1:
  \begin{align*}
    \|u_{j + 1}(t) - u_j(t)\|_{\Ell^p}
    &\leq \int_0^t \|\e^{-(t - s) A} \PP[ u_{j - 1}(s) \cdot \nabla (u_{j - 1}(s) - u_j(s))] \|_{\Ell^p} \d s \\
    &\hphantom{\leq} + \int_0^t \|\e^{-(t - s)A} \PP[ (u_{j - 1}(s) - u_j(s)) \cdot \nabla u_j(s) ] \|_{\Ell^p} \d s \\
    &\leq 2B \; T^{\frac{1}{2} - \frac{3}{2r}} \;M t^{-\sigma} \max\Big\{ \sup_{0 < s < T} s^\sigma \|u_j(s) - u_{j - 1}(s) \|_{\Ell^p},\\
    &\hphantom{2B M\; T^{\frac{1}{2} - \frac{3}{2r}} t^{-\sigma} \max\Big\{ }
    \sup_{0 < s < T} s^{\frac{1}{2} + \sigma} \|\nabla(u_j(s) - u_{j - 1}(s)) \|_{\Ell^p} \Big\}.
  \end{align*}
  Eine analoge Abschätzung ergibt sich für $\|\nabla(u_{j + 1}(t) - u_j(t)) \|_{\Ell^p}$.
  Damit erhalten wir unter Ausnutzung der Ungleichung (4)
  \begin{align*}
    &\max\Big\{\sup_{0 < s < T} s^\sigma \| u_{j + 1}(s) - u_j(s) \|_{\Ell^p}, \sup_{0 < s < T} s^{\frac{1}{2} + \sigma} \|\nabla(u_{j + 1}(s) - u_j(s)) \|_{\Ell^p} \Big\} \\
    &\qquad\leq
    \frac{1}{2}
    \max\Big\{\sup_{0 < s < T} s^\sigma \| u_{j}(s) - u_{j - 1}(s) \|_{\Ell^p}, \sup_{0 < s < T} s^{\frac{1}{2} + \sigma} \|\nabla(u_{j}(s) - u_{j - 1}(s)) \|_{\Ell^p} \Big\}.
  \end{align*}
  Nun gilt 
  \begin{align*}
    t^\sigma u_{j + 1} &= \sum_{n = 1}^{j + 1} t^\sigma (u_n - u_{n - 1}) + t^\sigma u_0 \\
    t^{\frac{1}{2} + \sigma} \nabla u_{j + 1} &= \sum_{n = 1}^{j + 1} t^{\frac{1}{2} + \sigma} \nabla (u_n - u_{n - 1}) + t^{\frac{1}{2} + \sigma} \nabla u_0.
  \end{align*}
  Zusammen mit der vorigen Abschätzung zeigt sich, dass die Teleskopsummen jeweils durch eine geometrische Reihe dominiert werden.
  Hiermit ist die Behauptung von Schritt 3 bewiesen.
  Insbesondere konvergiert
  \begin{align*}
    t^\sigma u_j \to t^\sigma u &\quad\text{in } \mathrm{BC}([0,T); \Ell^p_\sigma) \\
      t^{\frac{1}{2} + \sigma} \nabla u_j \to t^{\frac{1}{2} + \sigma} \nabla u &\quad\text{in } \mathrm{BC}([0,T); \Ell^p(\Omega; \C^9))
  \end{align*}
  und $u$ ist eine milde Lösung.

  \textbf{Schritt 4}:
  Der Fall $r = 3$.
  Wähle $p > 3$ mit $\frac{p}{2} \leq 3$.
  Dann folgt
  \begin{align*}
    \|u_{j + 1}(t) \|_{\Ell^3}
    &\leq \|u_0(t)\|_{\Ell^3} + \int_0^t \|\e^{-(t - s) A} \PP (u_j(s) \cdot \nabla) u_j(s) \|_{\Ell^3} \d s \\
    &\leq \|u_0(t)\|_{\Ell^3} + C \int_0^t (t - s)^{-\frac{3}{2} \big( \frac{2}{p} - \frac{1}{3} \big)} \|(u_j(s) \cdot \nabla) u_j(s) \|_{\Ell^{\frac{p}{2}}} \d s \\
    &\leq \|u_0(t)\|_{\Ell^3} + C M^2 \int_0^t (t - s)^{-\frac{3}{2} \big( \frac{2}{p} - \frac{1}{3}\big)} s^{-2\sigma - \frac{1}{2}} \d s \\
    &\leq \|u_0(t)\|_{\Ell^3} + C \, M^2 t^{1 - \frac{3}{2} \big( \frac{2}{p} - \frac{1}{3} \big) - 2\sigma - \frac{1}{2}} \int_0^1 (1 - x)^{-\frac{3}{2} \big( \frac{2}{p} - \frac{1}{3} \big)} x^{-2\sigma - \frac{1}{2}} \d x.
  \end{align*}
  Es gilt
  $$
  1 - \frac{3}{2} \big( \frac{2}{p} - \frac{1}{3} \big) - 2\sigma - \frac{1}{2}
  \;=\; \frac{1}{2} - \frac{3}{p} + \frac{1}{2} - 3 \big( \frac{1}{3} - \frac{1}{p} \big)
  \;=\; 0
  $$
  und hieraus folgt $u_{j + 1} \in \mathrm{BC}([0,T); \Ell^3_\sigma(\Omega))$.
  Analog zeigt man
  $$
  \|u_{j + 1}(t) - a\|_{\Ell^3}
  \leq \|u_0(t) - a\|_{\Ell^3} + C\, M^2 \to 0 \quad \text{für } t \text{ (und $T$)} \to 0
  $$
  nach (5), (7), (9).

  Analog folgt $(t \mapsto t^{\frac{1}{2}} \nabla u_j(t))_{j \in \N} \in \mathrm{BC}([0,T); \Ell^3(\Omega; \C^9))$ mit $t^{\frac{1}{2}} \|\nabla u_{j + 1}(t) \|_{\Ell^3} \to 0$ für $t \searrow 0$. (Dazu wird (9) mit $r = 3$ und $\sigma = 0$ benötigt).
  Weiterhin gilt
  \begin{align*}
    \|u_i(t) - u_j(t)\|_{\Ell^3}
    &\leq C \int_0^t (t - s)^{-\frac{3}{2} \big( \frac{2}{p} - \frac{1}{3} \big)} \|u_{i - 1}(s) \cdot \nabla (u_{i - 1}(s) - u_{j - 1}(s)) \|_{\Ell^{\frac{p}{2}}} \d s \\
    &\hphantom{\leq} + C \int_0^t (t - s)^{-\frac{3}{2} \big( \frac{2}{p} - \frac{1}{3} \big)} \|((u_{i - 1}(s) - u_{j - 1}(s)) \cdot \nabla) u_{j - 1}(s) \|_{\Ell^{\frac{p}{2}}} \d s \\
    &\leq 2 C M \int_0^t (t - s)^{-\frac{3}{2} \big(\frac{2}{p} - \frac{1}{3} \big)} s^{-2\sigma - \frac{1}{2}} \d s \\
    &\hphantom{\leq 2 C M }\cdot \max \Big\{ \sup_{0 < s < T} s^\sigma \|u_{i - 1}(s) - u_{j - 1}(s) \|_{\Ell^p},\\
    &\hphantom{\leq 2 C M\cdot \max \Big\{ } \sup_{0 < s < T} s^{\frac{1}{2} + \sigma} \|\nabla(u_{i - 1}(s) - u_{j - 1}(s)) \|_{\Ell^p} \Big\} \to 0 \quad\text{für } i,j \to \infty.
  \end{align*}
  Folglich ist $(t \mapsto t^{\frac{1}{2}} \nabla u_j(t))_{j \in \N}$ Cauchy in $\mathrm{BC}([0,T); \Ell^3(\Omega; \C^9))$.
\end{proof}
